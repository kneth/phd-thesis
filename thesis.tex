%%%%%%%%%%%%%%%%%%%%%%%%%%%%%%%%%%%%%%%%%%%%%%%%%%%%%%%%%%%%%%%%%%%%%
%% Ph.D. thesis
%%
%% (C) Copyright 1996-1997 by Kenneth Geisshirt (kneth@chem.ruc.dk)
%% Dept. of Life Sciences and Chemistry, Roskilde University
%%
%% Last updated: 1 December 1997
%%%%%%%%%%%%%%%%%%%%%%%%%%%%%%%%%%%%%%%%%%%%%%%%%%%%%%%%%%%%%%%%%%%%%


\documentclass[a4paper,12pt,dvips]{book}

\usepackage[latin1]{inputenc}    % Keyboard
\usepackage[british]{babel}

\usepackage{cite}
\usepackage{times}
\usepackage{xspace}

\usepackage{fancyheadings}
\usepackage[Conny]{fncychap}
\usepackage[bf,footnotesize]{caption}

% Math and symbols
\usepackage{latexsym}
\usepackage{amsmath, amssymb}

\usepackage{algorithm, algorithmic}
\usepackage{epic, rotating}
\usepackage{epsfig, eepic, float}
\usepackage{subfigure, longtable}


%% Layout of a page
\pagestyle{fancy}
%\addtolength{\headwidth}{\marginparsep}
%\addtolength{\headwidth}{\marginparwidth}
\renewcommand{\chaptermark}[1]{\markboth{#1}{}}
\renewcommand{\sectionmark}[1]{\markright{\thesection\ #1}}
\lhead[\fancyplain{}{\bfseries\thepage}]
  {\fancyplain{}{\bfseries\rightmark}}
\rhead[\fancyplain{}{\bfseries\leftmark}]
  {\fancyplain{}{\bfseries\thepage}}
\cfoot{}



\setlength{\parskip}{0.1cm}
\setlength{\parindent}{0.0cm}

\begin{document}

%% Hyphenation
\hyphenation{phy-si-cal}
\hyphenation{tem-pe-ra-ture}
\hyphenation{com-pu-ters}
\hyphenation{coef-fi-cients}
\hyphenation{ima-gi-ne}
\hyphenation{si-mu-la-ted}
\hyphenation{stu-died}
\hyphenation{theo-ries}
\hyphenation{mo-le-cu-lar}
\hyphenation{ve-lo-ci-ties}
\hyphenation{si-mu-la-tions}
\hyphenation{con-si-de-ra-tion}
\hyphenation{cri-ti-cal}
\hyphenation{mea-ning}
\hyphenation{equi-li-brium}
\hyphenation{sy-stems}
\hyphenation{che-mi-stry}
\hyphenation{expe-ri-men-tal}
\hyphenation{theo-re-ti-cal}
\hyphenation{mi-cro-ca-no-ni-cal}
\hyphenation{nu-me-ri-cal}
\hyphenation{par-ti-cles}
\hyphenation{ener-gy}
\hyphenation{gra-nu-lar}
\hyphenation{ve-lo-ci-ty}
\hyphenation{dy-na-mics}
\hyphenation{du-ring}
\hyphenation{nu-me-ri-cal-ly}
\hyphenation{sta-tio-na-ry}
\hyphenation{bi-mo-le-cu-lar}
\hyphenation{sto-rage}
\hyphenation{che-mi-cal-ly}



%% Latin
\newcommand{\ie}[0]{\textit{i.e.,\xspace}}
\newcommand{\eg}[0]{\textit{e.g.,\xspace}}
\newcommand{\etc}[0]{\textit{etc.\@\xspace}}
\newcommand{\abinitio}[0]{\textit{ab initio}}
\newcommand{\cf}[0]{\textit{cf.\@\xspace}}
\newcommand{\etal}[0]{\textit{et al.\@\xspace}}

%% Useful mathematics
\renewcommand{\vec}[1]{{\mathbf{#1}}}
\newcommand{\diff}[2]{{\frac{\mathrm{d}#1}{\mathrm{d}#2}}}
\newcommand{\ddiff}[2]{{\frac{\mathrm{d}^2#1}{\mathrm{d}#2^2}}}
\newcommand{\pdiff}[2]{{\frac{\partial #1}{\partial #2}}}
\newcommand{\mtrx}[1]{{\mathbf{#1}}}
\newcommand{\half}[0]{{\scriptstyle\frac{1}{2}}}
\newcommand{\twothird}[0]{{\small\frac{2}{3}}}
\newcommand{\bigO}[1]{{\mathcal{O}(#1)}}
\newcommand{\smbox}[1]{\mbox{\footnotesize #1}}
\newcommand{\Tr}[0]{\mathrm{Tr}}

%% Units
\newcommand{\liter}{{\mathrm{l}}}           % litre
\newcommand{\GB}{{\mathrm{GB}}}             % Gigabyte
\newcommand{\M}{{\mathrm{\small M}}}        % mol/litre
\newcommand{\mol}{{\mathrm{mol}}}           % mol
\newcommand{\meter}{{\mathrm{m}}}           % meter

%% Theorem-like environments

%% Useful commands
\newcommand{\inputchap}[1]{\input{#1}\clearpage{\pagestyle{empty}\cleardoublepage}}

%% Title page
\inputchap{Frontpage/frontpage.tex}

%% Abstracts and preface
\inputchap{Abstract/english.tex}
\inputchap{Abstract/dansk.tex}   %% Translation of above
\inputchap{Preface/preface.tex}

%% Table of contents, figures, ...
\tableofcontents
\listoffigures
%\listofalgorithms
\pagestyle{empty}
\inputchap{symbols.tex}
\pagestyle{fancy}
%% Chapters - generals
\inputchap{MetaPhys/metaphys.tex}
\inputchap{Mechanics/mechanics.tex}
\inputchap{PhaseSep/phasesep.tex}
\inputchap{Kinetics/kinetics.tex}
\inputchap{Inelastic/inelastic.tex}
\inputchap{NumTech/numtech.tex}

%% Chapters - results
\inputchap{Lotka/extlotka.tex}
\inputchap{Granular/granular.tex}

%% Conclusion
\inputchap{Concl/conclusion.tex}

%% Appendices
%% change to appendices
\appendix
\inputchap{Appendix/A.tex}
\inputchap{Appendix/B.tex}
\inputchap{Appendix/C.tex}
\inputchap{Appendix/D.tex}


%% Bibliography
\addtocounter{page}{8}
\addcontentsline{toc}{chapter}{\numberline{}Bibliography}
\bibliographystyle{plain}
\bibliography{kinetics,csm,datalogi,matematik,numanal,fysik,kneth,statmech,granular,fyskem,meta,compsci}

\end{document}
