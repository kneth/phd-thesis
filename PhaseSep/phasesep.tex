%%%%%%%%%%%%%%%%%%%%%%%%%%%%%%%%%%%%%%%%%%%%%%%%%%%%%%%%%%%%%%
%% (C) Kenneth Geisshirt (kneth@chem.ruc.dk)
%% Last modified: 30 January 1998
%%%%%%%%%%%%%%%%%%%%%%%%%%%%%%%%%%%%%%%%%%%%%%%%%%%%%%%%%%%%%%

\chapter{Phase Separation}
\label{chap:PhaseSep}

Simple liquids have been studied by computer simulations for more than
forty years. The first pioneering studies by \eg Metropolis \etal
\cite{Metropolis53} were concerned with
the equation of state. But since the early 1960s the focus has been on
phase transitions.

In this chapter we will review some of the basic theories about phase
transitions with emphasis on phase separation. We will begin the
chapter by giving a few general references: Binder \cite{Binder} has
reviewed the kinetics of phase separation very pedagogically.
The best and most general textbook is the
textbook by Goldenfeld \cite{Goldenfeld92}.


\section{Phase transitions}
\label{sect:PhaseTrans}
Phase transitions are a well-known phenomenon in everyday life \eg the
melting of an ice cube. The properties of phase transitions have been
studied experimentally and theoretically for more than a century. An
early observation was that at some phase transitions the heat capacity
is singular at the transition temperature. This observation led
Erhenfest to a classification of phase transitions. The classification
proposed by Erhenfest is nowadays simplified, and we have only two
types of phase transitions: 

\begin{itemize}
  \item the discontinuous transition where the heat capacity is singular
  \item and the continuous transition where the heat capacity is
  a continuous function of the temperature
\end{itemize}

The melting of ice is an example of a discontinuous phase transition
while the structural changes in a lipid is typically a continuous
transition. 

Let us consider a pure substance \eg carbon dioxide,
$\mathrm{CO}_2$. Given the pressure, the transition temperature from
solid to liquid, is uniquely determined. The information \ie the
set of transition pressures and temperature, is called the
\textit{phase diagram}. For a pure substance like carbon dioxide a
typical phase diagram is sketched in figure \ref{fig:PhaseDiagram}.

\begin{figure}
  \begin{center}
    \input{PhaseSep/phasediag.tex}
  \end{center}
  \caption[Phase diagram of a pure substance]{A typical phase diagram
  for a pure substance. The letters S, L, and G denote one phase
  regions (solid, liquid, and gas).\label{fig:PhaseDiagram}} 
\end{figure}

The lines in a phase diagram is where the phase transition
occurs. These lines are called the coexistence lines because two phases
exist \ie they are in equilibrium. 

A coexistence line can be calculated by using the Clapeyron equation \ie 

\begin{equation}
  \diff{P}{T} =
  \frac{\Delta_{\smbox{trans}}S_m}{\Delta_{\smbox{trans}}V_m}
\end{equation}
where $P$ is the transition pressure, $T$ the temperature,
$\Delta_{\smbox{trans}}S_m$ is the change in the molar entropy, and
$\Delta_{\smbox{trans}}V_m$ is the change in the molar volume. 

The phase diagram sketched in figure \ref{fig:PhaseDiagram} is for a
pure substance only. If we turn to two-component mixtures, we need one
extra variable in order to describe the phase transitions
uniquely. Often the mole fraction of one of the species is used, and
the phase diagram is three-dimensional. 

For a pure substance there exists one point in the phase diagram where
the three phases (solid, liquid, gas) coexist. We call this point
the \textit{triple point}. It is located where the three coexistence
lines meet. 


\section{Simple liquids}
\label{sect:SimpleLiq}
One-component simple liquids have been studied experimentally,
computationally and theoretically. We will concentrate on theoretical
and computational results on one simple liquid: the Lennard-Jones liquid.

The Lennard-Jones (LJ) liquid can be regarded as a family of liquids \ie
many simple liquids can be approximated well using the Lennard-Jones
potential. The LJ potential is a short-range potential which includes
two parameters:

\begin{equation}
  u(r) = 4\epsilon \left[\left(\frac{\sigma}{r}\right)^{12} -
    \left(\frac{\sigma}{r}\right)^6 \right]
\end{equation}

The parameter $\epsilon$ is the fundamental energy unit while the
parameter $\sigma$ is the fundamental length unit. Table
\ref{tab:LJparams} lists values for the parameters for a number of
simple liquids. 

\begin{table}
  \begin{center}
  \begin{tabular}{lrr}
    \hline
    Substance       & $\sigma$ (nm) & $\epsilon/k_B$ (K) \\ \hline
    $\mathrm{He}$   & 0.2556        & 10.2               \\
    $\mathrm{Ar}$   & 3.405         & 119.8              \\
    $\mathrm{H}_2$  & 2.959         & 36.7               \\
    $\mathrm{CH}_4$ & 3.817         & 148.2              \\
    \hline
  \end{tabular}
  \end{center}
  \caption[Lennard-Jones parameters]{The Lennard-Jones parameters for a
    number of substances. Data taken from Hansen \etal
    \cite{Hansen86}.\label{tab:LJparams}} 
\end{table}

Hansen \etal \cite{Hansen69} have performed computer simulations of
a three-dimensional Lennard-Jones system, and they have found that the
triple point is $\rho \approx 0.85 \sigma^3$ and $T \approx 0.68
\epsilon/k_B$. Kofke \cite{Kofke93} and Panagiotopoulos \cite{Panagio87}
have numerically studied the liquid-gas transition, and
this part of the phase diagram is shown in figure \ref{fig:LJphase}.

\begin{figure}
  \begin{center}
    % GNUPLOT: LaTeX picture
\setlength{\unitlength}{0.240900pt}
\ifx\plotpoint\undefined\newsavebox{\plotpoint}\fi
\sbox{\plotpoint}{\rule[-0.200pt]{0.400pt}{0.400pt}}%
\begin{picture}(1500,900)(0,0)
\font\gnuplot=cmr10 at 10pt
\gnuplot
\sbox{\plotpoint}{\rule[-0.200pt]{0.400pt}{0.400pt}}%
\put(201.0,163.0){\rule[-0.200pt]{4.818pt}{0.400pt}}
\put(181,163){\makebox(0,0)[r]{0}}
\put(1460.0,163.0){\rule[-0.200pt]{4.818pt}{0.400pt}}
\put(201.0,262.0){\rule[-0.200pt]{4.818pt}{0.400pt}}
\put(181,262){\makebox(0,0)[r]{0.02}}
\put(1460.0,262.0){\rule[-0.200pt]{4.818pt}{0.400pt}}
\put(201.0,362.0){\rule[-0.200pt]{4.818pt}{0.400pt}}
\put(181,362){\makebox(0,0)[r]{0.04}}
\put(1460.0,362.0){\rule[-0.200pt]{4.818pt}{0.400pt}}
\put(201.0,461.0){\rule[-0.200pt]{4.818pt}{0.400pt}}
\put(181,461){\makebox(0,0)[r]{0.06}}
\put(1460.0,461.0){\rule[-0.200pt]{4.818pt}{0.400pt}}
\put(201.0,561.0){\rule[-0.200pt]{4.818pt}{0.400pt}}
\put(181,561){\makebox(0,0)[r]{0.08}}
\put(1460.0,561.0){\rule[-0.200pt]{4.818pt}{0.400pt}}
\put(201.0,660.0){\rule[-0.200pt]{4.818pt}{0.400pt}}
\put(181,660){\makebox(0,0)[r]{0.1}}
\put(1460.0,660.0){\rule[-0.200pt]{4.818pt}{0.400pt}}
\put(201.0,760.0){\rule[-0.200pt]{4.818pt}{0.400pt}}
\put(181,760){\makebox(0,0)[r]{0.12}}
\put(1460.0,760.0){\rule[-0.200pt]{4.818pt}{0.400pt}}
\put(201.0,859.0){\rule[-0.200pt]{4.818pt}{0.400pt}}
\put(181,859){\makebox(0,0)[r]{0.14}}
\put(1460.0,859.0){\rule[-0.200pt]{4.818pt}{0.400pt}}
\put(201.0,163.0){\rule[-0.200pt]{0.400pt}{4.818pt}}
\put(201,122){\makebox(0,0){0.7}}
\put(201.0,839.0){\rule[-0.200pt]{0.400pt}{4.818pt}}
\put(384.0,163.0){\rule[-0.200pt]{0.400pt}{4.818pt}}
\put(384,122){\makebox(0,0){0.8}}
\put(384.0,839.0){\rule[-0.200pt]{0.400pt}{4.818pt}}
\put(566.0,163.0){\rule[-0.200pt]{0.400pt}{4.818pt}}
\put(566,122){\makebox(0,0){0.9}}
\put(566.0,839.0){\rule[-0.200pt]{0.400pt}{4.818pt}}
\put(749.0,163.0){\rule[-0.200pt]{0.400pt}{4.818pt}}
\put(749,122){\makebox(0,0){1}}
\put(749.0,839.0){\rule[-0.200pt]{0.400pt}{4.818pt}}
\put(932.0,163.0){\rule[-0.200pt]{0.400pt}{4.818pt}}
\put(932,122){\makebox(0,0){1.1}}
\put(932.0,839.0){\rule[-0.200pt]{0.400pt}{4.818pt}}
\put(1115.0,163.0){\rule[-0.200pt]{0.400pt}{4.818pt}}
\put(1115,122){\makebox(0,0){1.2}}
\put(1115.0,839.0){\rule[-0.200pt]{0.400pt}{4.818pt}}
\put(1297.0,163.0){\rule[-0.200pt]{0.400pt}{4.818pt}}
\put(1297,122){\makebox(0,0){1.3}}
\put(1297.0,839.0){\rule[-0.200pt]{0.400pt}{4.818pt}}
\put(1480.0,163.0){\rule[-0.200pt]{0.400pt}{4.818pt}}
\put(1480,122){\makebox(0,0){1.4}}
\put(1480.0,839.0){\rule[-0.200pt]{0.400pt}{4.818pt}}
\put(201.0,163.0){\rule[-0.200pt]{308.111pt}{0.400pt}}
\put(1480.0,163.0){\rule[-0.200pt]{0.400pt}{167.666pt}}
\put(201.0,859.0){\rule[-0.200pt]{308.111pt}{0.400pt}}
\put(41,511){\makebox(0,0){$P$}}
\put(840,61){\makebox(0,0){$T$}}
\put(201.0,163.0){\rule[-0.200pt]{0.400pt}{167.666pt}}
\put(276,174){\usebox{\plotpoint}}
\multiput(276.00,174.59)(5.608,0.477){7}{\rule{4.180pt}{0.115pt}}
\multiput(276.00,173.17)(42.324,5.000){2}{\rule{2.090pt}{0.400pt}}
\multiput(327.00,179.59)(5.101,0.482){9}{\rule{3.900pt}{0.116pt}}
\multiput(327.00,178.17)(48.905,6.000){2}{\rule{1.950pt}{0.400pt}}
\multiput(384.00,185.59)(3.465,0.489){15}{\rule{2.767pt}{0.118pt}}
\multiput(384.00,184.17)(54.258,9.000){2}{\rule{1.383pt}{0.400pt}}
\multiput(444.00,194.58)(2.677,0.493){23}{\rule{2.192pt}{0.119pt}}
\multiput(444.00,193.17)(63.450,13.000){2}{\rule{1.096pt}{0.400pt}}
\multiput(512.00,207.58)(2.001,0.495){33}{\rule{1.678pt}{0.119pt}}
\multiput(512.00,206.17)(67.518,18.000){2}{\rule{0.839pt}{0.400pt}}
\multiput(583.00,225.58)(1.641,0.496){45}{\rule{1.400pt}{0.120pt}}
\multiput(583.00,224.17)(75.094,24.000){2}{\rule{0.700pt}{0.400pt}}
\multiput(661.00,249.58)(1.301,0.498){65}{\rule{1.135pt}{0.120pt}}
\multiput(661.00,248.17)(85.644,34.000){2}{\rule{0.568pt}{0.400pt}}
\multiput(749.00,283.58)(1.013,0.498){93}{\rule{0.908pt}{0.120pt}}
\multiput(749.00,282.17)(95.115,48.000){2}{\rule{0.454pt}{0.400pt}}
\multiput(846.00,331.58)(0.804,0.499){129}{\rule{0.742pt}{0.120pt}}
\multiput(846.00,330.17)(104.459,66.000){2}{\rule{0.371pt}{0.400pt}}
\multiput(952.00,397.58)(0.633,0.499){185}{\rule{0.606pt}{0.120pt}}
\multiput(952.00,396.17)(117.741,94.000){2}{\rule{0.303pt}{0.400pt}}
\multiput(1071.58,491.00)(0.497,0.620){47}{\rule{0.120pt}{0.596pt}}
\multiput(1070.17,491.00)(25.000,29.763){2}{\rule{0.400pt}{0.298pt}}
\multiput(1096.00,522.58)(0.537,0.497){49}{\rule{0.531pt}{0.120pt}}
\multiput(1096.00,521.17)(26.898,26.000){2}{\rule{0.265pt}{0.400pt}}
\multiput(1124.58,548.00)(0.497,0.517){51}{\rule{0.120pt}{0.515pt}}
\multiput(1123.17,548.00)(27.000,26.931){2}{\rule{0.400pt}{0.257pt}}
\multiput(1151.58,576.00)(0.497,0.517){53}{\rule{0.120pt}{0.514pt}}
\multiput(1150.17,576.00)(28.000,27.933){2}{\rule{0.400pt}{0.257pt}}
\multiput(1179.58,605.00)(0.497,0.574){51}{\rule{0.120pt}{0.559pt}}
\multiput(1178.17,605.00)(27.000,29.839){2}{\rule{0.400pt}{0.280pt}}
\multiput(1206.58,636.00)(0.497,0.586){55}{\rule{0.120pt}{0.569pt}}
\multiput(1205.17,636.00)(29.000,32.819){2}{\rule{0.400pt}{0.284pt}}
\multiput(1235.58,670.00)(0.497,0.638){55}{\rule{0.120pt}{0.610pt}}
\multiput(1234.17,670.00)(29.000,35.733){2}{\rule{0.400pt}{0.305pt}}
\multiput(1264.58,707.00)(0.497,0.580){59}{\rule{0.120pt}{0.565pt}}
\multiput(1263.17,707.00)(31.000,34.828){2}{\rule{0.400pt}{0.282pt}}
\multiput(1295.58,743.00)(0.497,0.562){61}{\rule{0.120pt}{0.550pt}}
\multiput(1294.17,743.00)(32.000,34.858){2}{\rule{0.400pt}{0.275pt}}
\end{picture}

  \end{center}
  \caption[Phase diagram for three-dimensional LJ liquid]{The phase
   diagram of the pure three-dimensional Lennard-Jones system (temperature
   versus pressure; scaled so $\epsilon = 1$ and $\sigma = 1$). The gas phase
   is above the curve while the liquid phase exists below the curve. From
   \cite{Kofke93}.\label{fig:LJphase}} 
\end{figure}


\section{The scaling hypothesis}
\label{sect:scale}
In this section we will briefly discuss scaling in the context of phase
separation, for a more detailed description see Bray \cite{Bray94}. The
physics behind the scaling hypothesis is that only one variable is
relevant \cite{Bray94, Goldenfeld92}.

If we consider a fluid close to the critical point, we find that the
isothermal compressibility $\kappa_T$ behave as

\begin{equation}
  \kappa_T = - \frac{1}{V} \left(\pdiff{V}{p}\right)_T
     \sim |T - T_c|^{-\gamma}
\end{equation}
were $T_c$ is the critical temperature. Moreover, the a similar
behaviour is found for the heat capacity, namely

\begin{equation}
  C_V \sim |T - T_c|^{-\alpha}
\end{equation}

The scaling described above is the so called static scaling since there
is no temporal dependency. A large number of systems has the same value
for the exponents even though the system might be of very different
nature. This fact has lead the physicists to associate each value with
a universality class.


\section{Phase separation}
\label{sect:PhaseSep}
The basic phenomenology of phase separation is simple \cite{Goldenfeld92}.
Imagine that we
have a mixture of two substances, $A$ and $B$. At high temperature the
two substances will be miscible, and at low temperature they will be
immiscible. The transition from miscible to immiscible occurs at a
well-defined temperature, $T_C$.

Consider a mixture of two simple liquids; figure \ref{fig:SpinodalDiag}
outlines the general phase diagram. Outside the solid line the two
liquids are miscible, and the line is the coexistence line. Between the
solid and the dashed line, the system will phase separate through a
nucleation process, and inside the dashed line we will see a spinodal
decomposition. 

\begin{figure}
  \begin{center}
    \input{PhaseSep/spinodal.eepic}
  \end{center}
  \caption[Phase diagram of a mixture]{A phase diagram for a mixture of
  two simple liquids. The solid line is the coexistence line while the
  dashed line is the the spinodal line.\label{fig:SpinodalDiag}}
\end{figure}

\subsection{Nucleation}
As already mentioned, phase separation through a
nucleation occurs between the solid lines and the dashed lines in
figure \ref{fig:SpinodalDiag}. Thermal fluctuations form small droplets
in a homogeneous phase. The free energy change $\Delta F$, of forming a
droplet of radius $R$ is

\begin{equation}
  \Delta F =
    \begin{cases}
       4\pi \sigma R^2 - \frac{4}{3}\pi \epsilon R^3 &\text{in three
                    dimensions} \\
       2\pi \sigma R - \pi \epsilon R^2 &\text{in two dimensions}
    \end{cases}
\end{equation}
where $\epsilon$ is the free energy of the bulk and $\sigma$ is the
surface free energy. The critical size, $R_c$, of a droplet is the
maximum in the free energy. When the droplet is larger than the
critical droplet, the droplet will grow. We can find the critical
radius by differentiating the free energy and find the root. We find:

\begin{equation}
  R_c =
  \begin{cases}
    \frac{2\sigma}{\epsilon} &\text{three dimensions} \\
    \frac{\sigma}{\epsilon}  &\text{two dimensions}
  \end{cases}
\end{equation}


\subsection{Spinodal decomposition}
\label{sect:CahnHilliard}
Through a spinodal decomposition the system will phase separate by
forming a labyrinth structure which coarsen. The kinetics of the spinodal
decomposition was first investigated by Cahn \etal \cite{Cahn58}, and
the theory now goes under the name \textit{Cahn-Hilliard theory}. 

Let us define an order parameter, $\psi$, as $\psi(\vec{r}, t) \equiv
c(\vec{r}, t) - c_0$ where $c(\vec{r}, t)$ is the local concentration,
and $c_0$ is the equilibrium concentration. Moreover, let $f$ denotes
the free energy per molecule. The free total energy $F$ in a volume $V$ is

\begin{equation}
\label{eq:CHintegral}
  F = \int_V f \mathrm{d}V
\end{equation}

In order to evaluate this integral, we can Taylor expand the free
energy at $c_0$ ($\psi = 0$); the free energy of a spatially homogeneous system is
$f_0$. Moreover, $f$ does also depend on $\nabla c$, $\nabla^2 c$ and higher
order derivatives. The Taylor expansion to first order in $c$, $\nabla c$ and
$\nabla^2 c$ is

\begin{equation}
\label{eq:CHTaylor}
  f = f_0 + \sum_i L_i \pdiff{\psi}{x_i}
          + \sum_{ij} \kappa_{ij}^{(1)}
               \frac{\partial^2 \psi}{\partial x_i \partial x_j}
          + \half \sum_{ij} \kappa_{ij}^{(2)} \pdiff{\psi}{x_i}\pdiff{\psi}{x_j}
          + \cdots
\end{equation}
where $x_i$ and $x_j$ represent the spatial variables and

\begin{subequations}
  \begin{eqnarray}
    L_i &=& \left. \pdiff{f}{(\partial \psi/\partial x_i)}\right|_{\psi=0} \\
    \kappa_{ij}^{(1)} &=& \left.
           \pdiff{f}{(\partial^2 \psi/\partial x_i\partial x_j)}
              \right|_{\psi=0} \\
    \kappa_{ij}^{(2)} &=& \left.
           \frac{\partial^2 f}{\partial (\partial \psi/\partial x_i)
                   (\partial \psi/\partial x_j)} \right|_{\psi=0}
  \end{eqnarray}
\end{subequations}

Since we will assume that the mixture is isotropic, we will have

\begin{subequations}
  \begin{eqnarray}
    L_i &=& 0 \\
    \kappa_{ij}^{(1)} &=&
    \begin{cases}
       \kappa_1 = \pdiff{f}{\nabla^2 \psi} & \text{for $i=j$} \\
       0                                & \text{otherwise}
    \end{cases} \\
    \kappa_{ij}^{(2)} &=&
    \begin{cases}
       \kappa_2 = \frac{\partial^2 f}{\partial(|\nabla\psi|)^2} & \text{for $i=j$} \\
       0 & \text{otherwise}
    \end{cases}
  \end{eqnarray}
\end{subequations}

This reduces the expansion of $f$ to

\begin{equation}
  f = f_0 + \kappa_1 \nabla^2 \psi + \kappa_2 (\nabla \psi)^2
\end{equation}

By applying the divergence theorem to equation \eqref{eq:CHintegral}
and inserting equation \eqref{eq:CHTaylor}, we obtain the Cahn-Hilliard
free energy functional

\begin{equation}
\label{eq:CHfreeEnergy}
  F[\psi(\vec{r})] = \int \half (\nabla\psi)^2 + k f_0[\psi(\vec{r})]
                     \mathrm{d}\vec{r}
\end{equation}
where

\[
  k = - \left. \frac{\partial^2 f}{\partial \psi\partial\nabla^2 \psi}
             \right|_{\psi=0}
           + \left. \frac{\partial^2 f}{(\partial|\nabla\psi|)^2}
             \right|_{\psi=0}
\]

The free energy $f$ has the following properties. It has only one
minimum above the critical temperature, and it is located at $\psi =
0$ \ie above the critical temperature, the system can only be in a
homogeneous state. Below the critical temperature, the free energy has
two minima which are not located at $\psi = 0$. The free energy $f_0$ is
often assumed to be a Landau free energy near the critical temperature
\ie a free energy in the form

\[
  f_0[\psi] = f_0(T) + a(T)\psi^2 + b(T)\psi^4 + \cdots
\]

Even though the system phase separate, the mass is conserved, which is
the same as

\begin{equation}
\label{eq:MassConserve}
  \pdiff{\psi}{t} + \nabla\cdot \vec{j} = 0
\end{equation}
where $\vec{j}$ is the flux defined as

\begin{equation}
\label{eq:Mobil}
  \vec{j} = -M \nabla \frac{\delta F}{\delta \psi}
\end{equation}

The parameter $M$ is a measure of the mobility of the species (similar to
diffusion), and $\delta F/\delta \psi$ is the functional derivative of
$F$ with respect to $\psi$. Putting equations \eqref{eq:CHfreeEnergy},
\eqref{eq:MassConserve} and \eqref{eq:Mobil} together we obtain the
Cahn-Hilliard equation \ie

\begin{equation}
\label{eq:CahnHilliard}
  \pdiff{\psi}{t} = M\nabla^2 \left(\frac{\delta f_0}{\delta\psi} -
                    \nabla^2\psi\right)
\end{equation}

Initial conditions are the homogeneous state \ie a state where $\langle
\psi \rangle = 0$. During the spinodal decomposition patterns are
formed, and the final pattern is the equilibrium pattern \ie phases
which are pure in one of the species.

The Cahn-Hilliard equation cannot be solved analytically, but only
numerically. In section \ref{sect:PhaseSepChem} we will discuss how the
Cahn-Hilliard equation can be extended in order to include chemical
reactions.


\subsection{Dynamic scaling}
Now consider the phase separation process, and let $R(t)$ denote the
average domain size. As early as 1961, Lifshitz, Slyozov and Wagner
were able to predict that the growth law is algebraic \cite{Bray94} \ie

\begin{equation}
\label{eq:LSW}
  R(t) \sim t^{1/3}
\end{equation}

This growth law is called the diffusive regime, because the underlying
physical model is diffusion-controlled reaction \ie two domains diffuse
together and merge in order to form one large domain. At later times in
the phase separation process the growth law can cross over to
\cite{Bray94}

\begin{equation}
  R(t) \sim t^{2/3}
\end{equation}

The growth law predicted by Lifshitz \etal (equation \eqref{eq:LSW}) is
not the only possible growth law. Equation \eqref{eq:LSW} is typically
denoted the diffusive regime which the underlying model is a
diffusion-controlled reaction. Other regimes include the viscous and
the inertial regime.


\subsection{The Lennard-Jones liquid}
The mixture of Lennard-Jones liquids is a good system to study when we
wish to understand the kinetics of phase separation.

The phase diagram of a binary Lennard-Jones fluid has been published by
Toxv{\ae}rd \etal \cite{Toxvaerd95a, Velasco93}. The phase diagram is
shown in figure \ref{fig:PhaseLJmix}

\begin{figure}
%\centering
    \mbox{\subfigure[2 dimensions]{% GNUPLOT: LaTeX picture
\setlength{\unitlength}{0.240900pt}
\ifx\plotpoint\undefined\newsavebox{\plotpoint}\fi
\sbox{\plotpoint}{\rule[-0.200pt]{0.400pt}{0.400pt}}%
\begin{picture}(675,675)(0,0)
\font\gnuplot=cmr10 at 10pt
\gnuplot
\sbox{\plotpoint}{\rule[-0.200pt]{0.400pt}{0.400pt}}%
\put(181.0,163.0){\rule[-0.200pt]{4.818pt}{0.400pt}}
\put(161,163){\makebox(0,0)[r]{1}}
\put(632.0,163.0){\rule[-0.200pt]{4.818pt}{0.400pt}}
\put(181.0,257.0){\rule[-0.200pt]{4.818pt}{0.400pt}}
\put(161,257){\makebox(0,0)[r]{1.2}}
\put(632.0,257.0){\rule[-0.200pt]{4.818pt}{0.400pt}}
\put(181.0,351.0){\rule[-0.200pt]{4.818pt}{0.400pt}}
\put(161,351){\makebox(0,0)[r]{1.4}}
\put(632.0,351.0){\rule[-0.200pt]{4.818pt}{0.400pt}}
\put(181.0,446.0){\rule[-0.200pt]{4.818pt}{0.400pt}}
\put(161,446){\makebox(0,0)[r]{1.6}}
\put(632.0,446.0){\rule[-0.200pt]{4.818pt}{0.400pt}}
\put(181.0,540.0){\rule[-0.200pt]{4.818pt}{0.400pt}}
\put(161,540){\makebox(0,0)[r]{1.8}}
\put(632.0,540.0){\rule[-0.200pt]{4.818pt}{0.400pt}}
\put(181.0,634.0){\rule[-0.200pt]{4.818pt}{0.400pt}}
\put(161,634){\makebox(0,0)[r]{2}}
\put(632.0,634.0){\rule[-0.200pt]{4.818pt}{0.400pt}}
\put(264.0,163.0){\rule[-0.200pt]{0.400pt}{4.818pt}}
\put(264,122){\makebox(0,0){0.2}}
\put(264.0,614.0){\rule[-0.200pt]{0.400pt}{4.818pt}}
\put(361.0,163.0){\rule[-0.200pt]{0.400pt}{4.818pt}}
\put(361,122){\makebox(0,0){0.4}}
\put(361.0,614.0){\rule[-0.200pt]{0.400pt}{4.818pt}}
\put(458.0,163.0){\rule[-0.200pt]{0.400pt}{4.818pt}}
\put(458,122){\makebox(0,0){0.6}}
\put(458.0,614.0){\rule[-0.200pt]{0.400pt}{4.818pt}}
\put(555.0,163.0){\rule[-0.200pt]{0.400pt}{4.818pt}}
\put(555,122){\makebox(0,0){0.8}}
\put(555.0,614.0){\rule[-0.200pt]{0.400pt}{4.818pt}}
\put(652.0,163.0){\rule[-0.200pt]{0.400pt}{4.818pt}}
\put(652,122){\makebox(0,0){1}}
\put(652.0,614.0){\rule[-0.200pt]{0.400pt}{4.818pt}}
\put(181.0,163.0){\rule[-0.200pt]{113.464pt}{0.400pt}}
\put(652.0,163.0){\rule[-0.200pt]{0.400pt}{113.464pt}}
\put(181.0,634.0){\rule[-0.200pt]{113.464pt}{0.400pt}}
\put(41,398){\makebox(0,0){$T$}}
\put(416,61){\makebox(0,0){$x_A$}}
\put(181.0,163.0){\rule[-0.200pt]{0.400pt}{113.464pt}}
\put(181,163){\raisebox{-.8pt}{\makebox(0,0){$\Diamond$}}}
\put(186,304){\raisebox{-.8pt}{\makebox(0,0){$\Diamond$}}}
\put(191,351){\raisebox{-.8pt}{\makebox(0,0){$\Diamond$}}}
\put(205,399){\raisebox{-.8pt}{\makebox(0,0){$\Diamond$}}}
\put(237,446){\raisebox{-.8pt}{\makebox(0,0){$\Diamond$}}}
\put(324,493){\raisebox{-.8pt}{\makebox(0,0){$\Diamond$}}}
\put(494,493){\raisebox{-.8pt}{\makebox(0,0){$\Diamond$}}}
\put(582,446){\raisebox{-.8pt}{\makebox(0,0){$\Diamond$}}}
\put(613,399){\raisebox{-.8pt}{\makebox(0,0){$\Diamond$}}}
\put(628,351){\raisebox{-.8pt}{\makebox(0,0){$\Diamond$}}}
\put(633,304){\raisebox{-.8pt}{\makebox(0,0){$\Diamond$}}}
\put(637,163){\raisebox{-.8pt}{\makebox(0,0){$\Diamond$}}}
\end{picture}
} \quad
          \subfigure[3 dimensions]{% GNUPLOT: LaTeX picture
\setlength{\unitlength}{0.240900pt}
\ifx\plotpoint\undefined\newsavebox{\plotpoint}\fi
\sbox{\plotpoint}{\rule[-0.200pt]{0.400pt}{0.400pt}}%
\begin{picture}(675,675)(0,0)
\font\gnuplot=cmr10 at 10pt
\gnuplot
\sbox{\plotpoint}{\rule[-0.200pt]{0.400pt}{0.400pt}}%
\put(181.0,163.0){\rule[-0.200pt]{4.818pt}{0.400pt}}
\put(161,163){\makebox(0,0)[r]{2}}
\put(632.0,163.0){\rule[-0.200pt]{4.818pt}{0.400pt}}
\put(181.0,242.0){\rule[-0.200pt]{4.818pt}{0.400pt}}
\put(161,242){\makebox(0,0)[r]{2.5}}
\put(632.0,242.0){\rule[-0.200pt]{4.818pt}{0.400pt}}
\put(181.0,320.0){\rule[-0.200pt]{4.818pt}{0.400pt}}
\put(161,320){\makebox(0,0)[r]{3}}
\put(632.0,320.0){\rule[-0.200pt]{4.818pt}{0.400pt}}
\put(181.0,399.0){\rule[-0.200pt]{4.818pt}{0.400pt}}
\put(161,399){\makebox(0,0)[r]{3.5}}
\put(632.0,399.0){\rule[-0.200pt]{4.818pt}{0.400pt}}
\put(181.0,477.0){\rule[-0.200pt]{4.818pt}{0.400pt}}
\put(161,477){\makebox(0,0)[r]{4}}
\put(632.0,477.0){\rule[-0.200pt]{4.818pt}{0.400pt}}
\put(181.0,556.0){\rule[-0.200pt]{4.818pt}{0.400pt}}
\put(161,556){\makebox(0,0)[r]{4.5}}
\put(632.0,556.0){\rule[-0.200pt]{4.818pt}{0.400pt}}
\put(181.0,634.0){\rule[-0.200pt]{4.818pt}{0.400pt}}
\put(161,634){\makebox(0,0)[r]{5}}
\put(632.0,634.0){\rule[-0.200pt]{4.818pt}{0.400pt}}
\put(271.0,163.0){\rule[-0.200pt]{0.400pt}{4.818pt}}
\put(271,122){\makebox(0,0){0.2}}
\put(271.0,614.0){\rule[-0.200pt]{0.400pt}{4.818pt}}
\put(367.0,163.0){\rule[-0.200pt]{0.400pt}{4.818pt}}
\put(367,122){\makebox(0,0){0.4}}
\put(367.0,614.0){\rule[-0.200pt]{0.400pt}{4.818pt}}
\put(462.0,163.0){\rule[-0.200pt]{0.400pt}{4.818pt}}
\put(462,122){\makebox(0,0){0.6}}
\put(462.0,614.0){\rule[-0.200pt]{0.400pt}{4.818pt}}
\put(557.0,163.0){\rule[-0.200pt]{0.400pt}{4.818pt}}
\put(557,122){\makebox(0,0){0.8}}
\put(557.0,614.0){\rule[-0.200pt]{0.400pt}{4.818pt}}
\put(652.0,163.0){\rule[-0.200pt]{0.400pt}{4.818pt}}
\put(652,122){\makebox(0,0){1}}
\put(652.0,614.0){\rule[-0.200pt]{0.400pt}{4.818pt}}
\put(181.0,163.0){\rule[-0.200pt]{113.464pt}{0.400pt}}
\put(652.0,163.0){\rule[-0.200pt]{0.400pt}{113.464pt}}
\put(181.0,634.0){\rule[-0.200pt]{113.464pt}{0.400pt}}
\put(41,398){\makebox(0,0){$T$}}
\put(416,61){\makebox(0,0){$x_A$}}
\put(181.0,163.0){\rule[-0.200pt]{0.400pt}{113.464pt}}
\put(181,163){\raisebox{-.8pt}{\makebox(0,0){$\Diamond$}}}
\put(188,242){\raisebox{-.8pt}{\makebox(0,0){$\Diamond$}}}
\put(195,320){\raisebox{-.8pt}{\makebox(0,0){$\Diamond$}}}
\put(200,383){\raisebox{-.8pt}{\makebox(0,0){$\Diamond$}}}
\put(224,477){\raisebox{-.8pt}{\makebox(0,0){$\Diamond$}}}
\put(259,556){\raisebox{-.8pt}{\makebox(0,0){$\Diamond$}}}
\put(283,571){\raisebox{-.8pt}{\makebox(0,0){$\Diamond$}}}
\put(545,571){\raisebox{-.8pt}{\makebox(0,0){$\Diamond$}}}
\put(569,556){\raisebox{-.8pt}{\makebox(0,0){$\Diamond$}}}
\put(604,477){\raisebox{-.8pt}{\makebox(0,0){$\Diamond$}}}
\put(628,383){\raisebox{-.8pt}{\makebox(0,0){$\Diamond$}}}
\put(633,320){\raisebox{-.8pt}{\makebox(0,0){$\Diamond$}}}
\put(640,242){\raisebox{-.8pt}{\makebox(0,0){$\Diamond$}}}
\put(647,163){\raisebox{-.8pt}{\makebox(0,0){$\Diamond$}}}
\end{picture}
}}
  \caption[Phase diagram of LJ mixture]{The phase diagram of a binary
  mixture of two- and three-dimensional Lennard-Jones liquids. The data
  point are found using MD simulation. The density is
  $0.8\sigma^2$ and $0.8\sigma^3$ in the two cases.
  From \cite{Toxvaerd95a}.\label{fig:PhaseLJmix}}
\end{figure}

Furthermore, Toxv{\ae}rd \etal show that the growth of the domains at late
times is algebraic. They find the exponent to be $2/3$ when the
particle fraction of each component is $\half$ while the exponent is
$1/3$ when one of the components is dominating.

Going to three-component systems, we see a major difference. Laradji
\etal \cite{Laradji96} find that the growth exponent is $1/3$ for a
system where the particle fraction of each component is equal.


\section{Chemistry}
\label{sect:PhaseSepChem}
The phase separation discussed in the previous section was a physical
process. During the 1990s a number of papers have been published on
how chemical reactions might influence the phase separation process.

The Cahn-Hilliard equation \eqref{eq:CahnHilliard} can be modified in
order to include chemical reactions. Christensen \etal
\cite{Christensen96} and Glozter \etal \cite{Glotzer95} have simple
modifications. Consider the reaction $A \leftrightarrows B$. The
forward and the reverse rate constants are denoted $k_f$ and $k_r$,
respectively. The modified Cahn-Hilliard equation including these two
reactions is

\begin{equation}
  \pdiff{\psi}{t} = M\nabla^2 \left(\frac{\delta f_0}{\delta\psi} -
                    \nabla^2\psi\right) - k_f\psi + k_r (1-\psi)
\end{equation}
where $\psi$ is the local mole fraction of $A$. Both Christensen \etal
and Glozter \etal set $k = k_f = k_r$. The modification of the
Cahn-Hilliard equation is simple: we have added the velocity field coming
from the chemical reactions.

The average domain size $R$ scales with the time as

\begin{equation}
  R(t) \sim t^{\alpha}
\end{equation}

Without the competing chemical reactions, the exponent $\alpha$ is
initially $1/3$ and $2/3$ at late times \cite{Farrell91, Wu95}. From a
dimensional analysis Glotzer \etal \cite{Glotzer95} find that the
average domain size scales as 

\begin{equation}
  R \sim \left( \frac{1}{k} \right)^{\alpha}
\end{equation}

At low reaction rates ($k \approx 0$) Christensen \etal and Glotzer
\etal finds that the exponent $\alpha$ is approximately $1/3$. At
larger reaction rates Christensen \etal find that the exponent is
$1/4$.

Monte Carlo simulations of an Ising model with competing reactions ($A
\leftrightarrows B$) have been performed by Glotzer \etal
\cite{Glotzer94}. They find that the exponent $\alpha = 0.22 \pm 0.02$
which is consistent with the result by Christensen \etal. Toxv{\ae}rd,
on the other hand, has performed Molecular Dynamics simulations of a
Lennard-Jones mixture undergoing phase separation competing with
chemical reactions. At low reaction rates the growth exponent is
approximately $0.33$ while it is lower at higher reaction rate.

Glozter \etal and Toxv{\ae}rd observe that the phase separation process
is hindered by fast reactions. Toxv{\ae}rd argues that the reason is
that the fast reactions destroy the hydrodynamic modes which are
essential in the phase separation. A similar observation has been done
by Verdasca \etal \cite{Verdasca95} and Christensen \etal
\cite{Christensen96}. They see that chemical reactions might freeze
the phase separation. This freezing is only observed at high reaction
rates.

Carati \etal \cite{Carati97} have recently published a paper on the
chemical freezing. The theory presented by Carati \etal is based on the
Cahn-Hilliard theory. They are able to establish a number of criteria
for the freezing \eg that at least one reaction must be autocatalytic.
