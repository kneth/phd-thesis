%%%%%%%%%%%%%%%%%%%%%%%%%%%%%%%%%%%%%%%%%%%%%%%%%%%%%%%%%%%%%%%
%% Abstract in Dansk
%% (C) Copyright 1997 by Kenneth Geisshirt (kneth@chem.ruc.dk)
%% Last modified: 1 December 1997
%%%%%%%%%%%%%%%%%%%%%%%%%%%%%%%%%%%%%%%%%%%%%%%%%%%%%%%%%%%%%%%

\chapter*{Resum\'{e}}
\addcontentsline{toc}{chapter}{\numberline{}Dansk resum\'{e}}

Afhandlingen er en unders{\o}gelse af ikke-ligev{\ae}gtsf{\ae}nomener.
Den brugte metode er Molecular Dynamics som i den simpleste formulering
er en numerisk l{\o}sning af de klassiske bev{\ae}gelsesligninger. To
systemer er blevet unders{\o}gt ved hj{\ae}lp af denne
simuleringsmetode.

Det f{\o}rste system er en oscillerende reaktion i konkurrence med en
faseadskillelsesproces. Den oscillerende reaktion er holdt langt fra
ligev{\ae}gt ved at drive den energetisk. Det vises at Molecular
Dynamics er i stand til at reproducere de makroskopiske f{\ae}nomener.
Endvidere vises det, at faseadskillelsesprocessen kan {\ae}ndre
reaktionens underliggende kinetik.

Det andet system er en dissipativ gas koblet til en r{\ae}kke
termostaterende anordninger. Den dissipative gas er en model af et
granul{\ae}rt medium, og arbejdet pr{\ae}senteret i afhandlingen kaster
lys p{\aa} m{\o}nsterdannelsesprocessen i dissipative gasser.