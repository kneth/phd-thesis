%%%%%%%%%%%%%%%%%%%%%%%%%%%%%%%%%%%%%%%%%%%%%%%%%%%%%%%%%%%%%%%
%% Abstract in English
%% (C) Copyright 1997 by Kenneth Geisshirt (kneth@chem.ruc.dk)
%% Last modified: 8 January 1998
%%%%%%%%%%%%%%%%%%%%%%%%%%%%%%%%%%%%%%%%%%%%%%%%%%%%%%%%%%%%%%%

\chapter*{Abstract}
\pagenumbering{roman}
\addcontentsline{toc}{chapter}{\numberline{}English abstract}

The thesis is an investigation of non-equilibrium phenonema. The method
imployed is Molecular Dynamics which in its simple form is a numerical
solution of the classical equations of motion. Two systems are studied
by this simulational method.

The first system is an oscillating chemical reaction competing with a
phase separation process. The oscillating chemical reaction is kept far
from equilibrium by driving it energetically. It is shown that the
Molecular Dynamics method is able to reproduce the macroscopic
phenomena. Moreover, it is shown that the phase separation process
can modify the underlying chemical kinetics of the reaction.

The second system is a dissipative gas coupled to a number of
thermostating devices. The dissipative gas is a model of a granular
medium, and work presented in the thesis cast light on the pattern
formation in dissipative gases.
