%%%%%%%%%%%%%%%%%%%%%%%%%%%%%%%%%%%%%%%%%%%%%%%%%%%%%%%%%%%%%%%%%%%
%% Inelastic collisions
%% (C) Copyright 1997 by Kenneth Geisshirt (kneth@chem.ruc.dk)
%% Last modified: 8 January 1998
%%%%%%%%%%%%%%%%%%%%%%%%%%%%%%%%%%%%%%%%%%%%%%%%%%%%%%%%%%%%%%%%%%%

\chapter{Inelastic Collisions}
\label{chap:Inelas}
Systems consisting of particles undergoing inelastic collisions do
have a phenomenology of their own. Inelastic collisions are often used
in models of granular media, see \eg Jaeger \etal \cite{Jaeger96}. This
chapter will discuss the physical properties of granular media and how
inelastic collisions can be modelled. In chapter \ref{chap:thermostat}
simulational results obtained by the author of this thesis and his
collaborators can be found.

\section{Hard particles}
\label{sect:HardParticles}
Hard or rigid particles move, in the absence of an external field, on
straight lines between two collisions. The interaction between
the particles are through the collisions only. At the collision of two
particles, $i$ and $j$, the velocity of each particle is changed
instantaneously. Let $\vec{v}_i$ and $\vec{v}_j$ denote the velocity
before collision, and $\vec{v}_i^{\prime}$ and $\vec{v}_j^{\prime}$
denote the velocity after the collision of particle $i$ and $j$,
respectively. They are related as:

\begin{subequations}
  \begin{eqnarray}
    \vec{v}_i^{\prime} &=& \epsilon \vec{v}_i + (1-\epsilon) \vec{v}_j
    \\
    \vec{v}_j^{\prime} &=& (1-\epsilon) \vec{v}_i + \epsilon \vec{v}_j
  \end{eqnarray}
\end{subequations}
where $\epsilon$ is a measure of inelasticity and is related to the
coefficient of restitution $r$, as $\epsilon = \half (1-r)$. In the case
of elastic collisions the value of $\epsilon$ is $0$, and completely
sticky collisions $\epsilon = \half$. 


\section{Soft particles}
\label{sect:SoftParticles}
The rigid particles which were discussed in the previous section move
in straight lines between two collisions. When we turn to soft
particles, it is a completely different story. The particles move in a
potential field and collisions are not well-defined. 

Models for granular media using soft particles have recently been
proposed independently by Brilliantov \etal \cite{Brilliantov96} and
Morgado \etal \cite{Morgado97}. The purpose here is not to review the
complete work by Brilliantov \etal and Morgado \etal but merely to
state a simple model which we will use in chapter
\ref{chap:thermostat}. Our model is only usable in one dimension.

First, let us define what a collision is for soft particles. Consider
two particles with separation $r$. We will say that the two particles
are colliding when the separation is less that a given value
$r_{\smbox{coll}}$ \ie when $r < r_{\smbox{coll}}$. For soft particles,
collisions have a duration \ie the collisions are not instantaneous
as collisions of rigid particles.

Elastic colliding particles move according to given equations of motion \ie
equations in the form:

\begin{subequations}
  \begin{eqnarray}
    \diff{\vec{r}_i}{t} &=& \vec{v}_i \\
    \diff{\vec{v}_i}{t} &=& \frac{\vec{F}_i}{m_i}
  \end{eqnarray}
\end{subequations}
where $\vec{r}_i$ is the position of particle $i$, $\vec{v}_i$ the
velocity, $m_i$ the mass, and $\vec{F}_i$ the force. We will modify
the equations of motion so that they include a dissipative term which
accounts for the inelastic collisions. The new equations of motion are:

\begin{subequations}
  \begin{eqnarray}
    \diff{\vec{r}_i}{t} &=& \vec{v}_i \\
    \diff{\vec{v}_i}{t} &=& \frac{1}{m_i}\left(\vec{F}_i - \vec{F}_i^{(\smbox{coll})}\right) 
  \end{eqnarray}
\end{subequations}
where $\vec{F}_i^{(\smbox{coll})}$ is the dissipative force due to the
collisions. The force is given by

\begin{equation}
  \vec{F}_i^{(\smbox{coll})} = \sum_{j\not= i}
  \vec{F}_{ij}^{(\smbox{coll})}
\end{equation}
with

\begin{equation}
  \vec{F}_{ij}^{(\smbox{coll})} = 
  \begin{cases}
    - \gamma v_{ij}\sqrt{r_{\smbox{coll}}-r_{ij}} & \text{for
      $r<r_{\smbox{coll}}$} \\
    0 & \text{otherwise}
  \end{cases}
\end{equation}
where $r_{ij} = \|\vec{r}_i - \vec{r}_j\|$ and $v_{ij} = \|\vec{v}_i -
\vec{v}_j\|$. The parameter $\gamma$ controls the degree of
inelasticity and when $\gamma = 0$ the collisions are elastic.


\section{The cooling problem}
\label{sect:CoolProblem}
Haff \cite{Haff83} and McNamara \etal \cite{McNamara93a} have
investigated the problem of rigid particles undergoing inelastic
collisions. At every collision kinetic energy dissipates from the
system. The (granular) temperature is closely related to the kinetic
energy, namely as

\begin{equation}
  \mathcal{T} \equiv \frac{1}{N} \sum_{i=1}^N v_i^2 = \frac{2}{m} E_{\smbox{kin}}
\end{equation}

The definition of the temperature as given by the equation above is
not correct according to the statistical-mechanical relation. The
deviation is the factor $1/N$. In statistical mechanics one would find
that it should be the degrees of freedom which for a one-dimensional
system in equilibrium is $N-1$. The definition above is adopted of two
reasons. First, it has previously been used in the literature and
second, we wish to compare the temperature of simulations of both
equilibrium and non-equilibrium systems in chapter
\ref{chap:thermostat}.

As the energy dissipates the temperature is decreasing. Naively
one could imagine that the temperature will decay exponentially but
the collision rate (collisions per unit time) does depend on the
temperature. After one collision the collision rate is slightly lower
and the time until the next collision is larger. This physical
behaviour leads to the following cooling law \cite{McNamara93a}:

\begin{equation}
  \mathcal{T}(t) = \frac{\mathcal{T}_0}{(1+\epsilon\rho \mathcal{T}_0 t)^2}
\end{equation}
where $t$ is the time, $\mathcal{T}_0$ is the temperature at time $t=0$, and
$\rho$ is the density. At late times, $t\gg 1$, the cooling law
reduces to

\begin{equation}
  \mathcal{T} = a t^{-2}
\end{equation}
where $a$ is a parameter which depends on $\epsilon$ and $\rho$.

The cooling law is a theoretical prediction by Haff \cite{Haff83} and
McNamara \etal \cite{McNamara93a}. McNamara \etal \cite{McNamara93a,
McNamara94a} have numerically investigated rigid particles in
one and two dimensions, and they find that the cooling law is
obeyed. It is important to stress that the cooling law discussed here
is valid for rigid particles only. In chapter \ref{chap:thermostat} we
will study systems of rigid and soft particles numerically and try to
apply a more general cooling law, namely $\mathcal{T} \propto t^{\alpha}$.


\section{Clustering and inelastic collapse}
\label{InelasCollapse}
As discussed in section \ref{sect:CoolProblem} a system of inelastic
particles will algebraically cool down. It has been observed by \eg
Goldhirsch \etal \cite{Goldhirsch93} that in some cases clusters are
formed; they refer to this phenomenon as a clustering
instability. The physical origin of clustering can be explained as
follows: a spontaneous fluctuation in the density occurs. The temperature
is uniform, and when the density is locally slightly larger in one
region, the collision rate in that region will be larger. The
larger collision rate leads to a faster dissipation \ie the
temperature decays faster in this region that the rest of the
system. Now, when the temperature is lower, the pressure will be lower.
The pressure gradient will attract particles to the region \ie the
density will increase. We see that a fluctuation large enough will
lead to an increasing inhomogenity, and the clustering has
occured. Goldhirsch \etal \cite{Goldhirsch93} have performed Molecular
Dynamics simulations (in two dimensions) of $2\times 10^5$--$4\times
10^5$ rigid particles, and they show that clustering is dependent on
the system size.

McNamara \etal \cite{McNamara93a, McNamara94a} have observed in
one and two dimensions another phenomenon might occur. In some
cases inelastic colliding particles might end up in an inelastic
collapse. The physics of the inelastic collapse is that three or more
particles are aligned with no separation in between. The consequence is
that the number of collisions goes to infinity. The collapse depends
on the number of particles. If the number of particles exceeds a
certain treshold value, $N_{\smbox{min}}$, the collapse might
occur. NcNamara \etal estimate the value in the elastic limit to be

\begin{equation}
  N_{\smbox{min}} = \frac{\log(2/\epsilon)}{2\epsilon}
\end{equation}

Du \etal \cite{Du95} discuss breakdown of hydrodynamics for
many-particle systems undergoing inelastic collisions. The breakdown
reported by Du \etal is not an inelastic collapse but close to the
clustering instability. It is not too obvious whether the breakdown is
exactly the same as the clustering or not because the system studied
by Du \etal is coupled to a thermostating device. In chapter
\ref{chap:thermostat} we will analyse the thermostating of inelastic
particles more closely, and present our own simulational results.

\section{Closing remarks}
Granular media have an interesting and odd phenomenology, and it is not
an exaggeration to say that granular media are a state of matter
distinct from the ordinary three states.

Many problems are still open. Granular media can in many cases be
regarded as a liquid, but no fluid dynamics has been developed. Of
course a number of attempts has been made, see \eg Haff \cite{Haff83}
and Jenkins \etal \cite{Jenkins83}, but no complete
theory exists yet. The mixing and transport of grains and powders are
heavily used in industrial processes, and therefore a development of a
fluid mechanics for granular media is important for the design,
optimisation, and control of industrial processes.
