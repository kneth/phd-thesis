%%%%%%%%%%%%%%%%%%%%%%%%%%%%%%%%%%%%%%%%%%%%%%%%%%%%%%%%%%%%%%%%%%%
%% Chemical Kinetics
%% (C) Kenneth Geisshirt (kneth@chem.ruc.dk)
%% Last modified: 31 January 1998
%%%%%%%%%%%%%%%%%%%%%%%%%%%%%%%%%%%%%%%%%%%%%%%%%%%%%%%%%%%%%%%%%%%

\chapter{Chemical Kinetics}
\label{chap:ChemKin}
Chemical kinetics is the study of how fast chemical reactions
proceed. In this chapter we will review the basic theory of chemical
kinetics. The only criteria for selecting the material is that it is
going to be applied to the systems studied in the present thesis.

The theories presented in this chapter are of macroscopic nature. The
simulations presented in chapter \ref{chap:ExtLotka} are of microscopic
nature. The topic of the thesis is partly to see if the macroscopic
description of chemical reaction is valid on a microscopic level.

Many textbooks deal with chemical kinetics, and we will here only
mention a few. Pilling \etal \cite{Pilling95} have recently written
an excellent book. It is more experimental oriented than usual
textbooks, and it covers many of the new techniques and theories. The
book by Cox \cite{Cox} gives an excellent but short overview of
reactions in liquid phase.


\section{Phenomenological chemical kinetics}
\label{sect:PhenoChemKin}
We begin our introduction to chemical kinetics in the macroscopic
world. The phenomenological approach tries to elucidate the mechanism
of a complex reaction by writing down a number of differential
equations which are the ``equations of motion'' of the concentrations
\ie the goal is to find, for each species $X$, an equation of the form

\begin{equation}
  \diff{[X]}{t} = f([X], [Y], \ldots) = \nu v([X], [Y], \ldots)
\end{equation}
where $Y, \ldots$ represent every other species present in the system,
$f$ is a function related to its stoichiometric coefficient $\nu$, and
the velocity $v$ of the reaction. The stoichiometric coefficient is
positive for products and negative for reactants.

A special class of reactions is the class of \textit{elementary
  reactions}. For an elementary reaction, the velocity can be written
using the law of mass action. Consider the reaction

\begin{equation}
  a A + b B + \cdots \overset{k}{\rightarrow} x X + y Y + \cdots
\end{equation}
where $k$ is a proportionality constant called the rate constant. The
velocity of the reaction is 

\begin{equation}
  v = k [A]^a[B]^b \cdots
\end{equation}
\ie a product over all reactants. The ``equation of motion'' of the
reactant $A$ is then

\begin{equation}
  \diff{[A]}{t} = -a\cdot v = -a k[A]^a [B]^b \cdots
\end{equation}
while the ``equation of motion'' of the product $X$ is

\begin{equation}
  \diff{[X]}{t} = x\cdot v = x k[A]^a [B]^b \cdots
\end{equation}


\section{Temperature dependency}
\label{sect:TempDepend}
It is well-known that the rate of a reaction depends on the
temperature. Often Arrhenius is regarded as the discoverer of the
so-called Arrhenius expression. The expression relates the rate
constant $k$ and the absolute temperature $T$ as \cite{Pilling95}

\begin{equation}
\label{eq:Arrhenius}
  k = Ae^{-E_a/RT}
\end{equation}
where $A$ is the pre-exponential factor, $E_a$ is the activation
energy, and $R$ is the gas constant. A plot of $\log k$ versus $1/T$
is referred to as the Arrhenius plot for the given reaction. By
applying simple algebra, we see

\begin{equation}
  \log k = \log A - \frac{E_a}{R} \frac{1}{T}
\end{equation}
\ie an Arrhenius plot should be a straight line with slope
$-E_a/R$ and intercept $\log A$. 

A more general relationship between the rate constant and the
temperature can be derived from collision theory, see \eg Pilling
\etal \cite{Pilling95}. We imagine that the reaction $A+B \rightarrow
P$ (in gas phase) consists of three reactions, namely

\begin{subequations}
  \begin{eqnarray}
    A + B &\rightleftarrows& C \\
    C     &\rightarrow&     P 
  \end{eqnarray}
\end{subequations}
where $C$ is a collision complex. In words the idea is the
following. The reactants $A$ and $B$, collide and form a collision
complex $C$. If the energy of impact is larger than a certain
threshold $E_a$, the complex will break up and form the product
$P$, otherwise the reactants are reformed. From collision theory, we
the obtain the following expression:

\begin{equation}
\label{eq:rateCollision}
  k = A^{\prime} \sqrt{T} e^{-\frac{E_a}{RT}}
\end{equation}
where $A^{\prime}$ is a pre-exponential factor.

We notice that the equation \eqref{eq:rateCollision} differs from
equation \eqref{eq:Arrhenius} by a factor of $\sqrt{T}$. In a small
temperature interval, $A^{\prime}\sqrt{T}$ is almost constant, and we
will recover the Arrhenius expression.


\section{Diffusion-controlled reactions}
\label{sect:DiffCtrlReacts}
Diffusion may play an important role for reactions in a condensed
phase. The basic idea is that two species, say $A$ and $B$, have to
diffuse together before reacting. Diffusion in two dimensions is very
different from three dimensions, and we will briefly review the
results in this section. The problem has been studied by many chemists \eg
Naqvi \cite{Naqvi74}. We will here use the results obtained by Clegg
\cite{Clegg86}.

Consider the following reactions in a condensed phase:

\begin{equation}
  A + B \overset{k_f}{\underset{k_r}{\rightleftarrows}} C
\end{equation}
where $k_f$ is the rate constant of the forward reaction, and $k_r$ is
the rate constant of the reverse reaction.

The solvent does not enter the reactions explicitly, and $A$ and $B$
must diffuse together in order to react. Let $D_A$ and $D_B$ denote
the diffusion coefficients of $A$ and $B$, respectively. Moreover, the
sum of the diffusion coefficients is denoted $D_{AB}$ \ie $D_{AB} =
D_A + D_B$. The reaction between $A$ and $B$ occurs when the
separation is less than $R_{\smbox{reac}}$ \ie when $\|\vec{r}_A -
\vec{r}_B\| < R_{\smbox{reac}}$.

In three dimensions, Clegg \cite{Clegg86} finds the following
expressions for the rate constants:

\begin{subequations}
  \begin{eqnarray}
    k_f &=& \frac{4\pi N_{\smbox{Av}} D_{AB} R_{\smbox{reac}}}{1000} \\
    k_r &=& \frac{3D_{AB}}{R_{\smbox{reac}}^2} 
  \end{eqnarray}
\end{subequations}

However, in two dimensions the situation is more complicated. According
to Clegg \cite{Clegg86}, the rate constants are given as:  

\begin{subequations}
  \begin{eqnarray}
    k_f &=& \frac{2\pi
      D_{AB}N_{\smbox{Av}}}{\log\left(\frac{R_1(t)}{R_{\smbox{reac}}}\right)} \\
    k_r &=& \frac{2D_{AB}}{R_{\smbox{reac}}^2
      \log\left(\frac{R_2(t)}{R_{\smbox{reac}}}\right)}
  \end{eqnarray}
\end{subequations}
where $R_1(t)$ and $R_2(t)$ are two functions of time. 

In both cases, the rate constants depend linearly on the sum of the
diffusion coefficients. This leads us to the following conclusion: if a
reaction is diffusion-controlled, the ratio $k/D$ must be constant
($k$ is the rate constant of the reaction, and $D$ is the diffusion
coefficients of the reactants). 
    

\section{Oscillating reactions}
\label{sect:OscReacts}
One of the main topics of this thesis is the simulation of oscillating
chemical reactions. In this section, we will briefly discuss
oscillating chemical reactions. Recently, Scott \cite{Scott94} has
written an excellent (and short) introduction to oscillating
reactions. Oscillations are one of the exotic behaviours we can observe
in chemical systems. Quasi periodic oscillations and chaos have been
observed, and in distributed systems (reaction-diffusion systems) target
patterns and spiral waves have been observed.

As already mentioned in section \ref{sect:PhenoChemKin}, to a set of
chemical reactions there is an associated set of differential
equations. In its most condensed form, the set of differential
equations can be written as

\begin{equation}
\label{eq:ReacODE}
  \diff{\vec{c}}{t} = \vec{f}(\vec{c}; \vec{\mu})
\end{equation}
where $\vec{c}$ is a vector consisting of the concentrations of the
species, $\vec{f}$ is the velocity field, and $\vec{\mu}$ is a set of
parameters. On the mathematical properties of differential
equations, the monograph by Hirsch \etal \cite{Hirsch74}
gives a good and not too mathematical introduction while Perko
\cite{Perko93} treats the subject more rigorously.


\subsection{Conditions for oscillations}
If the solution of equation \eqref{eq:ReacODE} is periodic in time \ie
$\vec{c}(t+T) = \vec{c}(t)$ for all times $t$, then we have an
oscillating reaction. A number of conditions must be fulfilled in
order to obtain a periodic solution. Probably the most important
condition is that the system must be open (or at least driven). If the
system is closed, the system will seek an equilibrium state \ie the
concentrations will become constant. The equilibrium condition is a
consequence of the second law of thermodynamics. It turns out that a
crucial condition is that at least one reaction must be
autocatalytic, see \eg Clarke \cite{Clarke80}. An autocatalytic
reaction is the chemical term of feedback. A simple example of an
autocatalytic reaction is 

\begin{equation}
\label{eq:SimpleReac}
  A + B \overset{k}{\rightarrow} 2A
\end{equation}


\subsection{Stability}
If we look at equation \eqref{eq:ReacODE} we can wonder which
mathematical properties the solution might have. We have already
discussed the oscillatory behaviour from a chemical point of view. In
this section, we will be more mathematical than chemical.

The most simple solution of equation \eqref{eq:ReacODE} is a
stationary solution. The stationary solution is the same as a constant
solution \ie $\vec{c}(t) = \vec{c}_{ss}$ where $\vec{c}_{ss}$ is a
constant vector. This stationary state is the root of the function
$\vec{f}$ \ie the solution of

\begin{equation}
  \vec{f}(\vec{c}_{ss}; \vec{\mu}) = \vec{0}
\end{equation}

Now, a stationary state can either be structurally stable, stable or
unstable. The stability of the stationary state can be evaluated in the 
following manner. Consider a small perturbation of the stationary
state, $\vec{c}_p$. The trajectory is the
solution of equation \eqref{eq:ReacODE} with $\vec{c}_p$ as the
initial state. The question of stability is answered as follows.

\begin{description}
  \item[Structurally stable] The stationary state is structurally stable
    if \protect$d(t) \equiv \|\vec{c}(t) - \vec{c}_{ss}\| \rightarrow 0\protect$
    as $t \rightarrow \infty$ where $\|\cdot\|$ is a measure of the
    distance in the concentration space.
  \item[Unstable] If the function $d(t)$ diverges as $t \rightarrow
    \infty$, we say that the stationary state is unstable.
  \item[Stable] If $d(t)$ is bounded but does not go to zero as
    $t \rightarrow \infty$ we say that the stationary state is
    stable. If the chemical reaction is oscillatory, its
    trajectory in the concentration space is bounded but does not
    approach the stationary state \ie an oscillating chemical reaction
    has a stable stationary state.
\end{description}

The stability can easily be computed. Let $\delta \vec{c}$ be defined
as $\delta\vec{c}(t) \equiv \vec{c}(t)-\vec{c}_{ss}$. The function
$d(t)$ is related to $\delta\vec{c}$ as $d(t) = \|\delta\vec{c}\|$. If
we Taylor expand the velocity field to first order, we obtain

\begin{equation}
\label{eq:LinearODE}
  \diff{\delta\vec{c}}{t} = \mtrx{J}(\vec{c}_{ss})\cdot \delta\vec{c}
\end{equation}
where $\mtrx{J}$ is the Jacobian matrix which elements are given as

\begin{equation}
  J_{ij} = \frac{\partial f_i}{\partial c_j}
\end{equation}

Equation \eqref{eq:LinearODE} is a linear ordinary differential
equation, and the solution is a textbook example. The solution is

\begin{equation}
  \delta\vec{c} = \sum_{j} \vec{e}_j e^{\imath \omega_j t}
\end{equation}
where $\vec{e}_j$ is the $j$th eigenvector of the Jacobian matrix and
$\omega_j$ is the associated eigenvalue. The stability of the
stationary point can now be summarised:

\begin{description}
\item[Structurally stable] If the real part of the any eigenvalue is
  negative, the stationary point is structurally stable.
\item[Stable] The stationary point is stable, if the
  eigenvalues has zero real parts.
\item[Unstable] If one eigenvalue has a positive real part, the
  stationary state is unstable.
\end{description}

\subsection{An example}
At this point an example would be appropriate. Let us consider the famous
reaction mechanism called the Brusselator \cite{Prigogine68}.
The mechanism is

\begin{subequations}
  \begin{eqnarray}
    A       &\overset{k_1}{\rightarrow}& X \\
    2X + Y  &\overset{k_2}{\rightarrow}& 3Y \\
    B + X   &\overset{k_3}{\rightarrow}& Y + C \\
    X       &\overset{k_4}{\rightarrow}& D
  \end{eqnarray}
\end{subequations}

The species of interest are $X$ and $Y$, and the concentrations of $A$
and $B$ are assumed to be constant. The phenomenological equations for
$X$ and $Y$ can be written down. If we scale them appropriately, we
have

\begin{subequations}
  \begin{eqnarray}
    \diff{x}{t} &=& f(x, y) = A - (B+1)x + x^2 y \\
    \diff{y}{t} &=& g(x, y) = Bx - x^2 y
  \end{eqnarray}
\end{subequations}
where $x$ and $y$ are the scaled concentrations of $X$ and $Y$, and $A$
and $B$ are the constant concentrations of species $A$ and $B$. The
stationary point $(x_{ss}, y_{ss})$ can found by solving the set of
equations:

\begin{subequations}
  \begin{eqnarray}
    f(x_{ss}, y_{ss}) &=& A - (B+1)x_{ss} + x_{ss}^2 y_{ss} = 0 \\
    g(x_{ss}, y_{ss}) &=& Bx_{ss} - x_{ss}^2 y_{ss} = 0
  \end{eqnarray}
\end{subequations}

The stationary point is $x_{ss} = A$ and $y_{ss} = \frac{B}{A}$. In
order to evaluate the stability of the stationary point, we first find
the Jacobian matrix. It is:

\begin{equation}
  \mtrx{J}(x_{ss}, y_{ss}) =
  \left( \begin{array}{cc}
    B-1 & A^2 \\
    -B  & -A^2
  \end{array} \right)
\end{equation}

The eigenvalues are the solutions of a quadratic equation \ie

\begin{equation}
  \omega^2 + (A^2 - B +1)\omega + A^2 = 0
\end{equation}

We see that the stationary point is structurally stable if $A^2 - B +
1>0$ and $(A^2-B+1)^2 - 4A^2 < 0$. Moreover, we will have a stable
stationary point if $A^2 - B + 1 = 0$ and $A^2 - B + 1 < 0$ (which
implies $B > 1$). In the stable case the solution close to the
stationary point will be oscillatory \ie under certain conditions
($B>1$) the Brusselator will be an oscillating reaction.
