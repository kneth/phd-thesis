%%%%%%%%%%%%%%%%%%%%%%%%%%%%%%%%%%%%%%%%%%%%%%%%%%%%%%%%%%%%%%%%
%% (C) Copyright 1996 by Kenneth Geisshirt (kneth@chem.ruc.dk)
%% Last updated: 8 January 1998
%%%%%%%%%%%%%%%%%%%%%%%%%%%%%%%%%%%%%%%%%%%%%%%%%%%%%%%%%%%%%%%%

\chapter*{Preface}
\addcontentsline{toc}{chapter}{\numberline{}Preface}
This thesis is the result of my graduate work carried out at Department
of Life Sciences and Chemistry, Roskilde University during the period
December 1994--November 1997. The thesis is submitted as a partial
fulfilment of a Danish Ph.D.-degree.

The topic of the thesis is computational studies of non-equilibrium
states using Molecular Dynamics. More precisely I have been engaged in
two projects:

\begin{itemize}
\item The influence of phase separation on an oscillating chemical
  reaction.
\item Temperature-control of granular media - in this thesis modelled as
  particles undergoing inelastic collisions
\end{itemize}

The two projects are very different in nature but the numerical
technique is the same.

The intended reader is in principle me \ie I have written a thesis that
I would have found valuable three years ago when I began my Ph.D.\
studies. A number of papers is enclosed as appendices. An outline of
the chapters in the thesis is: 

\begin{description}
\item[Chapter 1] is meant as a justification of the use of computer
  simulations in the physical sciences.
\item[Chapter 2] introduces both classical and statistical mechanics.
\item[Chapter 3] discusses phase transitions and the emphasis is on
  phase separation.
\item[Chapter 4] is about chemical kinetics. The chapter introduces the
  topics which are applied to the simulational results
  presented in chapter 7.
\item[Chapter 5] deals with the granular state of matter which is
  typically modelled as particles undergoing inelastic collisions. The
  basic phenomenology of granular media is introduced.
\item[Chapter 6] discusses the numerical techniques which have been
  applied in order to obtain the results presented in chapter 7 and
  chapter 8.
\item[Chapter 7] presents the results from simulations of a simple
  oscillating chemical reaction.
\item[Chapter 8] discusses how to control the temperature of many-particle
  systems and the phenomenology associated with particles
  undergoing inelastic collisions coupled to various thermostats.
\item[Appendix A] is a paper about the role of computer in modern
  chemistry \cite{Geisshirt96b}.
\item[Appendix B] is a contribution to the 9th annual workshop of
  simulational physics at University of Georgia \cite{Geisshirt96a}.
\item[Appendix C] is a paper on phase separation and chemical reactions
  \cite{Geisshirt97a}.
\item[Appendix D] is a paper on the thermostating of dissipative gases
  \cite{Geisshirt97b}.
\end{description}


\section*{Acknowledgements}
There are so many to thank. Let me first of all thank my two
supervisors, Eigil L.\ Pr{\ae}stgaard and S{\o}ren Toxv{\ae}rd. I also
wish to thank Paz Padilla for the good team work on the granular medium.

The students, the staff, and the faculty at the Department of Life
Sciences and Chemistry have made the work at the department very
pleasant. Especially I wish the thank the ``snitte team'' for many good
discussions and small talks and my two proof readers, Morten Christoffersen
and Thure Skovsgaard

Finally, I wish to thank my family and especially my parents. Their
faith in me have made all it possible.

\vspace{1.5cm}
\begin{flushright}
  Kenneth Geisshirt \\
  Roskilde, December 1997 \\
  \footnotesize{Corrections: January 1998}
\end{flushright}
