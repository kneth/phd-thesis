%%%%%%%%%%%%%%%%%%%%%%%%%%%%%%%%%%%%%%%%%%%%%%%%%%%%%%%%%%%%%%%%%%
%% Liquid Theory
%% (C) Copyright by Kenneth Geisshirt (kneth@chem.ruc.dk)
%% Last modified: 4 May 1997
%%%%%%%%%%%%%%%%%%%%%%%%%%%%%%%%%%%%%%%%%%%%%%%%%%%%%%%%%%%%%%%%%%

\chapter{Theory of Liquids}
\label{chap:LiquidTheory}
The liquid state is very different from the gas state and the solid
state. In this chapter we will briefly outline the theory of liquids
with emphasis on simple liquids. The liquids state is a condensed
phase, but it has some characteristic in common with the gas phase. 

Let us begin this chapter by defining (and thereby restricting us)
what a simple liquid is. 


\section{Equation of state}
\label{sect:EOS}
The equation of state is the fundamental equation describing a given
thermodynamic system. It is a functional description of the density,
$\rho$, the temperature $T$ and the pressure, $p$, \ie

\begin{equation}
  f(\rho, T, p) = 0
\end{equation}

Put on the edge, we will claim that the search of the equation of
state is the fundamental search in thermodynamic studies. The only
problem we face is that it is virtually impossible to derive the
equation of almost any system from the equations of motion. 


\section{Idealised systems}
\label{sect:Ideal}
It is possible to derive the equation of state for a number of system
but these systems are idealised in the sense that they are only rough
models of the real Nature.

The ideal monatomic gas is a well know example, see \eg Huang
\cite[section 6.5]{Huang87}. The atoms do not interact, and the
Hamiltonian is 

\begin{equation}
  {\calif H}(\vec{r}_i, \vec{p}_i) = \sum_{i=1}^N
  \frac{\vec{p}_i^2}{2m}
\end{equation}

It is fairly easy to derive the equation of state; it turns out to be
$p = k_B \rho T$ where $k_B$ is Boltzmann's constant. The equation of
state is nothing else than the ideal gas law.

It is also possible to do analytical studies of crystals assuming that
the interaction between the atoms in the crystal follow a harmonic
potential. This is called an Einstein crystal.

The two example above do \textbf{not} belong to the liquid state of
matter. Because of the lack of solvable models of liquids (even simple
liquids), we have only computer simulations to obtain information from.




