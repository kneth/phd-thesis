%%%%%%%%%%%%%%%%%%%%%%%%%%%%%%%%%%%%%%%%%%%%%%%%%%%%%%%%%%%%
%% Metaphysics of simulations
%% (C) Kenneth Geisshirt (kneth@chem.ruc.dk)
%% Last modified: 8 January 1998
%%%%%%%%%%%%%%%%%%%%%%%%%%%%%%%%%%%%%%%%%%%%%%%%%%%%%%%%%%%%

\chapter{Computational Science}
\label{chap:CompSci}
\pagenumbering{arabic}
The present thesis is about computer simulations of
statistical-mechanical systems. In this chapter we will take a closer
look at the branch of science called \textit{computational
science}. The chapter is not restricted to chemistry, and we have
tried to write it in general terms but most of the examples will come
from chemistry since the author is most familiar with chemistry. 


\section{The new branch}
\label{sect:NewBranch}
Science - here understood as the physical sciences including physics
and chemistry - has traditionally been divided into two branches or
legs: theory and experiment. The experiment is nowadays seen as the
notion that makes \eg chemistry, scientific. The experiment was
introduced in science during the scientific revolution 3--4 centuries
ago \cite{Westfall71}. Before then, science was mainly theoretical.

The introduction of the computer into science has started a new
revolution: a new branch of science is emerging, namely computational
science. No precise definition of computational science exists,
and the question can easily trigger a passionate debate among
computational scientists - recently Hocquet asked\footnote{The debate
  was running on the Internet through the ``Computational Chemistry
  List'' (a mailing list), and an archive of the contributions can be
  found at
\texttt{http://ccl.osc.edu/ccl/archived{\_}messages.html }.}
whether computational chemistry was restricted to molecular modelling,
and his question started a long and interesting debate. According to
our personal taste, we will adopt the definition by Golub \etal
\cite{Golub92}: 

\begin{quotation}
\noindent Computational science is the set of tools, techniques, and
  theories used to solve problems in science and engineering by a
  computer. 
\end{quotation}

One of the first clear examples of computational investigation of
physical properties is the study by Metropolis \etal
\cite{Metropolis53}. In 1953 Metropolis \etal published a paper which
introduced the Monte Carlo (MC) method. Shortly thereafter Alder \etal
\cite{Alder57} introduced the Molecular Dynamics (MD) method. Adler
\etal and Metropolis \etal were interested in the equation of state and
phase transitions of simple liquids.

\section{Theoretical or experimental?}
\label{sect:TheoryOrExp}
Science has traditionally been divided into two branches; theoretical
and experimental. One can put forward the question, whether
computational science is experimental or theoretical.

We can easily answer the question: Computational science is neither
experimental nor theoretical, but it does have notion in common with both.
Computational science requires software 
which is produced by programming. Programming is the process where
ideas are formalised and written as a computer
program. The notation used in programming is not formul{\oe} as in
the theoretical approach but a programming language. Still, the
notation is unambiguous as the mathematical notation, and this aspect
of computational science is close to theoretical science. However, the
computer can be regarded as an instrument, and then computational science
suddenly has an experimental orientation \cite{OGM93}. Rapaport
\cite{Rapaport95} has pro-arguments for the term ``numerical
experiment''. Moreover, Rapaport argues, the distinction between
computational methods and theoretical approaches is the cost; theory
can be done with a piece of paper and a pencil, while computations
require an investment in hardware.

Traditionally, science has tried to construct
theories that explain experiments to some extend or to conduct
experiments that can verify a given theory. Figure
\ref{fig:ExpModelTheory} shows the role of computational
science. Theories are used to obtain approximations and general
explanations of the experimental data. The computer simulations are
instead used to investigate the model of the experiment in a more
naive fashion than it is analytically possible.


\begin{figure}
  \begin{center}
    \setlength{\unitlength}{0.00041667in}
%
\begingroup\makeatletter\ifx\SetFigFont\undefined
% extract first six characters in \fmtname
\def\x#1#2#3#4#5#6#7\relax{\def\x{#1#2#3#4#5#6}}%
\expandafter\x\fmtname xxxxxx\relax \def\y{splain}%
\ifx\x\y   % LaTeX or SliTeX?
\gdef\SetFigFont#1#2#3{%
  \ifnum #1<17\tiny\else \ifnum #1<20\small\else
  \ifnum #1<24\normalsize\else \ifnum #1<29\large\else
  \ifnum #1<34\Large\else \ifnum #1<41\LARGE\else
     \huge\fi\fi\fi\fi\fi\fi
  \csname #3\endcsname}%
\else
\gdef\SetFigFont#1#2#3{\begingroup
  \count@#1\relax \ifnum 25<\count@\count@25\fi
  \def\x{\endgroup\@setsize\SetFigFont{#2pt}}%
  \expandafter\x
    \csname \romannumeral\the\count@ pt\expandafter\endcsname
    \csname @\romannumeral\the\count@ pt\endcsname
  \csname #3\endcsname}%
\fi
\fi\endgroup
{\renewcommand{\dashlinestretch}{30}
\begin{picture}(9024,7839)(0,-10)
\put(117,6717){\arc{210}{1.5708}{3.1416}}
\put(117,7707){\arc{210}{3.1416}{4.7124}}
\put(1707,7707){\arc{210}{4.7124}{6.2832}}
\put(1707,6717){\arc{210}{0}{1.5708}}
\path(12,6717)(12,7707)
\path(117,7812)(1707,7812)
\path(1812,7707)(1812,6717)
\path(1707,6612)(117,6612)
\put(117,3117){\arc{210}{1.5708}{3.1416}}
\put(117,4107){\arc{210}{3.1416}{4.7124}}
\put(1707,4107){\arc{210}{4.7124}{6.2832}}
\put(1707,3117){\arc{210}{0}{1.5708}}
\path(12,3117)(12,4107)
\path(117,4212)(1707,4212)
\path(1812,4107)(1812,3117)
\path(1707,3012)(117,3012)
\path(912,6612)(912,4212)
\blacken\path(882.000,4332.000)(912.000,4212.000)(942.000,4332.000)(882.000,4332.000)
\put(5517,6717){\arc{210}{1.5708}{3.1416}}
\put(5517,7707){\arc{210}{3.1416}{4.7124}}
\put(7107,7707){\arc{210}{4.7124}{6.2832}}
\put(7107,6717){\arc{210}{0}{1.5708}}
\path(5412,6717)(5412,7707)
\path(5517,7812)(7107,7812)
\path(7212,7707)(7212,6717)
\path(7107,6612)(5517,6612)
\put(3717,3117){\arc{210}{1.5708}{3.1416}}
\put(3717,4107){\arc{210}{3.1416}{4.7124}}
\put(5307,4107){\arc{210}{4.7124}{6.2832}}
\put(5307,3117){\arc{210}{0}{1.5708}}
\path(3612,3117)(3612,4107)
\path(3717,4212)(5307,4212)
\path(5412,4107)(5412,3117)
\path(5307,3012)(3717,3012)
\put(7317,3117){\arc{210}{1.5708}{3.1416}}
\put(7317,4107){\arc{210}{3.1416}{4.7124}}
\put(8907,4107){\arc{210}{4.7124}{6.2832}}
\put(8907,3117){\arc{210}{0}{1.5708}}
\path(7212,3117)(7212,4107)
\path(7317,4212)(8907,4212)
\path(9012,4107)(9012,3117)
\path(8907,3012)(7317,3012)
\put(1917,117){\arc{210}{1.5708}{3.1416}}
\put(1917,1107){\arc{210}{3.1416}{4.7124}}
\put(3507,1107){\arc{210}{4.7124}{6.2832}}
\put(3507,117){\arc{210}{0}{1.5708}}
\path(1812,117)(1812,1107)
\path(1917,1212)(3507,1212)
\path(3612,1107)(3612,117)
\path(3507,12)(1917,12)
\put(5517,117){\arc{210}{1.5708}{3.1416}}
\put(5517,1107){\arc{210}{3.1416}{4.7124}}
\put(7107,1107){\arc{210}{4.7124}{6.2832}}
\put(7107,117){\arc{210}{0}{1.5708}}
\path(5412,117)(5412,1107)
\path(5517,1212)(7107,1212)
\path(7212,1107)(7212,117)
\path(7107,12)(5517,12)
\path(4512,3012)(2712,2412)
\path(912,3012)(2712,2412)(2712,1212)
\blacken\path(2682.000,1332.000)(2712.000,1212.000)(2742.000,1332.000)(2682.000,1332.000)
\path(4512,3012)(6312,2412)
\path(8112,3012)(6312,2412)(6312,1212)
\blacken\path(6282.000,1332.000)(6312.000,1212.000)(6342.000,1332.000)(6282.000,1332.000)
\path(1812,7212)(5412,7212)
\blacken\path(5292.000,7182.000)(5412.000,7212.000)(5292.000,7242.000)(5292.000,7182.000)
\path(6312,6612)(4512,4212)
\blacken\path(4560.000,4326.000)(4512.000,4212.000)(4608.000,4290.000)(4560.000,4326.000)
\path(6312,6612)(8112,4212)
\blacken\path(8016.000,4290.000)(8112.000,4212.000)(8064.000,4326.000)(8016.000,4290.000)
\put(2787,2037){\makebox(0,0)[lb]{\smash{{{\SetFigFont{6}{7.2}{rm}Compare}}}}}
\put(6387,2037){\makebox(0,0)[lb]{\smash{{{\SetFigFont{6}{7.2}{rm}Compare}}}}}
\put(987,4962){\makebox(0,0)[lb]{\smash{{{\SetFigFont{6}{7.2}{rm}experiments}}}}}
\put(987,5187){\makebox(0,0)[lb]{\smash{{{\SetFigFont{6}{7.2}{rm}Perform}}}}}
\put(5412,5187){\makebox(0,0)[lb]{\smash{{{\SetFigFont{6}{7.2}{rm}Perform}}}}}
\put(5412,4887){\makebox(0,0)[lb]{\smash{{{\SetFigFont{6}{7.2}{rm}simulations}}}}}
\put(7437,5187){\makebox(0,0)[lb]{\smash{{{\SetFigFont{6}{7.2}{rm}Deduce theories}}}}}
\put(2862,7287){\makebox(0,0)[lb]{\smash{{{\SetFigFont{6}{7.2}{rm}Make a model}}}}}
\put(537,3762){\makebox(0,0)[lb]{\smash{{{\SetFigFont{6}{7.2}{rm}Experi-}}}}}
\put(537,3462){\makebox(0,0)[lb]{\smash{{{\SetFigFont{6}{7.2}{rm}mental}}}}}
\put(537,3162){\makebox(0,0)[lb]{\smash{{{\SetFigFont{6}{7.2}{rm}results}}}}}
\put(2337,687){\makebox(0,0)[lb]{\smash{{{\SetFigFont{6}{7.2}{rm}Test of}}}}}
\put(2337,387){\makebox(0,0)[lb]{\smash{{{\SetFigFont{6}{7.2}{rm}the model}}}}}
\put(5712,612){\makebox(0,0)[lb]{\smash{{{\SetFigFont{6}{7.2}{rm}Test of the theory}}}}}
\put(4287,3837){\makebox(0,0)[lb]{\smash{{{\SetFigFont{6}{7.2}{rm}Exact}}}}}
\put(4287,3462){\makebox(0,0)[lb]{\smash{{{\SetFigFont{6}{7.2}{rm}results}}}}}
\put(4287,3162){\makebox(0,0)[lb]{\smash{{{\SetFigFont{6}{7.2}{rm}for the model}}}}}
\put(7662,3687){\makebox(0,0)[lb]{\smash{{{\SetFigFont{6}{7.2}{rm}Theoretical}}}}}
\put(7662,3387){\makebox(0,0)[lb]{\smash{{{\SetFigFont{6}{7.2}{rm}predictions}}}}}
\put(6012,7137){\makebox(0,0)[lb]{\smash{{{\SetFigFont{6}{7.2}{rm}Model}}}}}
\put(312,7287){\makebox(0,0)[lb]{\smash{{{\SetFigFont{6}{7.2}{rm}Physical/chemical}}}}}
\put(312,6987){\makebox(0,0)[lb]{\smash{{{\SetFigFont{6}{7.2}{rm}system}}}}}
\end{picture}
}

  \end{center}
  \caption[The role of computational science]{The relationship between
  experiment, computer simulation and theory in modern physical
  science. From \protect\cite{Geisshirt96b} and inspired by
  \protect\cite{Allen87}.\label{fig:ExpModelTheory}}
\end{figure}

Let us - through an example - explain the ideas formulated
above. Consider a simple liquid; it could be methane,
$\mathrm{CH}_4$. We can, by experimental means, measure a given physical
quantity; for instance the pressure at a given temperature and
density. Methane is a simple liquid, and we will expect that it
behaves as a classical-mechanical system \ie we will expect that the
Hamiltonian equations of motion are applicable, and the interaction
potential is the Lennard-Jones potential, see \eg Hansen \etal
\cite{Hansen86}. It is possible from the interaction potential to
evaluate the next few virial coefficients, and use these to obtain
approximative values of the pressure as function of temperature and
density. This is the theoretical or analytical approach. The
brute-force method (or the naive method) is to numerically solve
the equations of motion and calculate the pressure at a given density
and temperature. This approach is exact in the sense that the only two
approximations used are the application of classical mechanics and the
numerical scheme/the computer program. All three approaches
provide us with a set of data, which can now be compared and the model
and the approximation in the theoretical derivation can be verified or
falsified by comparing with the experimental data.


\section{Validation}
\label{sect:ModelValid}
In science in general, the concept of validation covers the process of
comparing the real-world facts and the predictions of the model, and
then conclude whether the model is reliable or not. The use of
computers to make the predictions, makes the validation process a bit
more complicated.

The complication of the validation process comes from the nature of
the solution strategy: the predictions by the model are obtained by
the use of a computer program. Typically, the computer program is
written by the scientist himself. The first step in making reliable
predictions is to justify that the computer program works correctly. 

The verification of computer programs is not a trivial matter, and
scientific programs might be even harder to verify. A typical
verification of a computer program is to test the program. The test
procedure is based on the idea that the program maps an input space
on an output space. For example, a user working with a database
at the library, the queries he types (the name of the author,
\textit{etc}.) is the input space, and the records that the database program
prints are the output. Most computer programs are deterministic \ie the
mapping between the input and the output is a deterministic mapping (the
mapping is mathematically speaking surjective and not invertible). The
testing procedure used is the following: the programmer constructs by
hand the output associated with a given input (this is called a test
example). If the program returns the same output as the programmer
deduced, the program is less likely to contain errors. The art of
testing is then to construct test examples which are as orthogonal as
possible, so the testing period will be as short as possible. This test
procedure is discussed at length by Myers \cite{Myers84}.

Now the construction of test examples is not as simple as outlined
above. The reason why we write computer programs for scientific
purposes is that we cannot solve the problem by hand \ie map the
input space onto the output space. Three procedures
can then be persuited: 

\begin{itemize}
  \item The reproduction of data obtained from other sources. In
    some fields, a set of test examples might exist.
  \item Often in some limiting cases, the solution can be found
    analytically. This can be used to construct test
    examples. For example, we can easily solve 2 linear equations with
    2 unknown variables, and this can be used to test a general linear
    equation solver.
  \item One can monitor some quantity which value is known from some
    exact analysis of the problem.
\end{itemize}

The last point is useful when it comes to
statistical-mechanical simulations. As an example consider a system of
$N$ particles in a microcanonical ensemble. We know that the total
energy $E_{\smbox{tot}}$, is a conserved quantity \ie
$E_{\smbox{tot}}$ must be constant. One can 
print $E_{\smbox{tot}}$ as function of time, and if it is constant
(within the precision of the computer) we have increased the degree of
reliability of the program.


\section{Simulational or computational scientists}
So far we have discussed computational science, but one sees a
division of the scientific community in the way, its members use
computing equipment. Scientists doing simulations or calculations can be
divided into two types. This classification is discussed by Mouritsen
\cite{OGM93}. The types are: 

\begin{itemize}
  \item The type that happily spend hours (or even weeks or months) on
    writing computer programs. Designing and implementing the computer
    program is more satisfactory for them than the actual scientific
    problem. We will call them \textit{computational scientists}.
  \item The second type is the scientist who is a user of a simulation
    program. He has never written a simulation program but the
    computing system (hardware and software) is his major scientific
    equipment. This type is called the \textit{simulational scientist}.
\end{itemize}

The classification above is of course extreme, and most scientists who
use computers, are somewhere in between. The two types have advantages
and disadvantages. The computational scientist knows everything about
the computational method - its strong and weak points. But he works
slowly, in the sense that he does not solve many problems but
develops new computational methods. The simulational scientist, on
the other hand, solves many scientific problems, but has to rely on
computer programs written by others. He tends to regard the computer
(hardware and software) as a black box. Neither of two types are the
best, but it seems to us that the population of simulational
scientists is growing faster than the population of computational
scientists. The desktop computers have become fast enough to solve
large problems, and many problems can be solved by using commercial
software packages \eg quantum chemical problems.

Even though commercial software packages can solve many of the
(standard) problems in science, we strongly believe that the best
scientific programs are written by computational scientists. A person
trained in science will, in general, be better to solve computational
problems related to science than the average programmer.

\section{Concluding remarks}
\label{sect:MetaConclusion}
The application of computers to scientific problems creates new
problems. But it also introduces new solution methods. Problems which
were regarded as impossible to solve, can now be examined numerically.
We regard this development as very exciting.

Moreover, computer simulations may also save us from many problems, and
be able to create short-cuts. In the pharmaceutical industry, computer
simulation techniques can be used to ``test'' compounds; it is often
referred to as \textit{rational drug design}. The excellent book by
Grant \etal \cite{Grant95} shows many of the modern application of
computer in chemistry, and we strongly believe that computers will help
both life sciences and physical sciences in solving complex problems
in the future.
