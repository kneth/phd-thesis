%%%%%%%%%%%%%%%%%%%%%%%%%%%%%%%%%%%%%%%%%%%%%%%%%%%%%
%% Conclusion
%% (C) Kenneth Geisshirt (kneth@chem.ruc.dk)
%% Last modified: 1 December 1997
%%%%%%%%%%%%%%%%%%%%%%%%%%%%%%%%%%%%%%%%%%%%%%%%%%%%%

\chapter{Discussion and Conclusion}
\label{chap:Conclusion}

We have in this thesis presented simulational results of two very
different systems: oscillating chemical reactions and dissipative
gases/ganular media.
Both systems have two things in common: they are in a non-equilibrium
state, and they have been investigated by Molecular Dynamics.

The systems studied in this thesis are ``simple''. They are simple in
the underlying physical model, but the phenomenology is indeed not
simple. Even though the models are simple, we strongly believe that
these models are able to capture some of the phenomena seen in the Nature.

Molecular Dynamics is an invaluable tool when many-particle systems are
studied. Two special-purpose MD programs have been designed and
implemented by the author; one for Lennard-Jones particles and one for
rigid particles. The advantage of MD is that a MD program can be used
in equilibrium and non-equilibrium situations. We see Molecular
Dynamics as the link between the advanced theories and the complicated
experiments. General-purpose Molecular Dynamics programs exist but they
have an orientation toward biological problems (proteins in aqueous
solution). MD programs for simple liquids (short-ranged 
potentials \eg the Lennard-Jones potential) and rigid particles are
typically written by the chemist/physicist who is going to use them.
But no chemist can keep up with
the pace of development of computers, especially not the new parallel
computers. An important project would be to implement - using modern
software engineering methods - novel algorithms on state-of-the-art
computers. Moreover, it is important to design and implement a number of
tools which can analyse data, and an easy-to-use graphical user interface
(GUI) is essential.

The rich phenomenology of dissipative gases and granular media has
surprised us. In order to be able to model granular media, we will need
macroscopic descriptions analoguous to fluid mechanics and
thermodynamics. As far as we see it, Molecular Dynamics is at the
moment the only reliable tool which can model granular media. The
validation of a fluid mechanics can therefore only be done with the
help from MD.

Using Molecular Dynamics to study chemical reactions is not a new idea;
in fact it is more than two decades ago since the first simulations were
performed. We have clearly demonstrated in this thesis that the
macroscopic phenomena can be reproduced in a finite system using MD. In
other words, MD is able to reproduce the macroscopic world, and even
better than that: the idea behind MD is so naive that it is much
easier to understand the physical model.


The interest in the interplay of (energetically driven) chemical
reactions and phase separation has increased in recent time (measured
by the number of papers per year). Almost all studies have so far been
carried out on a mesoscopic level (Hilliard-Cahn theory). Molecular
Dynamics is in this context an invaluable tool because the number of
simplifying assumptions is limited. Simulations of phase separation and
chemical reactions are computer intensive. In order to see pattern
formation we need a large number of particles (often $5\cdot 10^5$)
and follow the system over long time scales. The simulations
presented in this thesis are not exhausting in the sense that they are
only the beginning. We have demonstrated that a phase separation is able to
change the underlying mechanism of chemical reactions. The theoretical
predictions have not yet been verified by simulations, and moreover,
the theory has to be extended to three- and four-component system in
order to be more applicable.
