%% (C) Copyright 1996 by Kenneth Geisshirt
%% Last modified: 15 Mar 1996
%% Modified for Ph.D. thesis: 14 August 1997
\chapter{Oscillating Chemical Reactions Simulated by Molecular Dynamics}
\label{Paper:csp96}

This paper appeared as a contribution to the workshop ``Computer
Simulation Studies in Condensed-Matter Physics 9'' held in Athes,
Georgia, March 1996. It has been published by Springer-Verlag in
``Proceedings in Physics'' volume 82, pp.\ 144-149 in 1997. The
authors were Kenneth Geisshirt, Eigil Pr{\ae}gaard, and S{\o}ren
Toxv{\ae}rd.



\begin{center}
Abstract

We outline how chemical reactions can be simulated by Molecular Dynamics,
and we apply our technique to an oscillating chemical reaction scheme;
a modified Volterra-Lotka scheme. We observe deviations from the
phenomelogical equations in the phase separating temperature region.
\end{center}

\section{Introduction}
The phenomenon of oscillating chemical reactions is fairly new \ie it
has been known for the last three or four decades. The development
of the theory for oscillating chemical reactions is strongly connected to 
the theory of nonlinear dynamical systems and nonequilibrium 
thermodynamics. Until now oscillating chemical
reactions have experimentally been investigated macroscopicly and
the theory used for describing the reactions is
phenomelogical. Experimentally stu\-died oscillating reactions are
either external driven systems or systems allowed to relax to
equilibrium. 

On the other hand oscillating systems have been investigated
numrically as long as oscillating reactions have been known by means
of coupled differential equations in the consentrations. It is however
possible by today's computers to simulate chemical reactions in
continious space and on a molecular level by Monte-Carlo (MC) and
Molecular Dynamics (MD) simulations. It implies that it is possible to
investigate realistic systems where also other nonlinear dynamics,
like a phase seperation might influence the chemical reaction. It was
recently shown that chemical reactions have a large effect on phase  
separation; see \eg \cite{glotzer} for a Monte-Carlo study of simple 
first-order reactions $A \rightleftarrows B$ and Molecular Dynamics study of
first- and second-order reactions \cite{tox2}. 


\section{Chemical Kinetics and Molecular Dynamics}
Chemical reactions, on a microscopic level, are usually formulated as
coupled  elementary reactions. We will here only consider coupled
bimolecular reactions \ie a reactions where there are two reactants
in each reactions. 

The usual chemical viewpoint is the transition state theory. The basic idea
is that the two reactants collide and form an activated complex ($C^{\ddag}$),
which can either break into the reactants again or into the products.
The quantum mechanical details in this reversible dynamics are,
however, believed to be of no importance for the overall kinetics of
oscillating reactions. The main feature is that the exchanges of
covalent bonds occur with a certain probability when the reactants
are separated at a short molecular distance.  
The reaction rate is typically given by an Arrhenius expression where
the rate depends exponentially on the activation energy in units of
the kinetic energy (the temperature). Experimentally, reactions
with small activation energy are difficult to follow because they are
fast. In Molecular Dynamics, however, we are in the opposite
situation. With today's computers we are able to simulate of the order
$10^{5}$ (simple) molecules in of the order ${10^{6}}$ 
time steps, which correspond to reaction times of only nanoseconds. 
For this reason the transition energy is taken to be of the order
$T$ which automatically implies that the reaction takes place on a
time scale equal to the mean collision time. Futhermore, a chemical
reaction appears in an open system, \ie without 
a conserved energy and fixed number of reactant. On the other hand a
traditional MD system consists of a fixed number of particles 
and with a well-defined Hamiltonian, which makes it to a non-trivial
task to reformulate the dynamics in open systems with chemical
reactions. The actual implementation and its impact on 
the dynamics is described in in more details in Ref.\ \cite{tox2}.
In summary the reaction dynamics is implemented into the MD 
by spontaneously  and with a certain probability, to relabel the
colliding particles. Futhermore, in order not to introduce large force
gradients into the system by this reaction dynamics we only consider 
reactions between particles with almost the same excluded volumes,
\ie the chemical individualities of the reactants are given by their
long range forces and not by their short range forces.  
This feature is certainly unrealistic from a chemical view point; but
it is believed to be of no importance for the qualitative behavior of
the chemical reaction dynamics as will be demonstrated.


\section{A particular example: MD simulations of the modified
Volterra-Lotka reaction}
We have chosen a particular oscillating system as our system. It consists of
three species and three reactions. The reactions are:

\begin{eqnarray}
  A + X &\overset{k_1}{\rightarrow}& 2X \label{Reac1} \\
  X + Y &\overset{k_2}{\rightarrow}& 2Y \label{Reac2} \\
  Y + A &\overset{k_3}{\rightarrow}& 2A \label{Reac3}
\end{eqnarray}

To minimise the number of parameters in our model, we have chosen to set $k_3$
to the same value as $k_1$.

This reaction scheme is a simple modification of the Volterra-Lotka
equation for a homogeneous (in space) oscillating reaction
\cite{lotka1} \cite{lotka2}. In 
a traditional chemical reaction experiment the system is fed by a
reactant $A$ which through a consecutive set of unimolecular reactions
is transformed into the product $B$. In our case and for
an open MD system of particles in a volume with pe\-riodi\-cal boundaries
it is, however, convenient to equal reactant and product and thus to
recover $A$ through reaction \ref{Reac3}. It is the
only modification of the traditional reaction scheme for the
oscillating reaction and the flow through the system is obtained, as
usual by only forward elementary reactions without reverse
reactions. This corresponds to that the reactant has a significant
higher (free) energy than the product.

From these three reactions, one can easily derive three phenomelogical 
equations describing the concentrations of the three species. The equations
are:

\begin{eqnarray*}
  \frac{d[A]}{dt} &=& -k_1[A][X] + k_1[A][Y] \\
  \frac{d[X]}{dt} &=& k_1[A][X] - k_2[X][Y] \\
  \frac{d[Y]}{dt} &=& k_2[X][Y] - k_1[A][Y]
\end{eqnarray*}
where $[\cdot]$ means the concentration. The phenomelogical equations are
only valid in the complete homogeneous case, which experimentally is 
realised by stirring the contents of the chemical reactor.

We have three coupled ordinary differential equations which are
impossible to find an analytical solution to. In general, we have
equations of the type:

\[
  \frac{d\vec{x}}{dt} = \vec{f}(\vec{x})
\]

We can first find a stationary solution to the problem above \ie
find an $\vec{x}_0$ which satisfies $\vec{f}(\vec{x}_0) =
\vec{0}$. Then we can linearise the differential equation at this
stationary point and obtain

\[
  \frac{d\delta\vec{x}}{dt} = \mathbf{J}(\vec{x}_0) \delta\vec{x}
\]
where $\mathbf{J}$ is the Jacobian matrix and $\delta\vec{x}$ is the
deviation from the stationary point, i.e.\ $\delta\vec{x} =
\vec{x}-\vec{x}_0$. The linearised system is a linear differential
equation. The solution of the linear differential equation is 

\[
  \delta\vec{x} = \sum_{i} \vec{e}_i e^{\omega_i t}
\]
where $\omega_i$ is the $i$th eigenvalue and $\vec{e}_i$ is basically
the $i$th eigenvector which is normalised appropriately.

The modified Volterra-Lotka scheme has a stationary solution, which is

\begin{eqnarray*}
  {[A]}_0 &=& C \left( 1-\frac{2k_1}{2k_1+k_2} \right) \\
  {[X]}_0 &=& {[Y]}_0 = \frac{C k_1}{2k_1 + k_2}
\end{eqnarray*}
where $C$ is the total concentration \ie $C = [A]+[X]+[Y]$ which is a 
constant.

We can now linearise the system and calculate the eigenvalues of the Jacobian 
matrix at the stationary point. We get

\[
  \lambda_1 = \overline{\lambda_2} 
  = \frac{C k_1 \sqrt{k_2}}{\sqrt{2k_1 + k_2}}\imath
\]
which shows us that the system will have sustained oscillations - at least
close to the steady-state solution.


\section{Simulational details}
The chemical equations for the Volterra-Lotka reaction expresses a
\textit{homogeneous} change in the concentration space of the
reactants, which for certain values of the reaction rates,
oscillate. In a real experiment it means that large concentrations of
a certain component is built up for a while and then disappears
through a chemical reaction. If the three components are miscible for
all concentrations one would expect that the dynamics might be well
described by the traditional homogeneous differential equations,
whereas a new situation appears if one of the concentrations exceeds
the critical value of solubility and a competing phase separation
takes place in the oscillating system.  For this
reason we will expect a non-trivial difference between a set of
homogeneous chemical reactions and a corresponding MD simulation.
This is due to the fact that we in an MD system can create a phase
separation by a chemical reaction, simply by ensuring that the product
has an intermolecular potential which favors phase separation. This is
ensured in the  very same manner as for the chemical reaction by
varying the long range attractions between the molecules. For details
see \eg \cite{tox3}.

The system is - as already mentioned - a three-component system. The 
interaction of particles of the same type is a Lennard-Jones potential
which is cut at $r = 2.5$. The interaction between particles of different
types is also a Lennard-Jones potential, but it is cut at $r=1.0$ 
meaning that it is replusive only. The density of the system is
$0.6$. At low enough temperature a system with such interactions will
phase separate. We use the same reaction parameters 
throughout all our simulations. The collision diameter is set to 1.0 
(in Lennard-Jones units). 
The probability of reaction is set to 
$1.0\cdot 10^{-3}$ for reactions \ref{Reac1} and \ref{Reac3} while 
for reaction \ref{Reac2} it is set to $1.1\cdot 10^{-3}$.

All our simulations are performed in the $NVT$-ensemble using a 
Nos\'{e}-Hoover thermostat to control the temperature and a leap-frog
algorithm to integrate the equations of motion \cite{toxvaerd}. The
chosen time step is $h = 5\cdot 10^{-4}$. Furthermore, all simulations are
done in two spatial dimensions.

\section{Results and open questions}                           
We have performed a number of simulations at different temperatures with 
either 1024 or 8192 particles. Rate
constant for a reaction usually follow Arrhenius' law \ie
$k(T) = A\cdot e^{-E_a/RT}$, 
where $E_a$ is the activation energy, $A$ is a preexponetial factor, and $R$
is the gas constant.

During our simulations we have stored instantaneous values of the 
concentrations and the number of times the three reactions has
occured. Figure \ref{FigConc} shows the concentration of $A$ versus
time for a typical simulation.
The number of
times a reaction has occured (denoted $N_R$) is proprotional to the
velocity of that reaction 
and we therefore have a simple relationship for the rate constant, namely
$k = \frac{N_R}{[J][J^{\prime}]}$,
where $J$ and $J^{\prime}$ denote the two reactants in the reaction.

\begin{figure}
  % GNUPLOT: LaTeX picture
\setlength{\unitlength}{0.240900pt}
\ifx\plotpoint\undefined\newsavebox{\plotpoint}\fi
\sbox{\plotpoint}{\rule[-0.200pt]{0.400pt}{0.400pt}}%
\begin{picture}(1500,675)(0,0)
\font\gnuplot=cmr10 at 10pt
\gnuplot
\sbox{\plotpoint}{\rule[-0.200pt]{0.400pt}{0.400pt}}%
\put(220.0,113.0){\rule[-0.200pt]{292.934pt}{0.400pt}}
\put(220.0,113.0){\rule[-0.200pt]{0.400pt}{129.845pt}}
\put(220.0,113.0){\rule[-0.200pt]{4.818pt}{0.400pt}}
\put(198,113){\makebox(0,0)[r]{0}}
\put(1416.0,113.0){\rule[-0.200pt]{4.818pt}{0.400pt}}
\put(220.0,221.0){\rule[-0.200pt]{4.818pt}{0.400pt}}
\put(198,221){\makebox(0,0)[r]{0.2}}
\put(1416.0,221.0){\rule[-0.200pt]{4.818pt}{0.400pt}}
\put(220.0,329.0){\rule[-0.200pt]{4.818pt}{0.400pt}}
\put(198,329){\makebox(0,0)[r]{0.4}}
\put(1416.0,329.0){\rule[-0.200pt]{4.818pt}{0.400pt}}
\put(220.0,436.0){\rule[-0.200pt]{4.818pt}{0.400pt}}
\put(198,436){\makebox(0,0)[r]{0.6}}
\put(1416.0,436.0){\rule[-0.200pt]{4.818pt}{0.400pt}}
\put(220.0,544.0){\rule[-0.200pt]{4.818pt}{0.400pt}}
\put(198,544){\makebox(0,0)[r]{0.8}}
\put(1416.0,544.0){\rule[-0.200pt]{4.818pt}{0.400pt}}
\put(220.0,652.0){\rule[-0.200pt]{4.818pt}{0.400pt}}
\put(198,652){\makebox(0,0)[r]{1}}
\put(1416.0,652.0){\rule[-0.200pt]{4.818pt}{0.400pt}}
\put(220.0,113.0){\rule[-0.200pt]{0.400pt}{4.818pt}}
\put(220,68){\makebox(0,0){0}}
\put(220.0,632.0){\rule[-0.200pt]{0.400pt}{4.818pt}}
\put(394.0,113.0){\rule[-0.200pt]{0.400pt}{4.818pt}}
\put(394,68){\makebox(0,0){200000}}
\put(394.0,632.0){\rule[-0.200pt]{0.400pt}{4.818pt}}
\put(567.0,113.0){\rule[-0.200pt]{0.400pt}{4.818pt}}
\put(567,68){\makebox(0,0){400000}}
\put(567.0,632.0){\rule[-0.200pt]{0.400pt}{4.818pt}}
\put(741.0,113.0){\rule[-0.200pt]{0.400pt}{4.818pt}}
\put(741,68){\makebox(0,0){600000}}
\put(741.0,632.0){\rule[-0.200pt]{0.400pt}{4.818pt}}
\put(915.0,113.0){\rule[-0.200pt]{0.400pt}{4.818pt}}
\put(915,68){\makebox(0,0){800000}}
\put(915.0,632.0){\rule[-0.200pt]{0.400pt}{4.818pt}}
\put(1089.0,113.0){\rule[-0.200pt]{0.400pt}{4.818pt}}
\put(1089,68){\makebox(0,0){1e+06}}
\put(1089.0,632.0){\rule[-0.200pt]{0.400pt}{4.818pt}}
\put(1262.0,113.0){\rule[-0.200pt]{0.400pt}{4.818pt}}
\put(1262,68){\makebox(0,0){1.2e+06}}
\put(1262.0,632.0){\rule[-0.200pt]{0.400pt}{4.818pt}}
\put(1436.0,113.0){\rule[-0.200pt]{0.400pt}{4.818pt}}
\put(1436,68){\makebox(0,0){1.4e+06}}
\put(1436.0,632.0){\rule[-0.200pt]{0.400pt}{4.818pt}}
\put(220.0,113.0){\rule[-0.200pt]{292.934pt}{0.400pt}}
\put(1436.0,113.0){\rule[-0.200pt]{0.400pt}{129.845pt}}
\put(220.0,652.0){\rule[-0.200pt]{292.934pt}{0.400pt}}
\put(45,382){\makebox(0,0){$[A]$}}
\put(828,23){\makebox(0,0){time}}
\put(220.0,113.0){\rule[-0.200pt]{0.400pt}{129.845pt}}
\put(220,380){\usebox{\plotpoint}}
\put(220,380){\usebox{\plotpoint}}
\put(219.67,378){\rule{0.400pt}{0.482pt}}
\multiput(219.17,379.00)(1.000,-1.000){2}{\rule{0.400pt}{0.241pt}}
\put(220.67,372){\rule{0.400pt}{1.445pt}}
\multiput(220.17,375.00)(1.000,-3.000){2}{\rule{0.400pt}{0.723pt}}
\put(221.67,366){\rule{0.400pt}{1.445pt}}
\multiput(221.17,369.00)(1.000,-3.000){2}{\rule{0.400pt}{0.723pt}}
\put(223,360.67){\rule{0.241pt}{0.400pt}}
\multiput(223.00,361.17)(0.500,-1.000){2}{\rule{0.120pt}{0.400pt}}
\put(223.67,359){\rule{0.400pt}{0.482pt}}
\multiput(223.17,360.00)(1.000,-1.000){2}{\rule{0.400pt}{0.241pt}}
\put(225,357.67){\rule{0.241pt}{0.400pt}}
\multiput(225.00,358.17)(0.500,-1.000){2}{\rule{0.120pt}{0.400pt}}
\put(226,356.67){\rule{0.241pt}{0.400pt}}
\multiput(226.00,357.17)(0.500,-1.000){2}{\rule{0.120pt}{0.400pt}}
\put(226.67,355){\rule{0.400pt}{0.482pt}}
\multiput(226.17,356.00)(1.000,-1.000){2}{\rule{0.400pt}{0.241pt}}
\put(228,353.67){\rule{0.241pt}{0.400pt}}
\multiput(228.00,354.17)(0.500,-1.000){2}{\rule{0.120pt}{0.400pt}}
\put(229,353.67){\rule{0.241pt}{0.400pt}}
\multiput(229.00,353.17)(0.500,1.000){2}{\rule{0.120pt}{0.400pt}}
\put(223.0,362.0){\rule[-0.200pt]{0.400pt}{0.964pt}}
\put(230.0,354.0){\usebox{\plotpoint}}
\put(230.0,354.0){\rule[-0.200pt]{1.686pt}{0.400pt}}
\put(237,352.67){\rule{0.241pt}{0.400pt}}
\multiput(237.00,352.17)(0.500,1.000){2}{\rule{0.120pt}{0.400pt}}
\put(237.0,353.0){\usebox{\plotpoint}}
\put(238.0,354.0){\rule[-0.200pt]{1.204pt}{0.400pt}}
\put(243.0,353.0){\usebox{\plotpoint}}
\put(244,351.67){\rule{0.241pt}{0.400pt}}
\multiput(244.00,352.17)(0.500,-1.000){2}{\rule{0.120pt}{0.400pt}}
\put(243.0,353.0){\usebox{\plotpoint}}
\put(246,350.67){\rule{0.241pt}{0.400pt}}
\multiput(246.00,351.17)(0.500,-1.000){2}{\rule{0.120pt}{0.400pt}}
\put(247,349.67){\rule{0.241pt}{0.400pt}}
\multiput(247.00,350.17)(0.500,-1.000){2}{\rule{0.120pt}{0.400pt}}
\put(245.0,352.0){\usebox{\plotpoint}}
\put(248.0,350.0){\rule[-0.200pt]{0.482pt}{0.400pt}}
\put(250,347.67){\rule{0.241pt}{0.400pt}}
\multiput(250.00,348.17)(0.500,-1.000){2}{\rule{0.120pt}{0.400pt}}
\put(251,346.67){\rule{0.241pt}{0.400pt}}
\multiput(251.00,347.17)(0.500,-1.000){2}{\rule{0.120pt}{0.400pt}}
\put(250.0,349.0){\usebox{\plotpoint}}
\put(253,345.67){\rule{0.241pt}{0.400pt}}
\multiput(253.00,346.17)(0.500,-1.000){2}{\rule{0.120pt}{0.400pt}}
\put(254,344.67){\rule{0.241pt}{0.400pt}}
\multiput(254.00,345.17)(0.500,-1.000){2}{\rule{0.120pt}{0.400pt}}
\put(255,343.67){\rule{0.241pt}{0.400pt}}
\multiput(255.00,344.17)(0.500,-1.000){2}{\rule{0.120pt}{0.400pt}}
\put(252.0,347.0){\usebox{\plotpoint}}
\put(256,344){\usebox{\plotpoint}}
\put(256,342.67){\rule{0.241pt}{0.400pt}}
\multiput(256.00,343.17)(0.500,-1.000){2}{\rule{0.120pt}{0.400pt}}
\put(259,341.67){\rule{0.241pt}{0.400pt}}
\multiput(259.00,342.17)(0.500,-1.000){2}{\rule{0.120pt}{0.400pt}}
\put(260,340.67){\rule{0.241pt}{0.400pt}}
\multiput(260.00,341.17)(0.500,-1.000){2}{\rule{0.120pt}{0.400pt}}
\put(257.0,343.0){\rule[-0.200pt]{0.482pt}{0.400pt}}
\put(261.0,341.0){\rule[-0.200pt]{0.482pt}{0.400pt}}
\put(263.0,340.0){\usebox{\plotpoint}}
\put(264,338.67){\rule{0.241pt}{0.400pt}}
\multiput(264.00,339.17)(0.500,-1.000){2}{\rule{0.120pt}{0.400pt}}
\put(265,337.67){\rule{0.241pt}{0.400pt}}
\multiput(265.00,338.17)(0.500,-1.000){2}{\rule{0.120pt}{0.400pt}}
\put(263.0,340.0){\usebox{\plotpoint}}
\put(266.0,338.0){\usebox{\plotpoint}}
\put(267.0,337.0){\usebox{\plotpoint}}
\put(267.0,337.0){\usebox{\plotpoint}}
\put(267.0,337.0){\usebox{\plotpoint}}
\put(267,336.67){\rule{0.241pt}{0.400pt}}
\multiput(267.00,337.17)(0.500,-1.000){2}{\rule{0.120pt}{0.400pt}}
\put(267.0,337.0){\usebox{\plotpoint}}
\put(269,335.67){\rule{0.241pt}{0.400pt}}
\multiput(269.00,336.17)(0.500,-1.000){2}{\rule{0.120pt}{0.400pt}}
\put(268.0,337.0){\usebox{\plotpoint}}
\put(270,336){\usebox{\plotpoint}}
\put(270.0,336.0){\usebox{\plotpoint}}
\put(271.0,335.0){\usebox{\plotpoint}}
\put(271.67,333){\rule{0.400pt}{0.482pt}}
\multiput(271.17,334.00)(1.000,-1.000){2}{\rule{0.400pt}{0.241pt}}
\put(271.0,335.0){\usebox{\plotpoint}}
\put(275,331.67){\rule{0.241pt}{0.400pt}}
\multiput(275.00,332.17)(0.500,-1.000){2}{\rule{0.120pt}{0.400pt}}
\put(273.0,333.0){\rule[-0.200pt]{0.482pt}{0.400pt}}
\put(276,332){\usebox{\plotpoint}}
\put(276,332){\usebox{\plotpoint}}
\put(276.0,331.0){\usebox{\plotpoint}}
\put(276.0,331.0){\usebox{\plotpoint}}
\put(277.0,331.0){\usebox{\plotpoint}}
\put(277,329.67){\rule{0.241pt}{0.400pt}}
\multiput(277.00,330.17)(0.500,-1.000){2}{\rule{0.120pt}{0.400pt}}
\put(277.0,331.0){\usebox{\plotpoint}}
\put(279,328.67){\rule{0.241pt}{0.400pt}}
\multiput(279.00,329.17)(0.500,-1.000){2}{\rule{0.120pt}{0.400pt}}
\put(278.0,330.0){\usebox{\plotpoint}}
\put(280.0,329.0){\usebox{\plotpoint}}
\put(281.0,328.0){\usebox{\plotpoint}}
\put(281.0,328.0){\rule[-0.200pt]{0.482pt}{0.400pt}}
\put(283,325.67){\rule{0.241pt}{0.400pt}}
\multiput(283.00,326.17)(0.500,-1.000){2}{\rule{0.120pt}{0.400pt}}
\put(283.0,327.0){\usebox{\plotpoint}}
\put(285,324.67){\rule{0.241pt}{0.400pt}}
\multiput(285.00,325.17)(0.500,-1.000){2}{\rule{0.120pt}{0.400pt}}
\put(284.0,326.0){\usebox{\plotpoint}}
\put(289,323.67){\rule{0.241pt}{0.400pt}}
\multiput(289.00,324.17)(0.500,-1.000){2}{\rule{0.120pt}{0.400pt}}
\put(286.0,325.0){\rule[-0.200pt]{0.723pt}{0.400pt}}
\put(291,322.67){\rule{0.241pt}{0.400pt}}
\multiput(291.00,323.17)(0.500,-1.000){2}{\rule{0.120pt}{0.400pt}}
\put(292,321.67){\rule{0.241pt}{0.400pt}}
\multiput(292.00,322.17)(0.500,-1.000){2}{\rule{0.120pt}{0.400pt}}
\put(293,320.67){\rule{0.241pt}{0.400pt}}
\multiput(293.00,321.17)(0.500,-1.000){2}{\rule{0.120pt}{0.400pt}}
\put(294,319.67){\rule{0.241pt}{0.400pt}}
\multiput(294.00,320.17)(0.500,-1.000){2}{\rule{0.120pt}{0.400pt}}
\put(295,318.67){\rule{0.241pt}{0.400pt}}
\multiput(295.00,319.17)(0.500,-1.000){2}{\rule{0.120pt}{0.400pt}}
\put(290.0,324.0){\usebox{\plotpoint}}
\put(296.0,318.0){\usebox{\plotpoint}}
\put(296.0,318.0){\usebox{\plotpoint}}
\put(297,315.67){\rule{0.241pt}{0.400pt}}
\multiput(297.00,316.17)(0.500,-1.000){2}{\rule{0.120pt}{0.400pt}}
\put(297.0,317.0){\usebox{\plotpoint}}
\put(299,314.67){\rule{0.241pt}{0.400pt}}
\multiput(299.00,315.17)(0.500,-1.000){2}{\rule{0.120pt}{0.400pt}}
\put(298.0,316.0){\usebox{\plotpoint}}
\put(300,315){\usebox{\plotpoint}}
\put(300,315){\usebox{\plotpoint}}
\put(300,315){\usebox{\plotpoint}}
\put(300,313.67){\rule{0.241pt}{0.400pt}}
\multiput(300.00,314.17)(0.500,-1.000){2}{\rule{0.120pt}{0.400pt}}
\put(302.67,312){\rule{0.400pt}{0.482pt}}
\multiput(302.17,313.00)(1.000,-1.000){2}{\rule{0.400pt}{0.241pt}}
\put(301.0,314.0){\rule[-0.200pt]{0.482pt}{0.400pt}}
\put(305,310.67){\rule{0.241pt}{0.400pt}}
\multiput(305.00,311.17)(0.500,-1.000){2}{\rule{0.120pt}{0.400pt}}
\put(306,309.67){\rule{0.241pt}{0.400pt}}
\multiput(306.00,310.17)(0.500,-1.000){2}{\rule{0.120pt}{0.400pt}}
\put(304.0,312.0){\usebox{\plotpoint}}
\put(307,310){\usebox{\plotpoint}}
\put(307,310){\usebox{\plotpoint}}
\put(307,310){\usebox{\plotpoint}}
\put(307,310){\usebox{\plotpoint}}
\put(307,310){\usebox{\plotpoint}}
\put(307,310){\usebox{\plotpoint}}
\put(307,310){\usebox{\plotpoint}}
\put(307,310){\usebox{\plotpoint}}
\put(307,310){\usebox{\plotpoint}}
\put(307,308.67){\rule{0.241pt}{0.400pt}}
\multiput(307.00,309.17)(0.500,-1.000){2}{\rule{0.120pt}{0.400pt}}
\put(308.0,309.0){\usebox{\plotpoint}}
\put(309,306.67){\rule{0.241pt}{0.400pt}}
\multiput(309.00,307.17)(0.500,-1.000){2}{\rule{0.120pt}{0.400pt}}
\put(310,305.67){\rule{0.241pt}{0.400pt}}
\multiput(310.00,306.17)(0.500,-1.000){2}{\rule{0.120pt}{0.400pt}}
\put(311,304.67){\rule{0.241pt}{0.400pt}}
\multiput(311.00,305.17)(0.500,-1.000){2}{\rule{0.120pt}{0.400pt}}
\put(312,303.67){\rule{0.241pt}{0.400pt}}
\multiput(312.00,304.17)(0.500,-1.000){2}{\rule{0.120pt}{0.400pt}}
\put(313,302.67){\rule{0.241pt}{0.400pt}}
\multiput(313.00,303.17)(0.500,-1.000){2}{\rule{0.120pt}{0.400pt}}
\put(314,301.67){\rule{0.241pt}{0.400pt}}
\multiput(314.00,302.17)(0.500,-1.000){2}{\rule{0.120pt}{0.400pt}}
\put(315,300.67){\rule{0.241pt}{0.400pt}}
\multiput(315.00,301.17)(0.500,-1.000){2}{\rule{0.120pt}{0.400pt}}
\put(309.0,308.0){\usebox{\plotpoint}}
\put(316,301){\usebox{\plotpoint}}
\put(316,299.67){\rule{0.241pt}{0.400pt}}
\multiput(316.00,300.17)(0.500,-1.000){2}{\rule{0.120pt}{0.400pt}}
\put(317,298.67){\rule{0.241pt}{0.400pt}}
\multiput(317.00,299.17)(0.500,-1.000){2}{\rule{0.120pt}{0.400pt}}
\put(318.0,299.0){\rule[-0.200pt]{0.964pt}{0.400pt}}
\put(322,296.67){\rule{0.241pt}{0.400pt}}
\multiput(322.00,297.17)(0.500,-1.000){2}{\rule{0.120pt}{0.400pt}}
\put(323,295.67){\rule{0.241pt}{0.400pt}}
\multiput(323.00,296.17)(0.500,-1.000){2}{\rule{0.120pt}{0.400pt}}
\put(324,294.67){\rule{0.241pt}{0.400pt}}
\multiput(324.00,295.17)(0.500,-1.000){2}{\rule{0.120pt}{0.400pt}}
\put(322.0,298.0){\usebox{\plotpoint}}
\put(325,295){\usebox{\plotpoint}}
\put(325,295){\usebox{\plotpoint}}
\put(325.0,294.0){\usebox{\plotpoint}}
\put(327,292.67){\rule{0.241pt}{0.400pt}}
\multiput(327.00,293.17)(0.500,-1.000){2}{\rule{0.120pt}{0.400pt}}
\put(328,291.67){\rule{0.241pt}{0.400pt}}
\multiput(328.00,292.17)(0.500,-1.000){2}{\rule{0.120pt}{0.400pt}}
\put(325.0,294.0){\rule[-0.200pt]{0.482pt}{0.400pt}}
\put(329,289.67){\rule{0.241pt}{0.400pt}}
\multiput(329.00,290.17)(0.500,-1.000){2}{\rule{0.120pt}{0.400pt}}
\put(330,288.67){\rule{0.241pt}{0.400pt}}
\multiput(330.00,289.17)(0.500,-1.000){2}{\rule{0.120pt}{0.400pt}}
\put(329.0,291.0){\usebox{\plotpoint}}
\put(332,287.67){\rule{0.241pt}{0.400pt}}
\multiput(332.00,288.17)(0.500,-1.000){2}{\rule{0.120pt}{0.400pt}}
\put(333,286.67){\rule{0.241pt}{0.400pt}}
\multiput(333.00,287.17)(0.500,-1.000){2}{\rule{0.120pt}{0.400pt}}
\put(331.0,289.0){\usebox{\plotpoint}}
\put(335,285.67){\rule{0.241pt}{0.400pt}}
\multiput(335.00,286.17)(0.500,-1.000){2}{\rule{0.120pt}{0.400pt}}
\put(334.0,287.0){\usebox{\plotpoint}}
\put(335.67,285){\rule{0.400pt}{0.482pt}}
\multiput(335.17,286.00)(1.000,-1.000){2}{\rule{0.400pt}{0.241pt}}
\put(336.0,286.0){\usebox{\plotpoint}}
\put(340,283.67){\rule{0.241pt}{0.400pt}}
\multiput(340.00,284.17)(0.500,-1.000){2}{\rule{0.120pt}{0.400pt}}
\put(341,282.67){\rule{0.241pt}{0.400pt}}
\multiput(341.00,283.17)(0.500,-1.000){2}{\rule{0.120pt}{0.400pt}}
\put(337.0,285.0){\rule[-0.200pt]{0.723pt}{0.400pt}}
\put(342,280.67){\rule{0.241pt}{0.400pt}}
\multiput(342.00,281.17)(0.500,-1.000){2}{\rule{0.120pt}{0.400pt}}
\put(342.0,282.0){\usebox{\plotpoint}}
\put(343,281){\usebox{\plotpoint}}
\put(343,281){\usebox{\plotpoint}}
\put(343,281){\usebox{\plotpoint}}
\put(344.67,279){\rule{0.400pt}{0.482pt}}
\multiput(344.17,280.00)(1.000,-1.000){2}{\rule{0.400pt}{0.241pt}}
\put(343.0,281.0){\rule[-0.200pt]{0.482pt}{0.400pt}}
\put(347,277.67){\rule{0.241pt}{0.400pt}}
\multiput(347.00,278.17)(0.500,-1.000){2}{\rule{0.120pt}{0.400pt}}
\put(346.0,279.0){\usebox{\plotpoint}}
\put(349,276.67){\rule{0.241pt}{0.400pt}}
\multiput(349.00,277.17)(0.500,-1.000){2}{\rule{0.120pt}{0.400pt}}
\put(350,275.67){\rule{0.241pt}{0.400pt}}
\multiput(350.00,276.17)(0.500,-1.000){2}{\rule{0.120pt}{0.400pt}}
\put(348.0,278.0){\usebox{\plotpoint}}
\put(351,276){\usebox{\plotpoint}}
\put(351,276){\usebox{\plotpoint}}
\put(351,276){\usebox{\plotpoint}}
\put(351,276){\usebox{\plotpoint}}
\put(351,276){\usebox{\plotpoint}}
\put(351,276){\usebox{\plotpoint}}
\put(351,276){\usebox{\plotpoint}}
\put(351,274.67){\rule{0.241pt}{0.400pt}}
\multiput(351.00,275.17)(0.500,-1.000){2}{\rule{0.120pt}{0.400pt}}
\put(352,275){\usebox{\plotpoint}}
\put(352,275){\usebox{\plotpoint}}
\put(352,275){\usebox{\plotpoint}}
\put(352,275){\usebox{\plotpoint}}
\put(352,275){\usebox{\plotpoint}}
\put(352,275){\usebox{\plotpoint}}
\put(352,275){\usebox{\plotpoint}}
\put(352.0,275.0){\usebox{\plotpoint}}
\put(353,272.67){\rule{0.241pt}{0.400pt}}
\multiput(353.00,273.17)(0.500,-1.000){2}{\rule{0.120pt}{0.400pt}}
\put(353.0,274.0){\usebox{\plotpoint}}
\put(354.0,273.0){\usebox{\plotpoint}}
\put(355,270.67){\rule{0.241pt}{0.400pt}}
\multiput(355.00,271.17)(0.500,-1.000){2}{\rule{0.120pt}{0.400pt}}
\put(355.0,272.0){\usebox{\plotpoint}}
\put(358,269.67){\rule{0.241pt}{0.400pt}}
\multiput(358.00,270.17)(0.500,-1.000){2}{\rule{0.120pt}{0.400pt}}
\put(359,268.67){\rule{0.241pt}{0.400pt}}
\multiput(359.00,269.17)(0.500,-1.000){2}{\rule{0.120pt}{0.400pt}}
\put(356.0,271.0){\rule[-0.200pt]{0.482pt}{0.400pt}}
\put(360,269){\usebox{\plotpoint}}
\put(360,269){\usebox{\plotpoint}}
\put(360,269){\usebox{\plotpoint}}
\put(360,269){\usebox{\plotpoint}}
\put(361,267.67){\rule{0.241pt}{0.400pt}}
\multiput(361.00,268.17)(0.500,-1.000){2}{\rule{0.120pt}{0.400pt}}
\put(360.0,269.0){\usebox{\plotpoint}}
\put(362,268){\usebox{\plotpoint}}
\put(362,266.67){\rule{0.241pt}{0.400pt}}
\multiput(362.00,267.17)(0.500,-1.000){2}{\rule{0.120pt}{0.400pt}}
\put(364,265.67){\rule{0.241pt}{0.400pt}}
\multiput(364.00,266.17)(0.500,-1.000){2}{\rule{0.120pt}{0.400pt}}
\put(365,264.67){\rule{0.241pt}{0.400pt}}
\multiput(365.00,265.17)(0.500,-1.000){2}{\rule{0.120pt}{0.400pt}}
\put(366,263.67){\rule{0.241pt}{0.400pt}}
\multiput(366.00,264.17)(0.500,-1.000){2}{\rule{0.120pt}{0.400pt}}
\put(363.0,267.0){\usebox{\plotpoint}}
\put(367.0,264.0){\rule[-0.200pt]{0.482pt}{0.400pt}}
\put(369,261.67){\rule{0.241pt}{0.400pt}}
\multiput(369.00,262.17)(0.500,-1.000){2}{\rule{0.120pt}{0.400pt}}
\put(369.0,263.0){\usebox{\plotpoint}}
\put(372,260.67){\rule{0.241pt}{0.400pt}}
\multiput(372.00,261.17)(0.500,-1.000){2}{\rule{0.120pt}{0.400pt}}
\put(370.0,262.0){\rule[-0.200pt]{0.482pt}{0.400pt}}
\put(373.67,259){\rule{0.400pt}{0.482pt}}
\multiput(373.17,260.00)(1.000,-1.000){2}{\rule{0.400pt}{0.241pt}}
\put(373.0,261.0){\usebox{\plotpoint}}
\put(375,259){\usebox{\plotpoint}}
\put(376,258.67){\rule{0.241pt}{0.400pt}}
\multiput(376.00,258.17)(0.500,1.000){2}{\rule{0.120pt}{0.400pt}}
\put(375.0,259.0){\usebox{\plotpoint}}
\put(381,258.67){\rule{0.241pt}{0.400pt}}
\multiput(381.00,259.17)(0.500,-1.000){2}{\rule{0.120pt}{0.400pt}}
\put(377.0,260.0){\rule[-0.200pt]{0.964pt}{0.400pt}}
\put(382,259){\usebox{\plotpoint}}
\put(382,257.67){\rule{0.241pt}{0.400pt}}
\multiput(382.00,258.17)(0.500,-1.000){2}{\rule{0.120pt}{0.400pt}}
\put(386,256.67){\rule{0.241pt}{0.400pt}}
\multiput(386.00,257.17)(0.500,-1.000){2}{\rule{0.120pt}{0.400pt}}
\put(383.0,258.0){\rule[-0.200pt]{0.723pt}{0.400pt}}
\put(388,255.67){\rule{0.241pt}{0.400pt}}
\multiput(388.00,256.17)(0.500,-1.000){2}{\rule{0.120pt}{0.400pt}}
\put(387.0,257.0){\usebox{\plotpoint}}
\put(389,256){\usebox{\plotpoint}}
\put(389,255.67){\rule{0.241pt}{0.400pt}}
\multiput(389.00,255.17)(0.500,1.000){2}{\rule{0.120pt}{0.400pt}}
\put(390,257){\usebox{\plotpoint}}
\put(390,257){\usebox{\plotpoint}}
\put(390,257){\usebox{\plotpoint}}
\put(390,257){\usebox{\plotpoint}}
\put(390,257){\usebox{\plotpoint}}
\put(390,257){\usebox{\plotpoint}}
\put(390,257){\usebox{\plotpoint}}
\put(390,257){\usebox{\plotpoint}}
\put(390,257){\usebox{\plotpoint}}
\put(390,257){\usebox{\plotpoint}}
\put(390,257){\usebox{\plotpoint}}
\put(390.0,257.0){\rule[-0.200pt]{1.204pt}{0.400pt}}
\put(395.0,256.0){\usebox{\plotpoint}}
\put(396,254.67){\rule{0.241pt}{0.400pt}}
\multiput(396.00,255.17)(0.500,-1.000){2}{\rule{0.120pt}{0.400pt}}
\put(395.0,256.0){\usebox{\plotpoint}}
\put(398,253.67){\rule{0.241pt}{0.400pt}}
\multiput(398.00,254.17)(0.500,-1.000){2}{\rule{0.120pt}{0.400pt}}
\put(397.0,255.0){\usebox{\plotpoint}}
\put(400,252.67){\rule{0.241pt}{0.400pt}}
\multiput(400.00,253.17)(0.500,-1.000){2}{\rule{0.120pt}{0.400pt}}
\put(399.0,254.0){\usebox{\plotpoint}}
\put(401,253){\usebox{\plotpoint}}
\put(401,253){\usebox{\plotpoint}}
\put(401,253){\usebox{\plotpoint}}
\put(401,253){\usebox{\plotpoint}}
\put(401,253){\usebox{\plotpoint}}
\put(401,253){\usebox{\plotpoint}}
\put(401,253){\usebox{\plotpoint}}
\put(401,253){\usebox{\plotpoint}}
\put(401,253){\usebox{\plotpoint}}
\put(401,253){\usebox{\plotpoint}}
\put(401,253){\usebox{\plotpoint}}
\put(401,253){\usebox{\plotpoint}}
\put(401.0,253.0){\usebox{\plotpoint}}
\put(402,249.67){\rule{0.241pt}{0.400pt}}
\multiput(402.00,250.17)(0.500,-1.000){2}{\rule{0.120pt}{0.400pt}}
\put(403,248.67){\rule{0.241pt}{0.400pt}}
\multiput(403.00,249.17)(0.500,-1.000){2}{\rule{0.120pt}{0.400pt}}
\put(404,248.67){\rule{0.241pt}{0.400pt}}
\multiput(404.00,248.17)(0.500,1.000){2}{\rule{0.120pt}{0.400pt}}
\put(402.0,251.0){\rule[-0.200pt]{0.400pt}{0.482pt}}
\put(407,248.67){\rule{0.241pt}{0.400pt}}
\multiput(407.00,249.17)(0.500,-1.000){2}{\rule{0.120pt}{0.400pt}}
\put(405.0,250.0){\rule[-0.200pt]{0.482pt}{0.400pt}}
\put(408.0,248.0){\usebox{\plotpoint}}
\put(410,246.67){\rule{0.241pt}{0.400pt}}
\multiput(410.00,247.17)(0.500,-1.000){2}{\rule{0.120pt}{0.400pt}}
\put(411,245.67){\rule{0.241pt}{0.400pt}}
\multiput(411.00,246.17)(0.500,-1.000){2}{\rule{0.120pt}{0.400pt}}
\put(412,244.67){\rule{0.241pt}{0.400pt}}
\multiput(412.00,245.17)(0.500,-1.000){2}{\rule{0.120pt}{0.400pt}}
\put(408.0,248.0){\rule[-0.200pt]{0.482pt}{0.400pt}}
\put(413.0,245.0){\rule[-0.200pt]{0.482pt}{0.400pt}}
\put(415.0,244.0){\usebox{\plotpoint}}
\put(418,242.67){\rule{0.241pt}{0.400pt}}
\multiput(418.00,243.17)(0.500,-1.000){2}{\rule{0.120pt}{0.400pt}}
\put(415.0,244.0){\rule[-0.200pt]{0.723pt}{0.400pt}}
\put(421,241.67){\rule{0.241pt}{0.400pt}}
\multiput(421.00,242.17)(0.500,-1.000){2}{\rule{0.120pt}{0.400pt}}
\put(419.0,243.0){\rule[-0.200pt]{0.482pt}{0.400pt}}
\put(422,242){\usebox{\plotpoint}}
\put(424,240.67){\rule{0.241pt}{0.400pt}}
\multiput(424.00,241.17)(0.500,-1.000){2}{\rule{0.120pt}{0.400pt}}
\put(422.0,242.0){\rule[-0.200pt]{0.482pt}{0.400pt}}
\put(426,239.67){\rule{0.241pt}{0.400pt}}
\multiput(426.00,240.17)(0.500,-1.000){2}{\rule{0.120pt}{0.400pt}}
\put(425.0,241.0){\usebox{\plotpoint}}
\put(429,238.67){\rule{0.241pt}{0.400pt}}
\multiput(429.00,239.17)(0.500,-1.000){2}{\rule{0.120pt}{0.400pt}}
\put(427.0,240.0){\rule[-0.200pt]{0.482pt}{0.400pt}}
\put(436,238.67){\rule{0.241pt}{0.400pt}}
\multiput(436.00,238.17)(0.500,1.000){2}{\rule{0.120pt}{0.400pt}}
\put(430.0,239.0){\rule[-0.200pt]{1.445pt}{0.400pt}}
\put(438,238.67){\rule{0.241pt}{0.400pt}}
\multiput(438.00,239.17)(0.500,-1.000){2}{\rule{0.120pt}{0.400pt}}
\put(437.0,240.0){\usebox{\plotpoint}}
\put(439,239){\usebox{\plotpoint}}
\put(439,239){\usebox{\plotpoint}}
\put(439,239){\usebox{\plotpoint}}
\put(439,239){\usebox{\plotpoint}}
\put(440,237.67){\rule{0.241pt}{0.400pt}}
\multiput(440.00,238.17)(0.500,-1.000){2}{\rule{0.120pt}{0.400pt}}
\put(439.0,239.0){\usebox{\plotpoint}}
\put(441,238){\usebox{\plotpoint}}
\put(441,237.67){\rule{0.241pt}{0.400pt}}
\multiput(441.00,237.17)(0.500,1.000){2}{\rule{0.120pt}{0.400pt}}
\put(442,239){\usebox{\plotpoint}}
\put(442,239){\usebox{\plotpoint}}
\put(442,239){\usebox{\plotpoint}}
\put(446,238.67){\rule{0.241pt}{0.400pt}}
\multiput(446.00,238.17)(0.500,1.000){2}{\rule{0.120pt}{0.400pt}}
\put(442.0,239.0){\rule[-0.200pt]{0.964pt}{0.400pt}}
\put(447.0,240.0){\usebox{\plotpoint}}
\put(448,238.67){\rule{0.241pt}{0.400pt}}
\multiput(448.00,238.17)(0.500,1.000){2}{\rule{0.120pt}{0.400pt}}
\put(448.0,239.0){\usebox{\plotpoint}}
\put(452,239.67){\rule{0.241pt}{0.400pt}}
\multiput(452.00,239.17)(0.500,1.000){2}{\rule{0.120pt}{0.400pt}}
\put(449.0,240.0){\rule[-0.200pt]{0.723pt}{0.400pt}}
\put(455,240.67){\rule{0.241pt}{0.400pt}}
\multiput(455.00,240.17)(0.500,1.000){2}{\rule{0.120pt}{0.400pt}}
\put(456,241.67){\rule{0.241pt}{0.400pt}}
\multiput(456.00,241.17)(0.500,1.000){2}{\rule{0.120pt}{0.400pt}}
\put(453.0,241.0){\rule[-0.200pt]{0.482pt}{0.400pt}}
\put(460,242.67){\rule{0.241pt}{0.400pt}}
\multiput(460.00,242.17)(0.500,1.000){2}{\rule{0.120pt}{0.400pt}}
\put(457.0,243.0){\rule[-0.200pt]{0.723pt}{0.400pt}}
\put(461,244){\usebox{\plotpoint}}
\put(461,243.67){\rule{0.241pt}{0.400pt}}
\multiput(461.00,243.17)(0.500,1.000){2}{\rule{0.120pt}{0.400pt}}
\put(462.0,245.0){\rule[-0.200pt]{1.445pt}{0.400pt}}
\put(468.0,244.0){\usebox{\plotpoint}}
\put(475,242.67){\rule{0.241pt}{0.400pt}}
\multiput(475.00,243.17)(0.500,-1.000){2}{\rule{0.120pt}{0.400pt}}
\put(468.0,244.0){\rule[-0.200pt]{1.686pt}{0.400pt}}
\put(479,242.67){\rule{0.241pt}{0.400pt}}
\multiput(479.00,242.17)(0.500,1.000){2}{\rule{0.120pt}{0.400pt}}
\put(480,242.67){\rule{0.241pt}{0.400pt}}
\multiput(480.00,243.17)(0.500,-1.000){2}{\rule{0.120pt}{0.400pt}}
\put(476.0,243.0){\rule[-0.200pt]{0.723pt}{0.400pt}}
\put(481,243){\usebox{\plotpoint}}
\put(481,243){\usebox{\plotpoint}}
\put(481,243){\usebox{\plotpoint}}
\put(481,243){\usebox{\plotpoint}}
\put(481,243){\usebox{\plotpoint}}
\put(481,243){\usebox{\plotpoint}}
\put(481,243){\usebox{\plotpoint}}
\put(481,243){\usebox{\plotpoint}}
\put(481,243){\usebox{\plotpoint}}
\put(481,243){\usebox{\plotpoint}}
\put(482,242.67){\rule{0.241pt}{0.400pt}}
\multiput(482.00,242.17)(0.500,1.000){2}{\rule{0.120pt}{0.400pt}}
\put(481.0,243.0){\usebox{\plotpoint}}
\put(483,244){\usebox{\plotpoint}}
\put(483,244){\usebox{\plotpoint}}
\put(483,244){\usebox{\plotpoint}}
\put(483,243.67){\rule{0.241pt}{0.400pt}}
\multiput(483.00,243.17)(0.500,1.000){2}{\rule{0.120pt}{0.400pt}}
\put(484,245){\usebox{\plotpoint}}
\put(484,245){\usebox{\plotpoint}}
\put(484,245){\usebox{\plotpoint}}
\put(484,245){\usebox{\plotpoint}}
\put(484,245){\usebox{\plotpoint}}
\put(484,245){\usebox{\plotpoint}}
\put(484,245){\usebox{\plotpoint}}
\put(484,245){\usebox{\plotpoint}}
\put(484,245){\usebox{\plotpoint}}
\put(484,245){\usebox{\plotpoint}}
\put(484,245){\usebox{\plotpoint}}
\put(484,245){\usebox{\plotpoint}}
\put(484,245){\usebox{\plotpoint}}
\put(484,245){\usebox{\plotpoint}}
\put(484,245){\usebox{\plotpoint}}
\put(484,245){\usebox{\plotpoint}}
\put(484,245){\usebox{\plotpoint}}
\put(484,245){\usebox{\plotpoint}}
\put(484,245){\usebox{\plotpoint}}
\put(484,245){\usebox{\plotpoint}}
\put(484,245){\usebox{\plotpoint}}
\put(484,244.67){\rule{0.241pt}{0.400pt}}
\multiput(484.00,244.17)(0.500,1.000){2}{\rule{0.120pt}{0.400pt}}
\put(485,246){\usebox{\plotpoint}}
\put(485,246){\usebox{\plotpoint}}
\put(485,246){\usebox{\plotpoint}}
\put(485,246){\usebox{\plotpoint}}
\put(485,246){\usebox{\plotpoint}}
\put(485.0,246.0){\rule[-0.200pt]{0.482pt}{0.400pt}}
\put(487,245.67){\rule{0.241pt}{0.400pt}}
\multiput(487.00,246.17)(0.500,-1.000){2}{\rule{0.120pt}{0.400pt}}
\put(487.0,246.0){\usebox{\plotpoint}}
\put(488,246){\usebox{\plotpoint}}
\put(488,246){\usebox{\plotpoint}}
\put(488,246){\usebox{\plotpoint}}
\put(488,246){\usebox{\plotpoint}}
\put(488,246){\usebox{\plotpoint}}
\put(488,246){\usebox{\plotpoint}}
\put(488,246){\usebox{\plotpoint}}
\put(488,246){\usebox{\plotpoint}}
\put(488,246){\usebox{\plotpoint}}
\put(488,246){\usebox{\plotpoint}}
\put(488,246){\usebox{\plotpoint}}
\put(488,246){\usebox{\plotpoint}}
\put(488,246){\usebox{\plotpoint}}
\put(488,246){\usebox{\plotpoint}}
\put(488,246){\usebox{\plotpoint}}
\put(488,246){\usebox{\plotpoint}}
\put(488,246){\usebox{\plotpoint}}
\put(488,245.67){\rule{0.241pt}{0.400pt}}
\multiput(488.00,245.17)(0.500,1.000){2}{\rule{0.120pt}{0.400pt}}
\put(491,246.67){\rule{0.241pt}{0.400pt}}
\multiput(491.00,246.17)(0.500,1.000){2}{\rule{0.120pt}{0.400pt}}
\put(489.0,247.0){\rule[-0.200pt]{0.482pt}{0.400pt}}
\put(492,248){\usebox{\plotpoint}}
\put(492,248){\usebox{\plotpoint}}
\put(492,248){\usebox{\plotpoint}}
\put(492,248){\usebox{\plotpoint}}
\put(492,248){\usebox{\plotpoint}}
\put(492,248){\usebox{\plotpoint}}
\put(492,248){\usebox{\plotpoint}}
\put(492,248){\usebox{\plotpoint}}
\put(492,248){\usebox{\plotpoint}}
\put(492,248){\usebox{\plotpoint}}
\put(492,248){\usebox{\plotpoint}}
\put(493,247.67){\rule{0.241pt}{0.400pt}}
\multiput(493.00,247.17)(0.500,1.000){2}{\rule{0.120pt}{0.400pt}}
\put(492.0,248.0){\usebox{\plotpoint}}
\put(494,249){\usebox{\plotpoint}}
\put(495,248.67){\rule{0.241pt}{0.400pt}}
\multiput(495.00,248.17)(0.500,1.000){2}{\rule{0.120pt}{0.400pt}}
\put(494.0,249.0){\usebox{\plotpoint}}
\put(497,249.67){\rule{0.241pt}{0.400pt}}
\multiput(497.00,249.17)(0.500,1.000){2}{\rule{0.120pt}{0.400pt}}
\put(496.0,250.0){\usebox{\plotpoint}}
\put(498,251){\usebox{\plotpoint}}
\put(498,251){\usebox{\plotpoint}}
\put(498,251){\usebox{\plotpoint}}
\put(499,250.67){\rule{0.241pt}{0.400pt}}
\multiput(499.00,250.17)(0.500,1.000){2}{\rule{0.120pt}{0.400pt}}
\put(498.0,251.0){\usebox{\plotpoint}}
\put(503,251.67){\rule{0.241pt}{0.400pt}}
\multiput(503.00,251.17)(0.500,1.000){2}{\rule{0.120pt}{0.400pt}}
\put(500.0,252.0){\rule[-0.200pt]{0.723pt}{0.400pt}}
\put(506,252.67){\rule{0.241pt}{0.400pt}}
\multiput(506.00,252.17)(0.500,1.000){2}{\rule{0.120pt}{0.400pt}}
\put(504.0,253.0){\rule[-0.200pt]{0.482pt}{0.400pt}}
\put(507,254){\usebox{\plotpoint}}
\put(507,253.67){\rule{0.241pt}{0.400pt}}
\multiput(507.00,253.17)(0.500,1.000){2}{\rule{0.120pt}{0.400pt}}
\put(508,254.67){\rule{0.241pt}{0.400pt}}
\multiput(508.00,254.17)(0.500,1.000){2}{\rule{0.120pt}{0.400pt}}
\put(511,255.67){\rule{0.241pt}{0.400pt}}
\multiput(511.00,255.17)(0.500,1.000){2}{\rule{0.120pt}{0.400pt}}
\put(509.0,256.0){\rule[-0.200pt]{0.482pt}{0.400pt}}
\put(513,256.67){\rule{0.241pt}{0.400pt}}
\multiput(513.00,256.17)(0.500,1.000){2}{\rule{0.120pt}{0.400pt}}
\put(512.0,257.0){\usebox{\plotpoint}}
\put(514,258){\usebox{\plotpoint}}
\put(514,257.67){\rule{0.241pt}{0.400pt}}
\multiput(514.00,257.17)(0.500,1.000){2}{\rule{0.120pt}{0.400pt}}
\put(515,258.67){\rule{0.241pt}{0.400pt}}
\multiput(515.00,258.17)(0.500,1.000){2}{\rule{0.120pt}{0.400pt}}
\put(517,259.67){\rule{0.241pt}{0.400pt}}
\multiput(517.00,259.17)(0.500,1.000){2}{\rule{0.120pt}{0.400pt}}
\put(517.67,261){\rule{0.400pt}{0.482pt}}
\multiput(517.17,261.00)(1.000,1.000){2}{\rule{0.400pt}{0.241pt}}
\put(518.67,263){\rule{0.400pt}{0.482pt}}
\multiput(518.17,263.00)(1.000,1.000){2}{\rule{0.400pt}{0.241pt}}
\put(516.0,260.0){\usebox{\plotpoint}}
\put(520.0,265.0){\usebox{\plotpoint}}
\put(521.0,265.0){\usebox{\plotpoint}}
\put(522,265.67){\rule{0.241pt}{0.400pt}}
\multiput(522.00,265.17)(0.500,1.000){2}{\rule{0.120pt}{0.400pt}}
\put(521.0,266.0){\usebox{\plotpoint}}
\put(525,266.67){\rule{0.241pt}{0.400pt}}
\multiput(525.00,266.17)(0.500,1.000){2}{\rule{0.120pt}{0.400pt}}
\put(523.0,267.0){\rule[-0.200pt]{0.482pt}{0.400pt}}
\put(526.0,268.0){\usebox{\plotpoint}}
\put(527.0,268.0){\usebox{\plotpoint}}
\put(529,268.67){\rule{0.241pt}{0.400pt}}
\multiput(529.00,268.17)(0.500,1.000){2}{\rule{0.120pt}{0.400pt}}
\put(527.0,269.0){\rule[-0.200pt]{0.482pt}{0.400pt}}
\put(530,270){\usebox{\plotpoint}}
\put(530,270){\usebox{\plotpoint}}
\put(530,270){\usebox{\plotpoint}}
\put(530,270){\usebox{\plotpoint}}
\put(530,270){\usebox{\plotpoint}}
\put(530,270){\usebox{\plotpoint}}
\put(530,270){\usebox{\plotpoint}}
\put(530,270){\usebox{\plotpoint}}
\put(530,270){\usebox{\plotpoint}}
\put(530,270){\usebox{\plotpoint}}
\put(530.0,270.0){\usebox{\plotpoint}}
\put(530.0,271.0){\rule[-0.200pt]{0.964pt}{0.400pt}}
\put(534,271.67){\rule{0.241pt}{0.400pt}}
\multiput(534.00,271.17)(0.500,1.000){2}{\rule{0.120pt}{0.400pt}}
\put(534.0,271.0){\usebox{\plotpoint}}
\put(536,272.67){\rule{0.241pt}{0.400pt}}
\multiput(536.00,272.17)(0.500,1.000){2}{\rule{0.120pt}{0.400pt}}
\put(535.0,273.0){\usebox{\plotpoint}}
\put(537,274){\usebox{\plotpoint}}
\put(537,274){\usebox{\plotpoint}}
\put(537,274){\usebox{\plotpoint}}
\put(537,274){\usebox{\plotpoint}}
\put(537,274){\usebox{\plotpoint}}
\put(537,274){\usebox{\plotpoint}}
\put(537,274){\usebox{\plotpoint}}
\put(537,274){\usebox{\plotpoint}}
\put(537,274){\usebox{\plotpoint}}
\put(537.0,274.0){\usebox{\plotpoint}}
\put(538,274.67){\rule{0.241pt}{0.400pt}}
\multiput(538.00,274.17)(0.500,1.000){2}{\rule{0.120pt}{0.400pt}}
\put(537.0,275.0){\usebox{\plotpoint}}
\put(540,275.67){\rule{0.241pt}{0.400pt}}
\multiput(540.00,275.17)(0.500,1.000){2}{\rule{0.120pt}{0.400pt}}
\put(539.0,276.0){\usebox{\plotpoint}}
\put(541,277.67){\rule{0.241pt}{0.400pt}}
\multiput(541.00,277.17)(0.500,1.000){2}{\rule{0.120pt}{0.400pt}}
\put(541.0,277.0){\usebox{\plotpoint}}
\put(543,278.67){\rule{0.241pt}{0.400pt}}
\multiput(543.00,278.17)(0.500,1.000){2}{\rule{0.120pt}{0.400pt}}
\put(544,279.67){\rule{0.241pt}{0.400pt}}
\multiput(544.00,279.17)(0.500,1.000){2}{\rule{0.120pt}{0.400pt}}
\put(542.0,279.0){\usebox{\plotpoint}}
\put(546,280.67){\rule{0.241pt}{0.400pt}}
\multiput(546.00,280.17)(0.500,1.000){2}{\rule{0.120pt}{0.400pt}}
\put(545.0,281.0){\usebox{\plotpoint}}
\put(547,282){\usebox{\plotpoint}}
\put(548,281.67){\rule{0.241pt}{0.400pt}}
\multiput(548.00,281.17)(0.500,1.000){2}{\rule{0.120pt}{0.400pt}}
\put(547.0,282.0){\usebox{\plotpoint}}
\put(550,282.67){\rule{0.241pt}{0.400pt}}
\multiput(550.00,282.17)(0.500,1.000){2}{\rule{0.120pt}{0.400pt}}
\put(551,282.67){\rule{0.241pt}{0.400pt}}
\multiput(551.00,283.17)(0.500,-1.000){2}{\rule{0.120pt}{0.400pt}}
\put(551.67,283){\rule{0.400pt}{0.482pt}}
\multiput(551.17,283.00)(1.000,1.000){2}{\rule{0.400pt}{0.241pt}}
\put(553,284.67){\rule{0.241pt}{0.400pt}}
\multiput(553.00,284.17)(0.500,1.000){2}{\rule{0.120pt}{0.400pt}}
\put(549.0,283.0){\usebox{\plotpoint}}
\put(554,286.67){\rule{0.241pt}{0.400pt}}
\multiput(554.00,286.17)(0.500,1.000){2}{\rule{0.120pt}{0.400pt}}
\put(555,287.67){\rule{0.241pt}{0.400pt}}
\multiput(555.00,287.17)(0.500,1.000){2}{\rule{0.120pt}{0.400pt}}
\put(554.0,286.0){\usebox{\plotpoint}}
\put(558,288.67){\rule{0.241pt}{0.400pt}}
\multiput(558.00,288.17)(0.500,1.000){2}{\rule{0.120pt}{0.400pt}}
\put(558.67,290){\rule{0.400pt}{0.482pt}}
\multiput(558.17,290.00)(1.000,1.000){2}{\rule{0.400pt}{0.241pt}}
\put(556.0,289.0){\rule[-0.200pt]{0.482pt}{0.400pt}}
\put(560,292){\usebox{\plotpoint}}
\put(560,292){\usebox{\plotpoint}}
\put(560,292){\usebox{\plotpoint}}
\put(560.0,292.0){\usebox{\plotpoint}}
\put(561.0,292.0){\usebox{\plotpoint}}
\put(561.0,293.0){\rule[-0.200pt]{0.482pt}{0.400pt}}
\put(563.0,293.0){\usebox{\plotpoint}}
\put(564,293.67){\rule{0.241pt}{0.400pt}}
\multiput(564.00,293.17)(0.500,1.000){2}{\rule{0.120pt}{0.400pt}}
\put(565,294.67){\rule{0.241pt}{0.400pt}}
\multiput(565.00,294.17)(0.500,1.000){2}{\rule{0.120pt}{0.400pt}}
\put(563.0,294.0){\usebox{\plotpoint}}
\put(566.0,296.0){\usebox{\plotpoint}}
\put(567.0,296.0){\usebox{\plotpoint}}
\put(567.0,297.0){\usebox{\plotpoint}}
\put(568.0,297.0){\usebox{\plotpoint}}
\put(569,297.67){\rule{0.241pt}{0.400pt}}
\multiput(569.00,297.17)(0.500,1.000){2}{\rule{0.120pt}{0.400pt}}
\put(570,298.67){\rule{0.241pt}{0.400pt}}
\multiput(570.00,298.17)(0.500,1.000){2}{\rule{0.120pt}{0.400pt}}
\put(568.0,298.0){\usebox{\plotpoint}}
\put(572,299.67){\rule{0.241pt}{0.400pt}}
\multiput(572.00,299.17)(0.500,1.000){2}{\rule{0.120pt}{0.400pt}}
\put(573,300.67){\rule{0.241pt}{0.400pt}}
\multiput(573.00,300.17)(0.500,1.000){2}{\rule{0.120pt}{0.400pt}}
\put(571.0,300.0){\usebox{\plotpoint}}
\put(574,302){\usebox{\plotpoint}}
\put(574,301.67){\rule{0.241pt}{0.400pt}}
\multiput(574.00,301.17)(0.500,1.000){2}{\rule{0.120pt}{0.400pt}}
\put(575,302.67){\rule{0.241pt}{0.400pt}}
\multiput(575.00,302.17)(0.500,1.000){2}{\rule{0.120pt}{0.400pt}}
\put(576,303.67){\rule{0.241pt}{0.400pt}}
\multiput(576.00,303.17)(0.500,1.000){2}{\rule{0.120pt}{0.400pt}}
\put(578,304.67){\rule{0.241pt}{0.400pt}}
\multiput(578.00,304.17)(0.500,1.000){2}{\rule{0.120pt}{0.400pt}}
\put(579,305.67){\rule{0.241pt}{0.400pt}}
\multiput(579.00,305.17)(0.500,1.000){2}{\rule{0.120pt}{0.400pt}}
\put(577.0,305.0){\usebox{\plotpoint}}
\put(580.0,307.0){\usebox{\plotpoint}}
\put(581,307.67){\rule{0.241pt}{0.400pt}}
\multiput(581.00,307.17)(0.500,1.000){2}{\rule{0.120pt}{0.400pt}}
\put(582,308.67){\rule{0.241pt}{0.400pt}}
\multiput(582.00,308.17)(0.500,1.000){2}{\rule{0.120pt}{0.400pt}}
\put(580.0,308.0){\usebox{\plotpoint}}
\put(585,309.67){\rule{0.241pt}{0.400pt}}
\multiput(585.00,309.17)(0.500,1.000){2}{\rule{0.120pt}{0.400pt}}
\put(583.0,310.0){\rule[-0.200pt]{0.482pt}{0.400pt}}
\put(586.0,311.0){\usebox{\plotpoint}}
\put(587,311.67){\rule{0.241pt}{0.400pt}}
\multiput(587.00,311.17)(0.500,1.000){2}{\rule{0.120pt}{0.400pt}}
\put(587.0,311.0){\usebox{\plotpoint}}
\put(589.67,313){\rule{0.400pt}{0.482pt}}
\multiput(589.17,313.00)(1.000,1.000){2}{\rule{0.400pt}{0.241pt}}
\put(591,314.67){\rule{0.241pt}{0.400pt}}
\multiput(591.00,314.17)(0.500,1.000){2}{\rule{0.120pt}{0.400pt}}
\put(588.0,313.0){\rule[-0.200pt]{0.482pt}{0.400pt}}
\put(592.0,316.0){\usebox{\plotpoint}}
\put(593,316.67){\rule{0.241pt}{0.400pt}}
\multiput(593.00,316.17)(0.500,1.000){2}{\rule{0.120pt}{0.400pt}}
\put(593.0,316.0){\usebox{\plotpoint}}
\put(594,318){\usebox{\plotpoint}}
\put(594,318){\usebox{\plotpoint}}
\put(594,318){\usebox{\plotpoint}}
\put(594,318){\usebox{\plotpoint}}
\put(594,318){\usebox{\plotpoint}}
\put(594,318){\usebox{\plotpoint}}
\put(594,318){\usebox{\plotpoint}}
\put(594,318){\usebox{\plotpoint}}
\put(594,318){\usebox{\plotpoint}}
\put(594,318){\usebox{\plotpoint}}
\put(594,318){\usebox{\plotpoint}}
\put(594,318){\usebox{\plotpoint}}
\put(594,318){\usebox{\plotpoint}}
\put(594,318){\usebox{\plotpoint}}
\put(594,318){\usebox{\plotpoint}}
\put(594,318){\usebox{\plotpoint}}
\put(594,317.67){\rule{0.241pt}{0.400pt}}
\multiput(594.00,317.17)(0.500,1.000){2}{\rule{0.120pt}{0.400pt}}
\put(595.67,319){\rule{0.400pt}{0.482pt}}
\multiput(595.17,319.00)(1.000,1.000){2}{\rule{0.400pt}{0.241pt}}
\put(597,320.67){\rule{0.241pt}{0.400pt}}
\multiput(597.00,320.17)(0.500,1.000){2}{\rule{0.120pt}{0.400pt}}
\put(595.0,319.0){\usebox{\plotpoint}}
\put(598.67,322){\rule{0.400pt}{0.482pt}}
\multiput(598.17,322.00)(1.000,1.000){2}{\rule{0.400pt}{0.241pt}}
\put(598.0,322.0){\usebox{\plotpoint}}
\put(600.0,324.0){\usebox{\plotpoint}}
\put(602,324.67){\rule{0.241pt}{0.400pt}}
\multiput(602.00,324.17)(0.500,1.000){2}{\rule{0.120pt}{0.400pt}}
\put(603,325.67){\rule{0.241pt}{0.400pt}}
\multiput(603.00,325.17)(0.500,1.000){2}{\rule{0.120pt}{0.400pt}}
\put(600.0,325.0){\rule[-0.200pt]{0.482pt}{0.400pt}}
\put(605,326.67){\rule{0.241pt}{0.400pt}}
\multiput(605.00,326.17)(0.500,1.000){2}{\rule{0.120pt}{0.400pt}}
\put(604.0,327.0){\usebox{\plotpoint}}
\put(606.0,328.0){\usebox{\plotpoint}}
\put(607,328.67){\rule{0.241pt}{0.400pt}}
\multiput(607.00,328.17)(0.500,1.000){2}{\rule{0.120pt}{0.400pt}}
\put(608,329.67){\rule{0.241pt}{0.400pt}}
\multiput(608.00,329.17)(0.500,1.000){2}{\rule{0.120pt}{0.400pt}}
\put(607.0,328.0){\usebox{\plotpoint}}
\put(609.67,331){\rule{0.400pt}{0.482pt}}
\multiput(609.17,331.00)(1.000,1.000){2}{\rule{0.400pt}{0.241pt}}
\put(609.0,331.0){\usebox{\plotpoint}}
\put(612,331.67){\rule{0.241pt}{0.400pt}}
\multiput(612.00,332.17)(0.500,-1.000){2}{\rule{0.120pt}{0.400pt}}
\put(611.0,333.0){\usebox{\plotpoint}}
\put(613,332.67){\rule{0.241pt}{0.400pt}}
\multiput(613.00,332.17)(0.500,1.000){2}{\rule{0.120pt}{0.400pt}}
\put(613.0,332.0){\usebox{\plotpoint}}
\put(615,333.67){\rule{0.241pt}{0.400pt}}
\multiput(615.00,333.17)(0.500,1.000){2}{\rule{0.120pt}{0.400pt}}
\put(614.0,334.0){\usebox{\plotpoint}}
\put(616,335){\usebox{\plotpoint}}
\put(616,335){\usebox{\plotpoint}}
\put(616,335){\usebox{\plotpoint}}
\put(616,335){\usebox{\plotpoint}}
\put(616,335){\usebox{\plotpoint}}
\put(616,335){\usebox{\plotpoint}}
\put(616,335){\usebox{\plotpoint}}
\put(616,335){\usebox{\plotpoint}}
\put(616,335){\usebox{\plotpoint}}
\put(616,335){\usebox{\plotpoint}}
\put(616,335){\usebox{\plotpoint}}
\put(617,334.67){\rule{0.241pt}{0.400pt}}
\multiput(617.00,334.17)(0.500,1.000){2}{\rule{0.120pt}{0.400pt}}
\put(618,335.67){\rule{0.241pt}{0.400pt}}
\multiput(618.00,335.17)(0.500,1.000){2}{\rule{0.120pt}{0.400pt}}
\put(616.0,335.0){\usebox{\plotpoint}}
\put(619.0,337.0){\usebox{\plotpoint}}
\put(620,337.67){\rule{0.241pt}{0.400pt}}
\multiput(620.00,337.17)(0.500,1.000){2}{\rule{0.120pt}{0.400pt}}
\put(619.0,338.0){\usebox{\plotpoint}}
\put(622,338.67){\rule{0.241pt}{0.400pt}}
\multiput(622.00,338.17)(0.500,1.000){2}{\rule{0.120pt}{0.400pt}}
\put(623,339.67){\rule{0.241pt}{0.400pt}}
\multiput(623.00,339.17)(0.500,1.000){2}{\rule{0.120pt}{0.400pt}}
\put(624,340.67){\rule{0.241pt}{0.400pt}}
\multiput(624.00,340.17)(0.500,1.000){2}{\rule{0.120pt}{0.400pt}}
\put(625,341.67){\rule{0.241pt}{0.400pt}}
\multiput(625.00,341.17)(0.500,1.000){2}{\rule{0.120pt}{0.400pt}}
\put(621.0,339.0){\usebox{\plotpoint}}
\put(626,343){\usebox{\plotpoint}}
\put(631,342.67){\rule{0.241pt}{0.400pt}}
\multiput(631.00,342.17)(0.500,1.000){2}{\rule{0.120pt}{0.400pt}}
\put(632,343.67){\rule{0.241pt}{0.400pt}}
\multiput(632.00,343.17)(0.500,1.000){2}{\rule{0.120pt}{0.400pt}}
\put(626.0,343.0){\rule[-0.200pt]{1.204pt}{0.400pt}}
\put(633.0,345.0){\usebox{\plotpoint}}
\put(636,345.67){\rule{0.241pt}{0.400pt}}
\multiput(636.00,345.17)(0.500,1.000){2}{\rule{0.120pt}{0.400pt}}
\put(633.0,346.0){\rule[-0.200pt]{0.723pt}{0.400pt}}
\put(638,346.67){\rule{0.241pt}{0.400pt}}
\multiput(638.00,346.17)(0.500,1.000){2}{\rule{0.120pt}{0.400pt}}
\put(637.0,347.0){\usebox{\plotpoint}}
\put(639.0,348.0){\usebox{\plotpoint}}
\put(639.0,349.0){\usebox{\plotpoint}}
\put(640.0,349.0){\usebox{\plotpoint}}
\put(642,349.67){\rule{0.241pt}{0.400pt}}
\multiput(642.00,349.17)(0.500,1.000){2}{\rule{0.120pt}{0.400pt}}
\put(640.0,350.0){\rule[-0.200pt]{0.482pt}{0.400pt}}
\put(644,350.67){\rule{0.241pt}{0.400pt}}
\multiput(644.00,350.17)(0.500,1.000){2}{\rule{0.120pt}{0.400pt}}
\put(645,351.67){\rule{0.241pt}{0.400pt}}
\multiput(645.00,351.17)(0.500,1.000){2}{\rule{0.120pt}{0.400pt}}
\put(643.0,351.0){\usebox{\plotpoint}}
\put(646,353.67){\rule{0.241pt}{0.400pt}}
\multiput(646.00,353.17)(0.500,1.000){2}{\rule{0.120pt}{0.400pt}}
\put(647,354.67){\rule{0.241pt}{0.400pt}}
\multiput(647.00,354.17)(0.500,1.000){2}{\rule{0.120pt}{0.400pt}}
\put(648,355.67){\rule{0.241pt}{0.400pt}}
\multiput(648.00,355.17)(0.500,1.000){2}{\rule{0.120pt}{0.400pt}}
\put(649,356.67){\rule{0.241pt}{0.400pt}}
\multiput(649.00,356.17)(0.500,1.000){2}{\rule{0.120pt}{0.400pt}}
\put(650,357.67){\rule{0.241pt}{0.400pt}}
\multiput(650.00,357.17)(0.500,1.000){2}{\rule{0.120pt}{0.400pt}}
\put(651,357.67){\rule{0.241pt}{0.400pt}}
\multiput(651.00,358.17)(0.500,-1.000){2}{\rule{0.120pt}{0.400pt}}
\put(646.0,353.0){\usebox{\plotpoint}}
\put(652.0,358.0){\usebox{\plotpoint}}
\put(653,358.67){\rule{0.241pt}{0.400pt}}
\multiput(653.00,358.17)(0.500,1.000){2}{\rule{0.120pt}{0.400pt}}
\put(653.0,358.0){\usebox{\plotpoint}}
\put(654,360){\usebox{\plotpoint}}
\put(654,360){\usebox{\plotpoint}}
\put(654,360){\usebox{\plotpoint}}
\put(654,360){\usebox{\plotpoint}}
\put(654,360){\usebox{\plotpoint}}
\put(654,360){\usebox{\plotpoint}}
\put(654,360){\usebox{\plotpoint}}
\put(654,360){\usebox{\plotpoint}}
\put(654,360){\usebox{\plotpoint}}
\put(655,359.67){\rule{0.241pt}{0.400pt}}
\multiput(655.00,359.17)(0.500,1.000){2}{\rule{0.120pt}{0.400pt}}
\put(656,360.67){\rule{0.241pt}{0.400pt}}
\multiput(656.00,360.17)(0.500,1.000){2}{\rule{0.120pt}{0.400pt}}
\put(657,361.67){\rule{0.241pt}{0.400pt}}
\multiput(657.00,361.17)(0.500,1.000){2}{\rule{0.120pt}{0.400pt}}
\put(654.0,360.0){\usebox{\plotpoint}}
\put(658.0,363.0){\usebox{\plotpoint}}
\put(658.67,362){\rule{0.400pt}{0.482pt}}
\multiput(658.17,362.00)(1.000,1.000){2}{\rule{0.400pt}{0.241pt}}
\put(659.0,362.0){\usebox{\plotpoint}}
\put(661,363.67){\rule{0.241pt}{0.400pt}}
\multiput(661.00,363.17)(0.500,1.000){2}{\rule{0.120pt}{0.400pt}}
\put(662,364.67){\rule{0.241pt}{0.400pt}}
\multiput(662.00,364.17)(0.500,1.000){2}{\rule{0.120pt}{0.400pt}}
\put(660.0,364.0){\usebox{\plotpoint}}
\put(667,365.67){\rule{0.241pt}{0.400pt}}
\multiput(667.00,365.17)(0.500,1.000){2}{\rule{0.120pt}{0.400pt}}
\put(663.0,366.0){\rule[-0.200pt]{0.964pt}{0.400pt}}
\put(671,365.67){\rule{0.241pt}{0.400pt}}
\multiput(671.00,366.17)(0.500,-1.000){2}{\rule{0.120pt}{0.400pt}}
\put(668.0,367.0){\rule[-0.200pt]{0.723pt}{0.400pt}}
\put(674,365.67){\rule{0.241pt}{0.400pt}}
\multiput(674.00,365.17)(0.500,1.000){2}{\rule{0.120pt}{0.400pt}}
\put(672.0,366.0){\rule[-0.200pt]{0.482pt}{0.400pt}}
\put(677,366.67){\rule{0.241pt}{0.400pt}}
\multiput(677.00,366.17)(0.500,1.000){2}{\rule{0.120pt}{0.400pt}}
\put(675.0,367.0){\rule[-0.200pt]{0.482pt}{0.400pt}}
\put(679,367.67){\rule{0.241pt}{0.400pt}}
\multiput(679.00,367.17)(0.500,1.000){2}{\rule{0.120pt}{0.400pt}}
\put(678.0,368.0){\usebox{\plotpoint}}
\put(686,368.67){\rule{0.241pt}{0.400pt}}
\multiput(686.00,368.17)(0.500,1.000){2}{\rule{0.120pt}{0.400pt}}
\put(680.0,369.0){\rule[-0.200pt]{1.445pt}{0.400pt}}
\put(691,369.67){\rule{0.241pt}{0.400pt}}
\multiput(691.00,369.17)(0.500,1.000){2}{\rule{0.120pt}{0.400pt}}
\put(692,369.67){\rule{0.241pt}{0.400pt}}
\multiput(692.00,370.17)(0.500,-1.000){2}{\rule{0.120pt}{0.400pt}}
\put(687.0,370.0){\rule[-0.200pt]{0.964pt}{0.400pt}}
\put(693,370){\usebox{\plotpoint}}
\put(693.0,370.0){\rule[-0.200pt]{0.482pt}{0.400pt}}
\put(695.0,370.0){\usebox{\plotpoint}}
\put(695.0,371.0){\usebox{\plotpoint}}
\put(696.0,371.0){\usebox{\plotpoint}}
\put(696.0,371.0){\usebox{\plotpoint}}
\put(697,370.67){\rule{0.241pt}{0.400pt}}
\multiput(697.00,370.17)(0.500,1.000){2}{\rule{0.120pt}{0.400pt}}
\put(696.0,371.0){\usebox{\plotpoint}}
\put(700,370.67){\rule{0.241pt}{0.400pt}}
\multiput(700.00,371.17)(0.500,-1.000){2}{\rule{0.120pt}{0.400pt}}
\put(698.0,372.0){\rule[-0.200pt]{0.482pt}{0.400pt}}
\put(703,369.67){\rule{0.241pt}{0.400pt}}
\multiput(703.00,370.17)(0.500,-1.000){2}{\rule{0.120pt}{0.400pt}}
\put(704,369.67){\rule{0.241pt}{0.400pt}}
\multiput(704.00,369.17)(0.500,1.000){2}{\rule{0.120pt}{0.400pt}}
\put(701.0,371.0){\rule[-0.200pt]{0.482pt}{0.400pt}}
\put(710,369.67){\rule{0.241pt}{0.400pt}}
\multiput(710.00,370.17)(0.500,-1.000){2}{\rule{0.120pt}{0.400pt}}
\put(705.0,371.0){\rule[-0.200pt]{1.204pt}{0.400pt}}
\put(711.0,370.0){\usebox{\plotpoint}}
\put(712.0,369.0){\usebox{\plotpoint}}
\put(715,367.67){\rule{0.241pt}{0.400pt}}
\multiput(715.00,368.17)(0.500,-1.000){2}{\rule{0.120pt}{0.400pt}}
\put(716,367.67){\rule{0.241pt}{0.400pt}}
\multiput(716.00,367.17)(0.500,1.000){2}{\rule{0.120pt}{0.400pt}}
\put(712.0,369.0){\rule[-0.200pt]{0.723pt}{0.400pt}}
\put(717.0,369.0){\rule[-0.200pt]{0.482pt}{0.400pt}}
\put(719.0,368.0){\usebox{\plotpoint}}
\put(722,366.67){\rule{0.241pt}{0.400pt}}
\multiput(722.00,367.17)(0.500,-1.000){2}{\rule{0.120pt}{0.400pt}}
\put(719.0,368.0){\rule[-0.200pt]{0.723pt}{0.400pt}}
\put(726,365.67){\rule{0.241pt}{0.400pt}}
\multiput(726.00,366.17)(0.500,-1.000){2}{\rule{0.120pt}{0.400pt}}
\put(723.0,367.0){\rule[-0.200pt]{0.723pt}{0.400pt}}
\put(730,364.67){\rule{0.241pt}{0.400pt}}
\multiput(730.00,365.17)(0.500,-1.000){2}{\rule{0.120pt}{0.400pt}}
\put(727.0,366.0){\rule[-0.200pt]{0.723pt}{0.400pt}}
\put(731.0,365.0){\usebox{\plotpoint}}
\put(732.0,364.0){\usebox{\plotpoint}}
\put(734,362.67){\rule{0.241pt}{0.400pt}}
\multiput(734.00,363.17)(0.500,-1.000){2}{\rule{0.120pt}{0.400pt}}
\put(732.0,364.0){\rule[-0.200pt]{0.482pt}{0.400pt}}
\put(736,361.67){\rule{0.241pt}{0.400pt}}
\multiput(736.00,362.17)(0.500,-1.000){2}{\rule{0.120pt}{0.400pt}}
\put(735.0,363.0){\usebox{\plotpoint}}
\put(738,360.67){\rule{0.241pt}{0.400pt}}
\multiput(738.00,361.17)(0.500,-1.000){2}{\rule{0.120pt}{0.400pt}}
\put(737.0,362.0){\usebox{\plotpoint}}
\put(739,361){\usebox{\plotpoint}}
\put(740,359.67){\rule{0.241pt}{0.400pt}}
\multiput(740.00,360.17)(0.500,-1.000){2}{\rule{0.120pt}{0.400pt}}
\put(739.0,361.0){\usebox{\plotpoint}}
\put(741,360){\usebox{\plotpoint}}
\put(741,360){\usebox{\plotpoint}}
\put(741,360){\usebox{\plotpoint}}
\put(741,360){\usebox{\plotpoint}}
\put(741,360){\usebox{\plotpoint}}
\put(741,360){\usebox{\plotpoint}}
\put(741,360){\usebox{\plotpoint}}
\put(741,360){\usebox{\plotpoint}}
\put(741,360){\usebox{\plotpoint}}
\put(743,358.67){\rule{0.241pt}{0.400pt}}
\multiput(743.00,359.17)(0.500,-1.000){2}{\rule{0.120pt}{0.400pt}}
\put(741.0,360.0){\rule[-0.200pt]{0.482pt}{0.400pt}}
\put(744.0,359.0){\usebox{\plotpoint}}
\put(745.0,358.0){\usebox{\plotpoint}}
\put(748,356.67){\rule{0.241pt}{0.400pt}}
\multiput(748.00,357.17)(0.500,-1.000){2}{\rule{0.120pt}{0.400pt}}
\put(745.0,358.0){\rule[-0.200pt]{0.723pt}{0.400pt}}
\put(749.0,357.0){\rule[-0.200pt]{0.723pt}{0.400pt}}
\put(752,354.67){\rule{0.241pt}{0.400pt}}
\multiput(752.00,355.17)(0.500,-1.000){2}{\rule{0.120pt}{0.400pt}}
\put(752.0,356.0){\usebox{\plotpoint}}
\put(754,353.67){\rule{0.241pt}{0.400pt}}
\multiput(754.00,354.17)(0.500,-1.000){2}{\rule{0.120pt}{0.400pt}}
\put(755,353.67){\rule{0.241pt}{0.400pt}}
\multiput(755.00,353.17)(0.500,1.000){2}{\rule{0.120pt}{0.400pt}}
\put(756,353.67){\rule{0.241pt}{0.400pt}}
\multiput(756.00,354.17)(0.500,-1.000){2}{\rule{0.120pt}{0.400pt}}
\put(757,352.67){\rule{0.241pt}{0.400pt}}
\multiput(757.00,353.17)(0.500,-1.000){2}{\rule{0.120pt}{0.400pt}}
\put(758,351.67){\rule{0.241pt}{0.400pt}}
\multiput(758.00,352.17)(0.500,-1.000){2}{\rule{0.120pt}{0.400pt}}
\put(753.0,355.0){\usebox{\plotpoint}}
\put(759,352){\usebox{\plotpoint}}
\put(760,350.67){\rule{0.241pt}{0.400pt}}
\multiput(760.00,351.17)(0.500,-1.000){2}{\rule{0.120pt}{0.400pt}}
\put(759.0,352.0){\usebox{\plotpoint}}
\put(764,350.67){\rule{0.241pt}{0.400pt}}
\multiput(764.00,350.17)(0.500,1.000){2}{\rule{0.120pt}{0.400pt}}
\put(761.0,351.0){\rule[-0.200pt]{0.723pt}{0.400pt}}
\put(765.0,351.0){\usebox{\plotpoint}}
\put(767,349.67){\rule{0.241pt}{0.400pt}}
\multiput(767.00,350.17)(0.500,-1.000){2}{\rule{0.120pt}{0.400pt}}
\put(765.0,351.0){\rule[-0.200pt]{0.482pt}{0.400pt}}
\put(769,348.67){\rule{0.241pt}{0.400pt}}
\multiput(769.00,349.17)(0.500,-1.000){2}{\rule{0.120pt}{0.400pt}}
\put(768.0,350.0){\usebox{\plotpoint}}
\put(771,347.67){\rule{0.241pt}{0.400pt}}
\multiput(771.00,348.17)(0.500,-1.000){2}{\rule{0.120pt}{0.400pt}}
\put(770.0,349.0){\usebox{\plotpoint}}
\put(772,348){\usebox{\plotpoint}}
\put(772,346.67){\rule{0.241pt}{0.400pt}}
\multiput(772.00,347.17)(0.500,-1.000){2}{\rule{0.120pt}{0.400pt}}
\put(773,345.67){\rule{0.241pt}{0.400pt}}
\multiput(773.00,346.17)(0.500,-1.000){2}{\rule{0.120pt}{0.400pt}}
\put(774,344.67){\rule{0.241pt}{0.400pt}}
\multiput(774.00,345.17)(0.500,-1.000){2}{\rule{0.120pt}{0.400pt}}
\put(774.67,343){\rule{0.400pt}{0.482pt}}
\multiput(774.17,344.00)(1.000,-1.000){2}{\rule{0.400pt}{0.241pt}}
\put(779,341.67){\rule{0.241pt}{0.400pt}}
\multiput(779.00,342.17)(0.500,-1.000){2}{\rule{0.120pt}{0.400pt}}
\put(776.0,343.0){\rule[-0.200pt]{0.723pt}{0.400pt}}
\put(781,340.67){\rule{0.241pt}{0.400pt}}
\multiput(781.00,341.17)(0.500,-1.000){2}{\rule{0.120pt}{0.400pt}}
\put(782,340.67){\rule{0.241pt}{0.400pt}}
\multiput(782.00,340.17)(0.500,1.000){2}{\rule{0.120pt}{0.400pt}}
\put(783,340.67){\rule{0.241pt}{0.400pt}}
\multiput(783.00,341.17)(0.500,-1.000){2}{\rule{0.120pt}{0.400pt}}
\put(784,339.67){\rule{0.241pt}{0.400pt}}
\multiput(784.00,340.17)(0.500,-1.000){2}{\rule{0.120pt}{0.400pt}}
\put(780.0,342.0){\usebox{\plotpoint}}
\put(785.0,339.0){\usebox{\plotpoint}}
\put(786,337.67){\rule{0.241pt}{0.400pt}}
\multiput(786.00,338.17)(0.500,-1.000){2}{\rule{0.120pt}{0.400pt}}
\put(787,336.67){\rule{0.241pt}{0.400pt}}
\multiput(787.00,337.17)(0.500,-1.000){2}{\rule{0.120pt}{0.400pt}}
\put(788,335.67){\rule{0.241pt}{0.400pt}}
\multiput(788.00,336.17)(0.500,-1.000){2}{\rule{0.120pt}{0.400pt}}
\put(785.0,339.0){\usebox{\plotpoint}}
\put(789,336){\usebox{\plotpoint}}
\put(789,336){\usebox{\plotpoint}}
\put(789,336){\usebox{\plotpoint}}
\put(789,336){\usebox{\plotpoint}}
\put(789,336){\usebox{\plotpoint}}
\put(789,336){\usebox{\plotpoint}}
\put(789.0,336.0){\usebox{\plotpoint}}
\put(790,334.67){\rule{0.241pt}{0.400pt}}
\multiput(790.00,334.17)(0.500,1.000){2}{\rule{0.120pt}{0.400pt}}
\put(790.67,334){\rule{0.400pt}{0.482pt}}
\multiput(790.17,335.00)(1.000,-1.000){2}{\rule{0.400pt}{0.241pt}}
\put(790.0,335.0){\usebox{\plotpoint}}
\put(792,334){\usebox{\plotpoint}}
\put(791.67,332){\rule{0.400pt}{0.482pt}}
\multiput(791.17,333.00)(1.000,-1.000){2}{\rule{0.400pt}{0.241pt}}
\put(793,332){\usebox{\plotpoint}}
\put(793,332){\usebox{\plotpoint}}
\put(793,332){\usebox{\plotpoint}}
\put(793,332){\usebox{\plotpoint}}
\put(793,332){\usebox{\plotpoint}}
\put(793,332){\usebox{\plotpoint}}
\put(793,332){\usebox{\plotpoint}}
\put(793,332){\usebox{\plotpoint}}
\put(793,332){\usebox{\plotpoint}}
\put(794,330.67){\rule{0.241pt}{0.400pt}}
\multiput(794.00,331.17)(0.500,-1.000){2}{\rule{0.120pt}{0.400pt}}
\put(793.0,332.0){\usebox{\plotpoint}}
\put(795.0,331.0){\rule[-0.200pt]{0.723pt}{0.400pt}}
\put(798,328.67){\rule{0.241pt}{0.400pt}}
\multiput(798.00,329.17)(0.500,-1.000){2}{\rule{0.120pt}{0.400pt}}
\put(798.0,330.0){\usebox{\plotpoint}}
\put(799,329){\usebox{\plotpoint}}
\put(799,329){\usebox{\plotpoint}}
\put(799,329){\usebox{\plotpoint}}
\put(799,329){\usebox{\plotpoint}}
\put(799,329){\usebox{\plotpoint}}
\put(799,329){\usebox{\plotpoint}}
\put(799,329){\usebox{\plotpoint}}
\put(799,329){\usebox{\plotpoint}}
\put(799,329){\usebox{\plotpoint}}
\put(799,329){\usebox{\plotpoint}}
\put(799,329){\usebox{\plotpoint}}
\put(799,329){\usebox{\plotpoint}}
\put(799,329){\usebox{\plotpoint}}
\put(799,329){\usebox{\plotpoint}}
\put(799,329){\usebox{\plotpoint}}
\put(799,329){\usebox{\plotpoint}}
\put(799,329){\usebox{\plotpoint}}
\put(799,329){\usebox{\plotpoint}}
\put(798.67,327){\rule{0.400pt}{0.482pt}}
\multiput(798.17,328.00)(1.000,-1.000){2}{\rule{0.400pt}{0.241pt}}
\put(800,325.67){\rule{0.241pt}{0.400pt}}
\multiput(800.00,326.17)(0.500,-1.000){2}{\rule{0.120pt}{0.400pt}}
\put(801,324.67){\rule{0.241pt}{0.400pt}}
\multiput(801.00,325.17)(0.500,-1.000){2}{\rule{0.120pt}{0.400pt}}
\put(802,323.67){\rule{0.241pt}{0.400pt}}
\multiput(802.00,324.17)(0.500,-1.000){2}{\rule{0.120pt}{0.400pt}}
\put(804,322.67){\rule{0.241pt}{0.400pt}}
\multiput(804.00,323.17)(0.500,-1.000){2}{\rule{0.120pt}{0.400pt}}
\put(803.0,324.0){\usebox{\plotpoint}}
\put(805,320.67){\rule{0.241pt}{0.400pt}}
\multiput(805.00,321.17)(0.500,-1.000){2}{\rule{0.120pt}{0.400pt}}
\put(805.0,322.0){\usebox{\plotpoint}}
\put(808,319.67){\rule{0.241pt}{0.400pt}}
\multiput(808.00,320.17)(0.500,-1.000){2}{\rule{0.120pt}{0.400pt}}
\put(806.0,321.0){\rule[-0.200pt]{0.482pt}{0.400pt}}
\put(812,318.67){\rule{0.241pt}{0.400pt}}
\multiput(812.00,319.17)(0.500,-1.000){2}{\rule{0.120pt}{0.400pt}}
\put(813,317.67){\rule{0.241pt}{0.400pt}}
\multiput(813.00,318.17)(0.500,-1.000){2}{\rule{0.120pt}{0.400pt}}
\put(809.0,320.0){\rule[-0.200pt]{0.723pt}{0.400pt}}
\put(814,318){\usebox{\plotpoint}}
\put(814,316.67){\rule{0.241pt}{0.400pt}}
\multiput(814.00,317.17)(0.500,-1.000){2}{\rule{0.120pt}{0.400pt}}
\put(816.67,315){\rule{0.400pt}{0.482pt}}
\multiput(816.17,316.00)(1.000,-1.000){2}{\rule{0.400pt}{0.241pt}}
\put(815.0,317.0){\rule[-0.200pt]{0.482pt}{0.400pt}}
\put(818,315){\usebox{\plotpoint}}
\put(818,315){\usebox{\plotpoint}}
\put(818,315){\usebox{\plotpoint}}
\put(818,315){\usebox{\plotpoint}}
\put(818,315){\usebox{\plotpoint}}
\put(818,315){\usebox{\plotpoint}}
\put(818,315){\usebox{\plotpoint}}
\put(818,315){\usebox{\plotpoint}}
\put(818,315){\usebox{\plotpoint}}
\put(818,315){\usebox{\plotpoint}}
\put(818,315){\usebox{\plotpoint}}
\put(818,315){\usebox{\plotpoint}}
\put(818,315){\usebox{\plotpoint}}
\put(818,315){\usebox{\plotpoint}}
\put(818,315){\usebox{\plotpoint}}
\put(818,315){\usebox{\plotpoint}}
\put(818,312.67){\rule{0.241pt}{0.400pt}}
\multiput(818.00,313.17)(0.500,-1.000){2}{\rule{0.120pt}{0.400pt}}
\put(819,311.67){\rule{0.241pt}{0.400pt}}
\multiput(819.00,312.17)(0.500,-1.000){2}{\rule{0.120pt}{0.400pt}}
\put(818.0,314.0){\usebox{\plotpoint}}
\put(820.67,310){\rule{0.400pt}{0.482pt}}
\multiput(820.17,311.00)(1.000,-1.000){2}{\rule{0.400pt}{0.241pt}}
\put(820.0,312.0){\usebox{\plotpoint}}
\put(824,308.67){\rule{0.241pt}{0.400pt}}
\multiput(824.00,309.17)(0.500,-1.000){2}{\rule{0.120pt}{0.400pt}}
\put(822.0,310.0){\rule[-0.200pt]{0.482pt}{0.400pt}}
\put(825,306.67){\rule{0.241pt}{0.400pt}}
\multiput(825.00,307.17)(0.500,-1.000){2}{\rule{0.120pt}{0.400pt}}
\put(825.0,308.0){\usebox{\plotpoint}}
\put(827,305.67){\rule{0.241pt}{0.400pt}}
\multiput(827.00,306.17)(0.500,-1.000){2}{\rule{0.120pt}{0.400pt}}
\put(826.0,307.0){\usebox{\plotpoint}}
\put(828,306){\usebox{\plotpoint}}
\put(828,306){\usebox{\plotpoint}}
\put(828,306){\usebox{\plotpoint}}
\put(828,306){\usebox{\plotpoint}}
\put(828,306){\usebox{\plotpoint}}
\put(828,306){\usebox{\plotpoint}}
\put(828,306){\usebox{\plotpoint}}
\put(828,306){\usebox{\plotpoint}}
\put(828,306){\usebox{\plotpoint}}
\put(828,304.67){\rule{0.241pt}{0.400pt}}
\multiput(828.00,305.17)(0.500,-1.000){2}{\rule{0.120pt}{0.400pt}}
\put(829,303.67){\rule{0.241pt}{0.400pt}}
\multiput(829.00,304.17)(0.500,-1.000){2}{\rule{0.120pt}{0.400pt}}
\put(830,302.67){\rule{0.241pt}{0.400pt}}
\multiput(830.00,303.17)(0.500,-1.000){2}{\rule{0.120pt}{0.400pt}}
\put(831.0,302.0){\usebox{\plotpoint}}
\put(832,300.67){\rule{0.241pt}{0.400pt}}
\multiput(832.00,301.17)(0.500,-1.000){2}{\rule{0.120pt}{0.400pt}}
\put(831.0,302.0){\usebox{\plotpoint}}
\put(834,299.67){\rule{0.241pt}{0.400pt}}
\multiput(834.00,300.17)(0.500,-1.000){2}{\rule{0.120pt}{0.400pt}}
\put(833.0,301.0){\usebox{\plotpoint}}
\put(835.67,298){\rule{0.400pt}{0.482pt}}
\multiput(835.17,299.00)(1.000,-1.000){2}{\rule{0.400pt}{0.241pt}}
\put(837,296.67){\rule{0.241pt}{0.400pt}}
\multiput(837.00,297.17)(0.500,-1.000){2}{\rule{0.120pt}{0.400pt}}
\put(835.0,300.0){\usebox{\plotpoint}}
\put(838,297){\usebox{\plotpoint}}
\put(839,295.67){\rule{0.241pt}{0.400pt}}
\multiput(839.00,296.17)(0.500,-1.000){2}{\rule{0.120pt}{0.400pt}}
\put(840,294.67){\rule{0.241pt}{0.400pt}}
\multiput(840.00,295.17)(0.500,-1.000){2}{\rule{0.120pt}{0.400pt}}
\put(838.0,297.0){\usebox{\plotpoint}}
\put(842,293.67){\rule{0.241pt}{0.400pt}}
\multiput(842.00,294.17)(0.500,-1.000){2}{\rule{0.120pt}{0.400pt}}
\put(843,292.67){\rule{0.241pt}{0.400pt}}
\multiput(843.00,293.17)(0.500,-1.000){2}{\rule{0.120pt}{0.400pt}}
\put(841.0,295.0){\usebox{\plotpoint}}
\put(844.0,293.0){\usebox{\plotpoint}}
\put(845.0,292.0){\usebox{\plotpoint}}
\put(846,290.67){\rule{0.241pt}{0.400pt}}
\multiput(846.00,291.17)(0.500,-1.000){2}{\rule{0.120pt}{0.400pt}}
\put(845.0,292.0){\usebox{\plotpoint}}
\put(848,289.67){\rule{0.241pt}{0.400pt}}
\multiput(848.00,290.17)(0.500,-1.000){2}{\rule{0.120pt}{0.400pt}}
\put(847.0,291.0){\usebox{\plotpoint}}
\put(849.0,290.0){\rule[-0.200pt]{0.482pt}{0.400pt}}
\put(851.0,289.0){\usebox{\plotpoint}}
\put(852,287.67){\rule{0.241pt}{0.400pt}}
\multiput(852.00,288.17)(0.500,-1.000){2}{\rule{0.120pt}{0.400pt}}
\put(851.0,289.0){\usebox{\plotpoint}}
\put(855,286.67){\rule{0.241pt}{0.400pt}}
\multiput(855.00,287.17)(0.500,-1.000){2}{\rule{0.120pt}{0.400pt}}
\put(856,285.67){\rule{0.241pt}{0.400pt}}
\multiput(856.00,286.17)(0.500,-1.000){2}{\rule{0.120pt}{0.400pt}}
\put(853.0,288.0){\rule[-0.200pt]{0.482pt}{0.400pt}}
\put(857.0,286.0){\usebox{\plotpoint}}
\put(858,283.67){\rule{0.241pt}{0.400pt}}
\multiput(858.00,284.17)(0.500,-1.000){2}{\rule{0.120pt}{0.400pt}}
\put(858.0,285.0){\usebox{\plotpoint}}
\put(859.0,284.0){\rule[-0.200pt]{0.482pt}{0.400pt}}
\put(861.0,283.0){\usebox{\plotpoint}}
\put(863,281.67){\rule{0.241pt}{0.400pt}}
\multiput(863.00,282.17)(0.500,-1.000){2}{\rule{0.120pt}{0.400pt}}
\put(861.0,283.0){\rule[-0.200pt]{0.482pt}{0.400pt}}
\put(864,279.67){\rule{0.241pt}{0.400pt}}
\multiput(864.00,280.17)(0.500,-1.000){2}{\rule{0.120pt}{0.400pt}}
\put(864.0,281.0){\usebox{\plotpoint}}
\put(866,278.67){\rule{0.241pt}{0.400pt}}
\multiput(866.00,279.17)(0.500,-1.000){2}{\rule{0.120pt}{0.400pt}}
\put(865.0,280.0){\usebox{\plotpoint}}
\put(868,277.67){\rule{0.241pt}{0.400pt}}
\multiput(868.00,278.17)(0.500,-1.000){2}{\rule{0.120pt}{0.400pt}}
\put(869,276.67){\rule{0.241pt}{0.400pt}}
\multiput(869.00,277.17)(0.500,-1.000){2}{\rule{0.120pt}{0.400pt}}
\put(867.0,279.0){\usebox{\plotpoint}}
\put(870.0,277.0){\usebox{\plotpoint}}
\put(871.0,276.0){\usebox{\plotpoint}}
\put(873,274.67){\rule{0.241pt}{0.400pt}}
\multiput(873.00,275.17)(0.500,-1.000){2}{\rule{0.120pt}{0.400pt}}
\put(871.0,276.0){\rule[-0.200pt]{0.482pt}{0.400pt}}
\put(874,275){\usebox{\plotpoint}}
\put(874,275){\usebox{\plotpoint}}
\put(874,275){\usebox{\plotpoint}}
\put(874,275){\usebox{\plotpoint}}
\put(874,275){\usebox{\plotpoint}}
\put(874,275){\usebox{\plotpoint}}
\put(874,275){\usebox{\plotpoint}}
\put(874,275){\usebox{\plotpoint}}
\put(874,275){\usebox{\plotpoint}}
\put(874,275){\usebox{\plotpoint}}
\put(874,275){\usebox{\plotpoint}}
\put(874,275){\usebox{\plotpoint}}
\put(874,275){\usebox{\plotpoint}}
\put(876,273.67){\rule{0.241pt}{0.400pt}}
\multiput(876.00,274.17)(0.500,-1.000){2}{\rule{0.120pt}{0.400pt}}
\put(874.0,275.0){\rule[-0.200pt]{0.482pt}{0.400pt}}
\put(880,272.67){\rule{0.241pt}{0.400pt}}
\multiput(880.00,273.17)(0.500,-1.000){2}{\rule{0.120pt}{0.400pt}}
\put(877.0,274.0){\rule[-0.200pt]{0.723pt}{0.400pt}}
\put(881.0,273.0){\rule[-0.200pt]{0.723pt}{0.400pt}}
\put(884.0,272.0){\usebox{\plotpoint}}
\put(885,270.67){\rule{0.241pt}{0.400pt}}
\multiput(885.00,271.17)(0.500,-1.000){2}{\rule{0.120pt}{0.400pt}}
\put(886,269.67){\rule{0.241pt}{0.400pt}}
\multiput(886.00,270.17)(0.500,-1.000){2}{\rule{0.120pt}{0.400pt}}
\put(887,268.67){\rule{0.241pt}{0.400pt}}
\multiput(887.00,269.17)(0.500,-1.000){2}{\rule{0.120pt}{0.400pt}}
\put(884.0,272.0){\usebox{\plotpoint}}
\put(889,267.67){\rule{0.241pt}{0.400pt}}
\multiput(889.00,268.17)(0.500,-1.000){2}{\rule{0.120pt}{0.400pt}}
\put(888.0,269.0){\usebox{\plotpoint}}
\put(890.0,268.0){\usebox{\plotpoint}}
\put(891.0,267.0){\usebox{\plotpoint}}
\put(892,265.67){\rule{0.241pt}{0.400pt}}
\multiput(892.00,266.17)(0.500,-1.000){2}{\rule{0.120pt}{0.400pt}}
\put(891.0,267.0){\usebox{\plotpoint}}
\put(896,264.67){\rule{0.241pt}{0.400pt}}
\multiput(896.00,265.17)(0.500,-1.000){2}{\rule{0.120pt}{0.400pt}}
\put(893.0,266.0){\rule[-0.200pt]{0.723pt}{0.400pt}}
\put(897,265){\usebox{\plotpoint}}
\put(897,265){\usebox{\plotpoint}}
\put(897,265){\usebox{\plotpoint}}
\put(897,265){\usebox{\plotpoint}}
\put(897,265){\usebox{\plotpoint}}
\put(901,263.67){\rule{0.241pt}{0.400pt}}
\multiput(901.00,264.17)(0.500,-1.000){2}{\rule{0.120pt}{0.400pt}}
\put(897.0,265.0){\rule[-0.200pt]{0.964pt}{0.400pt}}
\put(903,262.67){\rule{0.241pt}{0.400pt}}
\multiput(903.00,263.17)(0.500,-1.000){2}{\rule{0.120pt}{0.400pt}}
\put(902.0,264.0){\usebox{\plotpoint}}
\put(904.0,262.0){\usebox{\plotpoint}}
\put(905,260.67){\rule{0.241pt}{0.400pt}}
\multiput(905.00,261.17)(0.500,-1.000){2}{\rule{0.120pt}{0.400pt}}
\put(906,260.67){\rule{0.241pt}{0.400pt}}
\multiput(906.00,260.17)(0.500,1.000){2}{\rule{0.120pt}{0.400pt}}
\put(904.0,262.0){\usebox{\plotpoint}}
\put(907,262){\usebox{\plotpoint}}
\put(907,262){\usebox{\plotpoint}}
\put(907,262){\usebox{\plotpoint}}
\put(907,262){\usebox{\plotpoint}}
\put(907,262){\usebox{\plotpoint}}
\put(907,262){\usebox{\plotpoint}}
\put(907,262){\usebox{\plotpoint}}
\put(907,262){\usebox{\plotpoint}}
\put(907,262){\usebox{\plotpoint}}
\put(907,262){\usebox{\plotpoint}}
\put(907.0,262.0){\rule[-0.200pt]{0.482pt}{0.400pt}}
\put(909.0,261.0){\usebox{\plotpoint}}
\put(910,259.67){\rule{0.241pt}{0.400pt}}
\multiput(910.00,260.17)(0.500,-1.000){2}{\rule{0.120pt}{0.400pt}}
\put(909.0,261.0){\usebox{\plotpoint}}
\put(911.0,260.0){\usebox{\plotpoint}}
\put(919,259.67){\rule{0.241pt}{0.400pt}}
\multiput(919.00,260.17)(0.500,-1.000){2}{\rule{0.120pt}{0.400pt}}
\put(911.0,261.0){\rule[-0.200pt]{1.927pt}{0.400pt}}
\put(922,258.67){\rule{0.241pt}{0.400pt}}
\multiput(922.00,259.17)(0.500,-1.000){2}{\rule{0.120pt}{0.400pt}}
\put(923,257.67){\rule{0.241pt}{0.400pt}}
\multiput(923.00,258.17)(0.500,-1.000){2}{\rule{0.120pt}{0.400pt}}
\put(920.0,260.0){\rule[-0.200pt]{0.482pt}{0.400pt}}
\put(924,258){\usebox{\plotpoint}}
\put(926,256.67){\rule{0.241pt}{0.400pt}}
\multiput(926.00,257.17)(0.500,-1.000){2}{\rule{0.120pt}{0.400pt}}
\put(927,256.67){\rule{0.241pt}{0.400pt}}
\multiput(927.00,256.17)(0.500,1.000){2}{\rule{0.120pt}{0.400pt}}
\put(928,256.67){\rule{0.241pt}{0.400pt}}
\multiput(928.00,257.17)(0.500,-1.000){2}{\rule{0.120pt}{0.400pt}}
\put(924.0,258.0){\rule[-0.200pt]{0.482pt}{0.400pt}}
\put(930,256.67){\rule{0.241pt}{0.400pt}}
\multiput(930.00,256.17)(0.500,1.000){2}{\rule{0.120pt}{0.400pt}}
\put(929.0,257.0){\usebox{\plotpoint}}
\put(934,257.67){\rule{0.241pt}{0.400pt}}
\multiput(934.00,257.17)(0.500,1.000){2}{\rule{0.120pt}{0.400pt}}
\put(931.0,258.0){\rule[-0.200pt]{0.723pt}{0.400pt}}
\put(935,259){\usebox{\plotpoint}}
\put(935.0,258.0){\usebox{\plotpoint}}
\put(937,257.67){\rule{0.241pt}{0.400pt}}
\multiput(937.00,257.17)(0.500,1.000){2}{\rule{0.120pt}{0.400pt}}
\put(938,257.67){\rule{0.241pt}{0.400pt}}
\multiput(938.00,258.17)(0.500,-1.000){2}{\rule{0.120pt}{0.400pt}}
\put(939,256.67){\rule{0.241pt}{0.400pt}}
\multiput(939.00,257.17)(0.500,-1.000){2}{\rule{0.120pt}{0.400pt}}
\put(940,256.67){\rule{0.241pt}{0.400pt}}
\multiput(940.00,256.17)(0.500,1.000){2}{\rule{0.120pt}{0.400pt}}
\put(941,256.67){\rule{0.241pt}{0.400pt}}
\multiput(941.00,257.17)(0.500,-1.000){2}{\rule{0.120pt}{0.400pt}}
\put(935.0,258.0){\rule[-0.200pt]{0.482pt}{0.400pt}}
\put(945,256.67){\rule{0.241pt}{0.400pt}}
\multiput(945.00,256.17)(0.500,1.000){2}{\rule{0.120pt}{0.400pt}}
\put(942.0,257.0){\rule[-0.200pt]{0.723pt}{0.400pt}}
\put(950,257.67){\rule{0.241pt}{0.400pt}}
\multiput(950.00,257.17)(0.500,1.000){2}{\rule{0.120pt}{0.400pt}}
\put(946.0,258.0){\rule[-0.200pt]{0.964pt}{0.400pt}}
\put(953,257.67){\rule{0.241pt}{0.400pt}}
\multiput(953.00,258.17)(0.500,-1.000){2}{\rule{0.120pt}{0.400pt}}
\put(951.0,259.0){\rule[-0.200pt]{0.482pt}{0.400pt}}
\put(955,256.67){\rule{0.241pt}{0.400pt}}
\multiput(955.00,257.17)(0.500,-1.000){2}{\rule{0.120pt}{0.400pt}}
\put(954.0,258.0){\usebox{\plotpoint}}
\put(962,255.67){\rule{0.241pt}{0.400pt}}
\multiput(962.00,256.17)(0.500,-1.000){2}{\rule{0.120pt}{0.400pt}}
\put(956.0,257.0){\rule[-0.200pt]{1.445pt}{0.400pt}}
\put(963,256){\usebox{\plotpoint}}
\put(963,255.67){\rule{0.241pt}{0.400pt}}
\multiput(963.00,255.17)(0.500,1.000){2}{\rule{0.120pt}{0.400pt}}
\put(964.0,257.0){\rule[-0.200pt]{0.723pt}{0.400pt}}
\put(967.0,257.0){\usebox{\plotpoint}}
\put(967.0,257.0){\usebox{\plotpoint}}
\put(967.0,257.0){\usebox{\plotpoint}}
\put(968,256.67){\rule{0.241pt}{0.400pt}}
\multiput(968.00,257.17)(0.500,-1.000){2}{\rule{0.120pt}{0.400pt}}
\put(968.0,257.0){\usebox{\plotpoint}}
\put(969.0,257.0){\usebox{\plotpoint}}
\put(970.0,257.0){\usebox{\plotpoint}}
\put(972,256.67){\rule{0.241pt}{0.400pt}}
\multiput(972.00,257.17)(0.500,-1.000){2}{\rule{0.120pt}{0.400pt}}
\put(970.0,258.0){\rule[-0.200pt]{0.482pt}{0.400pt}}
\put(974,256.67){\rule{0.241pt}{0.400pt}}
\multiput(974.00,256.17)(0.500,1.000){2}{\rule{0.120pt}{0.400pt}}
\put(973.0,257.0){\usebox{\plotpoint}}
\put(976,256.67){\rule{0.241pt}{0.400pt}}
\multiput(976.00,257.17)(0.500,-1.000){2}{\rule{0.120pt}{0.400pt}}
\put(975.0,258.0){\usebox{\plotpoint}}
\put(977,257){\usebox{\plotpoint}}
\put(978,256.67){\rule{0.241pt}{0.400pt}}
\multiput(978.00,256.17)(0.500,1.000){2}{\rule{0.120pt}{0.400pt}}
\put(977.0,257.0){\usebox{\plotpoint}}
\put(986,257.67){\rule{0.241pt}{0.400pt}}
\multiput(986.00,257.17)(0.500,1.000){2}{\rule{0.120pt}{0.400pt}}
\put(979.0,258.0){\rule[-0.200pt]{1.686pt}{0.400pt}}
\put(988,258.67){\rule{0.241pt}{0.400pt}}
\multiput(988.00,258.17)(0.500,1.000){2}{\rule{0.120pt}{0.400pt}}
\put(987.0,259.0){\usebox{\plotpoint}}
\put(989.0,260.0){\usebox{\plotpoint}}
\put(990.0,260.0){\usebox{\plotpoint}}
\put(990.0,261.0){\rule[-0.200pt]{0.482pt}{0.400pt}}
\put(992.0,261.0){\usebox{\plotpoint}}
\put(994,260.67){\rule{0.241pt}{0.400pt}}
\multiput(994.00,261.17)(0.500,-1.000){2}{\rule{0.120pt}{0.400pt}}
\put(995,260.67){\rule{0.241pt}{0.400pt}}
\multiput(995.00,260.17)(0.500,1.000){2}{\rule{0.120pt}{0.400pt}}
\put(992.0,262.0){\rule[-0.200pt]{0.482pt}{0.400pt}}
\put(1002,261.67){\rule{0.241pt}{0.400pt}}
\multiput(1002.00,261.17)(0.500,1.000){2}{\rule{0.120pt}{0.400pt}}
\put(996.0,262.0){\rule[-0.200pt]{1.445pt}{0.400pt}}
\put(1003,263){\usebox{\plotpoint}}
\put(1003,262.67){\rule{0.241pt}{0.400pt}}
\multiput(1003.00,262.17)(0.500,1.000){2}{\rule{0.120pt}{0.400pt}}
\put(1009,263.67){\rule{0.241pt}{0.400pt}}
\multiput(1009.00,263.17)(0.500,1.000){2}{\rule{0.120pt}{0.400pt}}
\put(1004.0,264.0){\rule[-0.200pt]{1.204pt}{0.400pt}}
\put(1010,265){\usebox{\plotpoint}}
\put(1010,263.67){\rule{0.241pt}{0.400pt}}
\multiput(1010.00,264.17)(0.500,-1.000){2}{\rule{0.120pt}{0.400pt}}
\put(1016,263.67){\rule{0.241pt}{0.400pt}}
\multiput(1016.00,263.17)(0.500,1.000){2}{\rule{0.120pt}{0.400pt}}
\put(1011.0,264.0){\rule[-0.200pt]{1.204pt}{0.400pt}}
\put(1020,264.67){\rule{0.241pt}{0.400pt}}
\multiput(1020.00,264.17)(0.500,1.000){2}{\rule{0.120pt}{0.400pt}}
\put(1017.0,265.0){\rule[-0.200pt]{0.723pt}{0.400pt}}
\put(1021.0,266.0){\rule[-0.200pt]{0.964pt}{0.400pt}}
\put(1025.0,266.0){\usebox{\plotpoint}}
\put(1025.0,266.0){\usebox{\plotpoint}}
\put(1025.0,266.0){\usebox{\plotpoint}}
\put(1026.0,266.0){\usebox{\plotpoint}}
\put(1026,265.67){\rule{0.241pt}{0.400pt}}
\multiput(1026.00,265.17)(0.500,1.000){2}{\rule{0.120pt}{0.400pt}}
\put(1027,266.67){\rule{0.241pt}{0.400pt}}
\multiput(1027.00,266.17)(0.500,1.000){2}{\rule{0.120pt}{0.400pt}}
\put(1026.0,266.0){\usebox{\plotpoint}}
\put(1028,268){\usebox{\plotpoint}}
\put(1028,268){\usebox{\plotpoint}}
\put(1028,268){\usebox{\plotpoint}}
\put(1028,268){\usebox{\plotpoint}}
\put(1028,268){\usebox{\plotpoint}}
\put(1028,268){\usebox{\plotpoint}}
\put(1028,268){\usebox{\plotpoint}}
\put(1028,268){\usebox{\plotpoint}}
\put(1028,268){\usebox{\plotpoint}}
\put(1028,268){\usebox{\plotpoint}}
\put(1028,268){\usebox{\plotpoint}}
\put(1028,268){\usebox{\plotpoint}}
\put(1028,268){\usebox{\plotpoint}}
\put(1028,268){\usebox{\plotpoint}}
\put(1028,268){\usebox{\plotpoint}}
\put(1028,268){\usebox{\plotpoint}}
\put(1028,268){\usebox{\plotpoint}}
\put(1028,268){\usebox{\plotpoint}}
\put(1028,268){\usebox{\plotpoint}}
\put(1028,268){\usebox{\plotpoint}}
\put(1028,268){\usebox{\plotpoint}}
\put(1028,268){\usebox{\plotpoint}}
\put(1028,268){\usebox{\plotpoint}}
\put(1029,267.67){\rule{0.241pt}{0.400pt}}
\multiput(1029.00,267.17)(0.500,1.000){2}{\rule{0.120pt}{0.400pt}}
\put(1028.0,268.0){\usebox{\plotpoint}}
\put(1030,269){\usebox{\plotpoint}}
\put(1031,268.67){\rule{0.241pt}{0.400pt}}
\multiput(1031.00,268.17)(0.500,1.000){2}{\rule{0.120pt}{0.400pt}}
\put(1030.0,269.0){\usebox{\plotpoint}}
\put(1032.0,270.0){\rule[-0.200pt]{0.723pt}{0.400pt}}
\put(1035.0,270.0){\usebox{\plotpoint}}
\put(1035.0,271.0){\usebox{\plotpoint}}
\put(1036.0,271.0){\usebox{\plotpoint}}
\put(1038,271.67){\rule{0.241pt}{0.400pt}}
\multiput(1038.00,271.17)(0.500,1.000){2}{\rule{0.120pt}{0.400pt}}
\put(1036.0,272.0){\rule[-0.200pt]{0.482pt}{0.400pt}}
\put(1039,271.67){\rule{0.241pt}{0.400pt}}
\multiput(1039.00,271.17)(0.500,1.000){2}{\rule{0.120pt}{0.400pt}}
\put(1039.0,272.0){\usebox{\plotpoint}}
\put(1049,272.67){\rule{0.241pt}{0.400pt}}
\multiput(1049.00,272.17)(0.500,1.000){2}{\rule{0.120pt}{0.400pt}}
\put(1040.0,273.0){\rule[-0.200pt]{2.168pt}{0.400pt}}
\put(1051,273.67){\rule{0.241pt}{0.400pt}}
\multiput(1051.00,273.17)(0.500,1.000){2}{\rule{0.120pt}{0.400pt}}
\put(1050.0,274.0){\usebox{\plotpoint}}
\put(1052,275){\usebox{\plotpoint}}
\put(1052,275){\usebox{\plotpoint}}
\put(1052,275){\usebox{\plotpoint}}
\put(1052,275){\usebox{\plotpoint}}
\put(1052,275){\usebox{\plotpoint}}
\put(1052,275){\usebox{\plotpoint}}
\put(1052,275){\usebox{\plotpoint}}
\put(1053,274.67){\rule{0.241pt}{0.400pt}}
\multiput(1053.00,274.17)(0.500,1.000){2}{\rule{0.120pt}{0.400pt}}
\put(1054,275.67){\rule{0.241pt}{0.400pt}}
\multiput(1054.00,275.17)(0.500,1.000){2}{\rule{0.120pt}{0.400pt}}
\put(1052.0,275.0){\usebox{\plotpoint}}
\put(1056,275.67){\rule{0.241pt}{0.400pt}}
\multiput(1056.00,276.17)(0.500,-1.000){2}{\rule{0.120pt}{0.400pt}}
\put(1056.67,276){\rule{0.400pt}{0.482pt}}
\multiput(1056.17,276.00)(1.000,1.000){2}{\rule{0.400pt}{0.241pt}}
\put(1055.0,277.0){\usebox{\plotpoint}}
\put(1059,277.67){\rule{0.241pt}{0.400pt}}
\multiput(1059.00,277.17)(0.500,1.000){2}{\rule{0.120pt}{0.400pt}}
\put(1060,278.67){\rule{0.241pt}{0.400pt}}
\multiput(1060.00,278.17)(0.500,1.000){2}{\rule{0.120pt}{0.400pt}}
\put(1058.0,278.0){\usebox{\plotpoint}}
\put(1061.0,280.0){\rule[-0.200pt]{0.482pt}{0.400pt}}
\put(1063.0,280.0){\usebox{\plotpoint}}
\put(1065,280.67){\rule{0.241pt}{0.400pt}}
\multiput(1065.00,280.17)(0.500,1.000){2}{\rule{0.120pt}{0.400pt}}
\put(1063.0,281.0){\rule[-0.200pt]{0.482pt}{0.400pt}}
\put(1067,281.67){\rule{0.241pt}{0.400pt}}
\multiput(1067.00,281.17)(0.500,1.000){2}{\rule{0.120pt}{0.400pt}}
\put(1068,282.67){\rule{0.241pt}{0.400pt}}
\multiput(1068.00,282.17)(0.500,1.000){2}{\rule{0.120pt}{0.400pt}}
\put(1066.0,282.0){\usebox{\plotpoint}}
\put(1069,284.67){\rule{0.241pt}{0.400pt}}
\multiput(1069.00,284.17)(0.500,1.000){2}{\rule{0.120pt}{0.400pt}}
\put(1070,285.67){\rule{0.241pt}{0.400pt}}
\multiput(1070.00,285.17)(0.500,1.000){2}{\rule{0.120pt}{0.400pt}}
\put(1071,286.67){\rule{0.241pt}{0.400pt}}
\multiput(1071.00,286.17)(0.500,1.000){2}{\rule{0.120pt}{0.400pt}}
\put(1069.0,284.0){\usebox{\plotpoint}}
\put(1074,287.67){\rule{0.241pt}{0.400pt}}
\multiput(1074.00,287.17)(0.500,1.000){2}{\rule{0.120pt}{0.400pt}}
\put(1072.0,288.0){\rule[-0.200pt]{0.482pt}{0.400pt}}
\put(1077,288.67){\rule{0.241pt}{0.400pt}}
\multiput(1077.00,288.17)(0.500,1.000){2}{\rule{0.120pt}{0.400pt}}
\put(1075.0,289.0){\rule[-0.200pt]{0.482pt}{0.400pt}}
\put(1079,289.67){\rule{0.241pt}{0.400pt}}
\multiput(1079.00,289.17)(0.500,1.000){2}{\rule{0.120pt}{0.400pt}}
\put(1078.0,290.0){\usebox{\plotpoint}}
\put(1081,290.67){\rule{0.241pt}{0.400pt}}
\multiput(1081.00,290.17)(0.500,1.000){2}{\rule{0.120pt}{0.400pt}}
\put(1080.0,291.0){\usebox{\plotpoint}}
\put(1082,292){\usebox{\plotpoint}}
\put(1082,291.67){\rule{0.241pt}{0.400pt}}
\multiput(1082.00,291.17)(0.500,1.000){2}{\rule{0.120pt}{0.400pt}}
\put(1085,292.67){\rule{0.241pt}{0.400pt}}
\multiput(1085.00,292.17)(0.500,1.000){2}{\rule{0.120pt}{0.400pt}}
\put(1086,293.67){\rule{0.241pt}{0.400pt}}
\multiput(1086.00,293.17)(0.500,1.000){2}{\rule{0.120pt}{0.400pt}}
\put(1083.0,293.0){\rule[-0.200pt]{0.482pt}{0.400pt}}
\put(1087.0,295.0){\usebox{\plotpoint}}
\put(1088.0,295.0){\usebox{\plotpoint}}
\put(1090,295.67){\rule{0.241pt}{0.400pt}}
\multiput(1090.00,295.17)(0.500,1.000){2}{\rule{0.120pt}{0.400pt}}
\put(1088.0,296.0){\rule[-0.200pt]{0.482pt}{0.400pt}}
\put(1093,296.67){\rule{0.241pt}{0.400pt}}
\multiput(1093.00,296.17)(0.500,1.000){2}{\rule{0.120pt}{0.400pt}}
\put(1091.0,297.0){\rule[-0.200pt]{0.482pt}{0.400pt}}
\put(1094.0,298.0){\usebox{\plotpoint}}
\put(1095.0,298.0){\usebox{\plotpoint}}
\put(1095.0,299.0){\usebox{\plotpoint}}
\put(1096.0,299.0){\usebox{\plotpoint}}
\put(1096.0,299.0){\usebox{\plotpoint}}
\put(1096.0,299.0){\usebox{\plotpoint}}
\put(1096.0,300.0){\rule[-0.200pt]{1.204pt}{0.400pt}}
\put(1101.0,300.0){\usebox{\plotpoint}}
\put(1101.0,301.0){\usebox{\plotpoint}}
\put(1102,301.67){\rule{0.241pt}{0.400pt}}
\multiput(1102.00,301.17)(0.500,1.000){2}{\rule{0.120pt}{0.400pt}}
\put(1102.0,301.0){\usebox{\plotpoint}}
\put(1103.0,303.0){\usebox{\plotpoint}}
\put(1104.0,303.0){\usebox{\plotpoint}}
\put(1106.67,304){\rule{0.400pt}{0.482pt}}
\multiput(1106.17,304.00)(1.000,1.000){2}{\rule{0.400pt}{0.241pt}}
\put(1104.0,304.0){\rule[-0.200pt]{0.723pt}{0.400pt}}
\put(1108.0,306.0){\usebox{\plotpoint}}
\put(1109,306.67){\rule{0.241pt}{0.400pt}}
\multiput(1109.00,306.17)(0.500,1.000){2}{\rule{0.120pt}{0.400pt}}
\put(1110,307.67){\rule{0.241pt}{0.400pt}}
\multiput(1110.00,307.17)(0.500,1.000){2}{\rule{0.120pt}{0.400pt}}
\put(1111,308.67){\rule{0.241pt}{0.400pt}}
\multiput(1111.00,308.17)(0.500,1.000){2}{\rule{0.120pt}{0.400pt}}
\put(1112,309.67){\rule{0.241pt}{0.400pt}}
\multiput(1112.00,309.17)(0.500,1.000){2}{\rule{0.120pt}{0.400pt}}
\put(1113,310.67){\rule{0.241pt}{0.400pt}}
\multiput(1113.00,310.17)(0.500,1.000){2}{\rule{0.120pt}{0.400pt}}
\put(1109.0,306.0){\usebox{\plotpoint}}
\put(1114.0,312.0){\usebox{\plotpoint}}
\put(1115,312.67){\rule{0.241pt}{0.400pt}}
\multiput(1115.00,312.17)(0.500,1.000){2}{\rule{0.120pt}{0.400pt}}
\put(1115.0,312.0){\usebox{\plotpoint}}
\put(1118,313.67){\rule{0.241pt}{0.400pt}}
\multiput(1118.00,313.17)(0.500,1.000){2}{\rule{0.120pt}{0.400pt}}
\put(1116.0,314.0){\rule[-0.200pt]{0.482pt}{0.400pt}}
\put(1120,313.67){\rule{0.241pt}{0.400pt}}
\multiput(1120.00,314.17)(0.500,-1.000){2}{\rule{0.120pt}{0.400pt}}
\put(1119.0,315.0){\usebox{\plotpoint}}
\put(1121,314){\usebox{\plotpoint}}
\put(1121.0,314.0){\usebox{\plotpoint}}
\put(1121.0,314.0){\usebox{\plotpoint}}
\put(1121.0,314.0){\usebox{\plotpoint}}
\put(1122.0,314.0){\usebox{\plotpoint}}
\put(1123,314.67){\rule{0.241pt}{0.400pt}}
\multiput(1123.00,314.17)(0.500,1.000){2}{\rule{0.120pt}{0.400pt}}
\put(1124,315.67){\rule{0.241pt}{0.400pt}}
\multiput(1124.00,315.17)(0.500,1.000){2}{\rule{0.120pt}{0.400pt}}
\put(1122.0,315.0){\usebox{\plotpoint}}
\put(1125.0,317.0){\rule[-0.200pt]{0.723pt}{0.400pt}}
\put(1128.0,317.0){\usebox{\plotpoint}}
\put(1128.0,317.0){\usebox{\plotpoint}}
\put(1128.0,317.0){\usebox{\plotpoint}}
\put(1129.0,317.0){\usebox{\plotpoint}}
\put(1129.0,318.0){\usebox{\plotpoint}}
\put(1130.0,318.0){\usebox{\plotpoint}}
\put(1132,318.67){\rule{0.241pt}{0.400pt}}
\multiput(1132.00,318.17)(0.500,1.000){2}{\rule{0.120pt}{0.400pt}}
\put(1130.0,319.0){\rule[-0.200pt]{0.482pt}{0.400pt}}
\put(1134,319.67){\rule{0.241pt}{0.400pt}}
\multiput(1134.00,319.17)(0.500,1.000){2}{\rule{0.120pt}{0.400pt}}
\put(1133.0,320.0){\usebox{\plotpoint}}
\put(1135,321){\usebox{\plotpoint}}
\put(1135,320.67){\rule{0.241pt}{0.400pt}}
\multiput(1135.00,320.17)(0.500,1.000){2}{\rule{0.120pt}{0.400pt}}
\put(1138,321.67){\rule{0.241pt}{0.400pt}}
\multiput(1138.00,321.17)(0.500,1.000){2}{\rule{0.120pt}{0.400pt}}
\put(1136.0,322.0){\rule[-0.200pt]{0.482pt}{0.400pt}}
\put(1141,322.67){\rule{0.241pt}{0.400pt}}
\multiput(1141.00,322.17)(0.500,1.000){2}{\rule{0.120pt}{0.400pt}}
\put(1139.0,323.0){\rule[-0.200pt]{0.482pt}{0.400pt}}
\put(1142,324){\usebox{\plotpoint}}
\put(1142,323.67){\rule{0.241pt}{0.400pt}}
\multiput(1142.00,323.17)(0.500,1.000){2}{\rule{0.120pt}{0.400pt}}
\put(1144,324.67){\rule{0.241pt}{0.400pt}}
\multiput(1144.00,324.17)(0.500,1.000){2}{\rule{0.120pt}{0.400pt}}
\put(1143.0,325.0){\usebox{\plotpoint}}
\put(1148,325.67){\rule{0.241pt}{0.400pt}}
\multiput(1148.00,325.17)(0.500,1.000){2}{\rule{0.120pt}{0.400pt}}
\put(1145.0,326.0){\rule[-0.200pt]{0.723pt}{0.400pt}}
\put(1149,327){\usebox{\plotpoint}}
\put(1150,326.67){\rule{0.241pt}{0.400pt}}
\multiput(1150.00,326.17)(0.500,1.000){2}{\rule{0.120pt}{0.400pt}}
\put(1149.0,327.0){\usebox{\plotpoint}}
\put(1151.0,328.0){\usebox{\plotpoint}}
\put(1152.0,328.0){\usebox{\plotpoint}}
\put(1152.0,329.0){\usebox{\plotpoint}}
\put(1153.0,328.0){\usebox{\plotpoint}}
\put(1153.0,328.0){\usebox{\plotpoint}}
\put(1153.0,329.0){\rule[-0.200pt]{0.482pt}{0.400pt}}
\put(1155,328.67){\rule{0.241pt}{0.400pt}}
\multiput(1155.00,329.17)(0.500,-1.000){2}{\rule{0.120pt}{0.400pt}}
\put(1155.0,329.0){\usebox{\plotpoint}}
\put(1156,329){\usebox{\plotpoint}}
\put(1156,329){\usebox{\plotpoint}}
\put(1156,329){\usebox{\plotpoint}}
\put(1156,329){\usebox{\plotpoint}}
\put(1156,328.67){\rule{0.241pt}{0.400pt}}
\multiput(1156.00,329.17)(0.500,-1.000){2}{\rule{0.120pt}{0.400pt}}
\put(1156.0,329.0){\usebox{\plotpoint}}
\put(1157.0,329.0){\usebox{\plotpoint}}
\put(1157.0,330.0){\usebox{\plotpoint}}
\put(1158.0,330.0){\usebox{\plotpoint}}
\put(1158.0,331.0){\rule[-0.200pt]{0.482pt}{0.400pt}}
\put(1160.0,331.0){\usebox{\plotpoint}}
\put(1160.0,331.0){\usebox{\plotpoint}}
\put(1160.0,331.0){\usebox{\plotpoint}}
\put(1160.0,332.0){\rule[-0.200pt]{0.482pt}{0.400pt}}
\put(1162.0,332.0){\usebox{\plotpoint}}
\put(1164,332.67){\rule{0.241pt}{0.400pt}}
\multiput(1164.00,332.17)(0.500,1.000){2}{\rule{0.120pt}{0.400pt}}
\put(1162.0,333.0){\rule[-0.200pt]{0.482pt}{0.400pt}}
\put(1166,333.67){\rule{0.241pt}{0.400pt}}
\multiput(1166.00,333.17)(0.500,1.000){2}{\rule{0.120pt}{0.400pt}}
\put(1167,334.67){\rule{0.241pt}{0.400pt}}
\multiput(1167.00,334.17)(0.500,1.000){2}{\rule{0.120pt}{0.400pt}}
\put(1165.0,334.0){\usebox{\plotpoint}}
\put(1168,336){\usebox{\plotpoint}}
\put(1172,335.67){\rule{0.241pt}{0.400pt}}
\multiput(1172.00,335.17)(0.500,1.000){2}{\rule{0.120pt}{0.400pt}}
\put(1168.0,336.0){\rule[-0.200pt]{0.964pt}{0.400pt}}
\put(1173.0,337.0){\rule[-0.200pt]{0.482pt}{0.400pt}}
\put(1175.0,337.0){\usebox{\plotpoint}}
\put(1175.0,338.0){\usebox{\plotpoint}}
\put(1176.0,338.0){\usebox{\plotpoint}}
\put(1178,338.67){\rule{0.241pt}{0.400pt}}
\multiput(1178.00,338.17)(0.500,1.000){2}{\rule{0.120pt}{0.400pt}}
\put(1176.0,339.0){\rule[-0.200pt]{0.482pt}{0.400pt}}
\put(1181,339.67){\rule{0.241pt}{0.400pt}}
\multiput(1181.00,339.17)(0.500,1.000){2}{\rule{0.120pt}{0.400pt}}
\put(1179.0,340.0){\rule[-0.200pt]{0.482pt}{0.400pt}}
\put(1182,341){\usebox{\plotpoint}}
\put(1183,340.67){\rule{0.241pt}{0.400pt}}
\multiput(1183.00,340.17)(0.500,1.000){2}{\rule{0.120pt}{0.400pt}}
\put(1182.0,341.0){\usebox{\plotpoint}}
\put(1186,341.67){\rule{0.241pt}{0.400pt}}
\multiput(1186.00,341.17)(0.500,1.000){2}{\rule{0.120pt}{0.400pt}}
\put(1187,342.67){\rule{0.241pt}{0.400pt}}
\multiput(1187.00,342.17)(0.500,1.000){2}{\rule{0.120pt}{0.400pt}}
\put(1184.0,342.0){\rule[-0.200pt]{0.482pt}{0.400pt}}
\put(1188,344){\usebox{\plotpoint}}
\put(1189,342.67){\rule{0.241pt}{0.400pt}}
\multiput(1189.00,343.17)(0.500,-1.000){2}{\rule{0.120pt}{0.400pt}}
\put(1190,342.67){\rule{0.241pt}{0.400pt}}
\multiput(1190.00,342.17)(0.500,1.000){2}{\rule{0.120pt}{0.400pt}}
\put(1188.0,344.0){\usebox{\plotpoint}}
\put(1192,343.67){\rule{0.241pt}{0.400pt}}
\multiput(1192.00,343.17)(0.500,1.000){2}{\rule{0.120pt}{0.400pt}}
\put(1191.0,344.0){\usebox{\plotpoint}}
\put(1193.0,345.0){\usebox{\plotpoint}}
\put(1197,345.67){\rule{0.241pt}{0.400pt}}
\multiput(1197.00,345.17)(0.500,1.000){2}{\rule{0.120pt}{0.400pt}}
\put(1193.0,346.0){\rule[-0.200pt]{0.964pt}{0.400pt}}
\put(1198.0,347.0){\rule[-0.200pt]{0.723pt}{0.400pt}}
\put(1201,346.67){\rule{0.241pt}{0.400pt}}
\multiput(1201.00,347.17)(0.500,-1.000){2}{\rule{0.120pt}{0.400pt}}
\put(1202,346.67){\rule{0.241pt}{0.400pt}}
\multiput(1202.00,346.17)(0.500,1.000){2}{\rule{0.120pt}{0.400pt}}
\put(1201.0,347.0){\usebox{\plotpoint}}
\put(1203.0,348.0){\usebox{\plotpoint}}
\put(1204.0,348.0){\usebox{\plotpoint}}
\put(1204.0,348.0){\usebox{\plotpoint}}
\put(1205,347.67){\rule{0.241pt}{0.400pt}}
\multiput(1205.00,347.17)(0.500,1.000){2}{\rule{0.120pt}{0.400pt}}
\put(1204.0,348.0){\usebox{\plotpoint}}
\put(1206,349){\usebox{\plotpoint}}
\put(1206,349){\usebox{\plotpoint}}
\put(1206,349){\usebox{\plotpoint}}
\put(1206,349){\usebox{\plotpoint}}
\put(1206,349){\usebox{\plotpoint}}
\put(1206,349){\usebox{\plotpoint}}
\put(1206,349){\usebox{\plotpoint}}
\put(1206,349){\usebox{\plotpoint}}
\put(1206,349){\usebox{\plotpoint}}
\put(1206,349){\usebox{\plotpoint}}
\put(1208,348.67){\rule{0.241pt}{0.400pt}}
\multiput(1208.00,348.17)(0.500,1.000){2}{\rule{0.120pt}{0.400pt}}
\put(1209,348.67){\rule{0.241pt}{0.400pt}}
\multiput(1209.00,349.17)(0.500,-1.000){2}{\rule{0.120pt}{0.400pt}}
\put(1210,348.67){\rule{0.241pt}{0.400pt}}
\multiput(1210.00,348.17)(0.500,1.000){2}{\rule{0.120pt}{0.400pt}}
\put(1206.0,349.0){\rule[-0.200pt]{0.482pt}{0.400pt}}
\put(1214,349.67){\rule{0.241pt}{0.400pt}}
\multiput(1214.00,349.17)(0.500,1.000){2}{\rule{0.120pt}{0.400pt}}
\put(1211.0,350.0){\rule[-0.200pt]{0.723pt}{0.400pt}}
\put(1215,351){\usebox{\plotpoint}}
\put(1218,349.67){\rule{0.241pt}{0.400pt}}
\multiput(1218.00,350.17)(0.500,-1.000){2}{\rule{0.120pt}{0.400pt}}
\put(1215.0,351.0){\rule[-0.200pt]{0.723pt}{0.400pt}}
\put(1219.0,350.0){\rule[-0.200pt]{0.482pt}{0.400pt}}
\put(1221.0,349.0){\usebox{\plotpoint}}
\put(1222,348.67){\rule{0.241pt}{0.400pt}}
\multiput(1222.00,348.17)(0.500,1.000){2}{\rule{0.120pt}{0.400pt}}
\put(1223,349.67){\rule{0.241pt}{0.400pt}}
\multiput(1223.00,349.17)(0.500,1.000){2}{\rule{0.120pt}{0.400pt}}
\put(1221.0,349.0){\usebox{\plotpoint}}
\put(1227,350.67){\rule{0.241pt}{0.400pt}}
\multiput(1227.00,350.17)(0.500,1.000){2}{\rule{0.120pt}{0.400pt}}
\put(1224.0,351.0){\rule[-0.200pt]{0.723pt}{0.400pt}}
\put(1228.0,351.0){\usebox{\plotpoint}}
\put(1229,350.67){\rule{0.241pt}{0.400pt}}
\multiput(1229.00,350.17)(0.500,1.000){2}{\rule{0.120pt}{0.400pt}}
\put(1230,350.67){\rule{0.241pt}{0.400pt}}
\multiput(1230.00,351.17)(0.500,-1.000){2}{\rule{0.120pt}{0.400pt}}
\put(1228.0,351.0){\usebox{\plotpoint}}
\put(1232,350.67){\rule{0.241pt}{0.400pt}}
\multiput(1232.00,350.17)(0.500,1.000){2}{\rule{0.120pt}{0.400pt}}
\put(1231.0,351.0){\usebox{\plotpoint}}
\put(1233.0,352.0){\usebox{\plotpoint}}
\put(1234.0,352.0){\usebox{\plotpoint}}
\put(1235,351.67){\rule{0.241pt}{0.400pt}}
\multiput(1235.00,352.17)(0.500,-1.000){2}{\rule{0.120pt}{0.400pt}}
\put(1236,351.67){\rule{0.241pt}{0.400pt}}
\multiput(1236.00,351.17)(0.500,1.000){2}{\rule{0.120pt}{0.400pt}}
\put(1237,351.67){\rule{0.241pt}{0.400pt}}
\multiput(1237.00,352.17)(0.500,-1.000){2}{\rule{0.120pt}{0.400pt}}
\put(1238,350.67){\rule{0.241pt}{0.400pt}}
\multiput(1238.00,351.17)(0.500,-1.000){2}{\rule{0.120pt}{0.400pt}}
\put(1234.0,353.0){\usebox{\plotpoint}}
\put(1243,350.67){\rule{0.241pt}{0.400pt}}
\multiput(1243.00,350.17)(0.500,1.000){2}{\rule{0.120pt}{0.400pt}}
\put(1239.0,351.0){\rule[-0.200pt]{0.964pt}{0.400pt}}
\put(1245,350.67){\rule{0.241pt}{0.400pt}}
\multiput(1245.00,351.17)(0.500,-1.000){2}{\rule{0.120pt}{0.400pt}}
\put(1244.0,352.0){\usebox{\plotpoint}}
\put(1246.0,351.0){\rule[-0.200pt]{0.482pt}{0.400pt}}
\put(1248.0,351.0){\usebox{\plotpoint}}
\put(1248.0,352.0){\rule[-0.200pt]{1.445pt}{0.400pt}}
\put(1254.0,351.0){\usebox{\plotpoint}}
\put(1258,350.67){\rule{0.241pt}{0.400pt}}
\multiput(1258.00,350.17)(0.500,1.000){2}{\rule{0.120pt}{0.400pt}}
\put(1254.0,351.0){\rule[-0.200pt]{0.964pt}{0.400pt}}
\put(1259.0,352.0){\usebox{\plotpoint}}
\put(1260.0,351.0){\usebox{\plotpoint}}
\put(1260.0,351.0){\usebox{\plotpoint}}
\put(1261.0,350.0){\usebox{\plotpoint}}
\put(1262,348.67){\rule{0.241pt}{0.400pt}}
\multiput(1262.00,349.17)(0.500,-1.000){2}{\rule{0.120pt}{0.400pt}}
\put(1261.0,350.0){\usebox{\plotpoint}}
\put(1263,349){\usebox{\plotpoint}}
\put(1263,349){\usebox{\plotpoint}}
\put(1263,349){\usebox{\plotpoint}}
\put(1263,349){\usebox{\plotpoint}}
\put(1263,349){\usebox{\plotpoint}}
\put(1264,347.67){\rule{0.241pt}{0.400pt}}
\multiput(1264.00,348.17)(0.500,-1.000){2}{\rule{0.120pt}{0.400pt}}
\put(1263.0,349.0){\usebox{\plotpoint}}
\put(1270,346.67){\rule{0.241pt}{0.400pt}}
\multiput(1270.00,347.17)(0.500,-1.000){2}{\rule{0.120pt}{0.400pt}}
\put(1271,346.67){\rule{0.241pt}{0.400pt}}
\multiput(1271.00,346.17)(0.500,1.000){2}{\rule{0.120pt}{0.400pt}}
\put(1265.0,348.0){\rule[-0.200pt]{1.204pt}{0.400pt}}
\put(1276,346.67){\rule{0.241pt}{0.400pt}}
\multiput(1276.00,347.17)(0.500,-1.000){2}{\rule{0.120pt}{0.400pt}}
\put(1277,346.67){\rule{0.241pt}{0.400pt}}
\multiput(1277.00,346.17)(0.500,1.000){2}{\rule{0.120pt}{0.400pt}}
\put(1272.0,348.0){\rule[-0.200pt]{0.964pt}{0.400pt}}
\put(1280,346.67){\rule{0.241pt}{0.400pt}}
\multiput(1280.00,347.17)(0.500,-1.000){2}{\rule{0.120pt}{0.400pt}}
\put(1278.0,348.0){\rule[-0.200pt]{0.482pt}{0.400pt}}
\put(1281.0,346.0){\usebox{\plotpoint}}
\put(1284,344.67){\rule{0.241pt}{0.400pt}}
\multiput(1284.00,345.17)(0.500,-1.000){2}{\rule{0.120pt}{0.400pt}}
\put(1285,344.67){\rule{0.241pt}{0.400pt}}
\multiput(1285.00,344.17)(0.500,1.000){2}{\rule{0.120pt}{0.400pt}}
\put(1286,344.67){\rule{0.241pt}{0.400pt}}
\multiput(1286.00,345.17)(0.500,-1.000){2}{\rule{0.120pt}{0.400pt}}
\put(1281.0,346.0){\rule[-0.200pt]{0.723pt}{0.400pt}}
\put(1287,345){\usebox{\plotpoint}}
\put(1287,343.67){\rule{0.241pt}{0.400pt}}
\multiput(1287.00,344.17)(0.500,-1.000){2}{\rule{0.120pt}{0.400pt}}
\put(1289,342.67){\rule{0.241pt}{0.400pt}}
\multiput(1289.00,343.17)(0.500,-1.000){2}{\rule{0.120pt}{0.400pt}}
\put(1288.0,344.0){\usebox{\plotpoint}}
\put(1291,341.67){\rule{0.241pt}{0.400pt}}
\multiput(1291.00,342.17)(0.500,-1.000){2}{\rule{0.120pt}{0.400pt}}
\put(1290.0,343.0){\usebox{\plotpoint}}
\put(1292.0,342.0){\rule[-0.200pt]{0.482pt}{0.400pt}}
\put(1294.0,341.0){\usebox{\plotpoint}}
\put(1296,339.67){\rule{0.241pt}{0.400pt}}
\multiput(1296.00,340.17)(0.500,-1.000){2}{\rule{0.120pt}{0.400pt}}
\put(1294.0,341.0){\rule[-0.200pt]{0.482pt}{0.400pt}}
\put(1299,338.67){\rule{0.241pt}{0.400pt}}
\multiput(1299.00,339.17)(0.500,-1.000){2}{\rule{0.120pt}{0.400pt}}
\put(1300,338.67){\rule{0.241pt}{0.400pt}}
\multiput(1300.00,338.17)(0.500,1.000){2}{\rule{0.120pt}{0.400pt}}
\put(1297.0,340.0){\rule[-0.200pt]{0.482pt}{0.400pt}}
\put(1301.0,339.0){\usebox{\plotpoint}}
\put(1302,337.67){\rule{0.241pt}{0.400pt}}
\multiput(1302.00,338.17)(0.500,-1.000){2}{\rule{0.120pt}{0.400pt}}
\put(1303,336.67){\rule{0.241pt}{0.400pt}}
\multiput(1303.00,337.17)(0.500,-1.000){2}{\rule{0.120pt}{0.400pt}}
\put(1301.0,339.0){\usebox{\plotpoint}}
\put(1309,335.67){\rule{0.241pt}{0.400pt}}
\multiput(1309.00,336.17)(0.500,-1.000){2}{\rule{0.120pt}{0.400pt}}
\put(1304.0,337.0){\rule[-0.200pt]{1.204pt}{0.400pt}}
\put(1312,334.67){\rule{0.241pt}{0.400pt}}
\multiput(1312.00,335.17)(0.500,-1.000){2}{\rule{0.120pt}{0.400pt}}
\put(1310.0,336.0){\rule[-0.200pt]{0.482pt}{0.400pt}}
\put(1314,333.67){\rule{0.241pt}{0.400pt}}
\multiput(1314.00,334.17)(0.500,-1.000){2}{\rule{0.120pt}{0.400pt}}
\put(1315,332.67){\rule{0.241pt}{0.400pt}}
\multiput(1315.00,333.17)(0.500,-1.000){2}{\rule{0.120pt}{0.400pt}}
\put(1316,331.67){\rule{0.241pt}{0.400pt}}
\multiput(1316.00,332.17)(0.500,-1.000){2}{\rule{0.120pt}{0.400pt}}
\put(1313.0,335.0){\usebox{\plotpoint}}
\put(1317.0,332.0){\rule[-0.200pt]{0.723pt}{0.400pt}}
\put(1320.0,331.0){\usebox{\plotpoint}}
\put(1323,329.67){\rule{0.241pt}{0.400pt}}
\multiput(1323.00,330.17)(0.500,-1.000){2}{\rule{0.120pt}{0.400pt}}
\put(1320.0,331.0){\rule[-0.200pt]{0.723pt}{0.400pt}}
\put(1326,328.67){\rule{0.241pt}{0.400pt}}
\multiput(1326.00,329.17)(0.500,-1.000){2}{\rule{0.120pt}{0.400pt}}
\put(1324.0,330.0){\rule[-0.200pt]{0.482pt}{0.400pt}}
\put(1327,329){\usebox{\plotpoint}}
\put(1327,329){\usebox{\plotpoint}}
\put(1327,329){\usebox{\plotpoint}}
\put(1327,329){\usebox{\plotpoint}}
\put(1327,329){\usebox{\plotpoint}}
\put(1327,329){\usebox{\plotpoint}}
\put(1327,329){\usebox{\plotpoint}}
\put(1327,329){\usebox{\plotpoint}}
\put(1327,327.67){\rule{0.241pt}{0.400pt}}
\multiput(1327.00,328.17)(0.500,-1.000){2}{\rule{0.120pt}{0.400pt}}
\put(1328,326.67){\rule{0.241pt}{0.400pt}}
\multiput(1328.00,327.17)(0.500,-1.000){2}{\rule{0.120pt}{0.400pt}}
\put(1329,326.67){\rule{0.241pt}{0.400pt}}
\multiput(1329.00,326.17)(0.500,1.000){2}{\rule{0.120pt}{0.400pt}}
\put(1330,326.67){\rule{0.241pt}{0.400pt}}
\multiput(1330.00,327.17)(0.500,-1.000){2}{\rule{0.120pt}{0.400pt}}
\put(1331,325.67){\rule{0.241pt}{0.400pt}}
\multiput(1331.00,326.17)(0.500,-1.000){2}{\rule{0.120pt}{0.400pt}}
\put(1333,324.67){\rule{0.241pt}{0.400pt}}
\multiput(1333.00,325.17)(0.500,-1.000){2}{\rule{0.120pt}{0.400pt}}
\put(1332.0,326.0){\usebox{\plotpoint}}
\put(1334.0,324.0){\usebox{\plotpoint}}
\put(1334.0,324.0){\rule[-0.200pt]{1.445pt}{0.400pt}}
\put(1340,320.67){\rule{0.241pt}{0.400pt}}
\multiput(1340.00,321.17)(0.500,-1.000){2}{\rule{0.120pt}{0.400pt}}
\put(1340.0,322.0){\rule[-0.200pt]{0.400pt}{0.482pt}}
\put(1342,319.67){\rule{0.241pt}{0.400pt}}
\multiput(1342.00,320.17)(0.500,-1.000){2}{\rule{0.120pt}{0.400pt}}
\put(1343,318.67){\rule{0.241pt}{0.400pt}}
\multiput(1343.00,319.17)(0.500,-1.000){2}{\rule{0.120pt}{0.400pt}}
\put(1341.0,321.0){\usebox{\plotpoint}}
\put(1345,318.67){\rule{0.241pt}{0.400pt}}
\multiput(1345.00,318.17)(0.500,1.000){2}{\rule{0.120pt}{0.400pt}}
\put(1345.67,318){\rule{0.400pt}{0.482pt}}
\multiput(1345.17,319.00)(1.000,-1.000){2}{\rule{0.400pt}{0.241pt}}
\put(1344.0,319.0){\usebox{\plotpoint}}
\put(1347,318){\usebox{\plotpoint}}
\put(1349,316.67){\rule{0.241pt}{0.400pt}}
\multiput(1349.00,317.17)(0.500,-1.000){2}{\rule{0.120pt}{0.400pt}}
\put(1347.0,318.0){\rule[-0.200pt]{0.482pt}{0.400pt}}
\put(1350.0,317.0){\rule[-0.200pt]{0.723pt}{0.400pt}}
\put(1353,314.67){\rule{0.241pt}{0.400pt}}
\multiput(1353.00,315.17)(0.500,-1.000){2}{\rule{0.120pt}{0.400pt}}
\put(1354,314.67){\rule{0.241pt}{0.400pt}}
\multiput(1354.00,314.17)(0.500,1.000){2}{\rule{0.120pt}{0.400pt}}
\put(1355,314.67){\rule{0.241pt}{0.400pt}}
\multiput(1355.00,315.17)(0.500,-1.000){2}{\rule{0.120pt}{0.400pt}}
\put(1353.0,316.0){\usebox{\plotpoint}}
\put(1356,315){\usebox{\plotpoint}}
\put(1356,315){\usebox{\plotpoint}}
\put(1356,315){\usebox{\plotpoint}}
\put(1356.0,315.0){\rule[-0.200pt]{0.482pt}{0.400pt}}
\end{picture}

  \caption{The concentration of \protect$A\protect$ versus time. The
simulation shown consists of 8192 particles and the temperature is
\protect$1.4\protect$. The time is measured in unit of time
steps.\label{FigConc}} 
\end{figure}


Figure \ref{ArrhPlot} shows the rate constant for reaction \ref{Reac1}
versus the inverse tempera\-ture. Similar plots can be made for the two
other reactions. The estimated error bars are very small. Our estimates
are that they are at 4th or 5th significant digit \cite{Blocking}.

\begin{figure}[t]
  % GNUPLOT: LaTeX picture
\setlength{\unitlength}{0.240900pt}
\ifx\plotpoint\undefined\newsavebox{\plotpoint}\fi
\sbox{\plotpoint}{\rule[-0.200pt]{0.400pt}{0.400pt}}%
\begin{picture}(1500,675)(0,0)
\font\gnuplot=cmr10 at 10pt
\gnuplot
\sbox{\plotpoint}{\rule[-0.200pt]{0.400pt}{0.400pt}}%
\put(220.0,113.0){\rule[-0.200pt]{4.818pt}{0.400pt}}
\put(198,113){\makebox(0,0)[r]{0.001}}
\put(1416.0,113.0){\rule[-0.200pt]{4.818pt}{0.400pt}}
\put(220.0,167.0){\rule[-0.200pt]{2.409pt}{0.400pt}}
\put(1426.0,167.0){\rule[-0.200pt]{2.409pt}{0.400pt}}
\put(220.0,199.0){\rule[-0.200pt]{2.409pt}{0.400pt}}
\put(1426.0,199.0){\rule[-0.200pt]{2.409pt}{0.400pt}}
\put(220.0,221.0){\rule[-0.200pt]{2.409pt}{0.400pt}}
\put(1426.0,221.0){\rule[-0.200pt]{2.409pt}{0.400pt}}
\put(220.0,239.0){\rule[-0.200pt]{2.409pt}{0.400pt}}
\put(1426.0,239.0){\rule[-0.200pt]{2.409pt}{0.400pt}}
\put(220.0,253.0){\rule[-0.200pt]{2.409pt}{0.400pt}}
\put(1426.0,253.0){\rule[-0.200pt]{2.409pt}{0.400pt}}
\put(220.0,265.0){\rule[-0.200pt]{2.409pt}{0.400pt}}
\put(1426.0,265.0){\rule[-0.200pt]{2.409pt}{0.400pt}}
\put(220.0,275.0){\rule[-0.200pt]{2.409pt}{0.400pt}}
\put(1426.0,275.0){\rule[-0.200pt]{2.409pt}{0.400pt}}
\put(220.0,284.0){\rule[-0.200pt]{2.409pt}{0.400pt}}
\put(1426.0,284.0){\rule[-0.200pt]{2.409pt}{0.400pt}}
\put(220.0,293.0){\rule[-0.200pt]{4.818pt}{0.400pt}}
\put(198,293){\makebox(0,0)[r]{0.01}}
\put(1416.0,293.0){\rule[-0.200pt]{4.818pt}{0.400pt}}
\put(220.0,347.0){\rule[-0.200pt]{2.409pt}{0.400pt}}
\put(1426.0,347.0){\rule[-0.200pt]{2.409pt}{0.400pt}}
\put(220.0,378.0){\rule[-0.200pt]{2.409pt}{0.400pt}}
\put(1426.0,378.0){\rule[-0.200pt]{2.409pt}{0.400pt}}
\put(220.0,401.0){\rule[-0.200pt]{2.409pt}{0.400pt}}
\put(1426.0,401.0){\rule[-0.200pt]{2.409pt}{0.400pt}}
\put(220.0,418.0){\rule[-0.200pt]{2.409pt}{0.400pt}}
\put(1426.0,418.0){\rule[-0.200pt]{2.409pt}{0.400pt}}
\put(220.0,432.0){\rule[-0.200pt]{2.409pt}{0.400pt}}
\put(1426.0,432.0){\rule[-0.200pt]{2.409pt}{0.400pt}}
\put(220.0,445.0){\rule[-0.200pt]{2.409pt}{0.400pt}}
\put(1426.0,445.0){\rule[-0.200pt]{2.409pt}{0.400pt}}
\put(220.0,455.0){\rule[-0.200pt]{2.409pt}{0.400pt}}
\put(1426.0,455.0){\rule[-0.200pt]{2.409pt}{0.400pt}}
\put(220.0,464.0){\rule[-0.200pt]{2.409pt}{0.400pt}}
\put(1426.0,464.0){\rule[-0.200pt]{2.409pt}{0.400pt}}
\put(220.0,472.0){\rule[-0.200pt]{4.818pt}{0.400pt}}
\put(198,472){\makebox(0,0)[r]{0.1}}
\put(1416.0,472.0){\rule[-0.200pt]{4.818pt}{0.400pt}}
\put(220.0,526.0){\rule[-0.200pt]{2.409pt}{0.400pt}}
\put(1426.0,526.0){\rule[-0.200pt]{2.409pt}{0.400pt}}
\put(220.0,558.0){\rule[-0.200pt]{2.409pt}{0.400pt}}
\put(1426.0,558.0){\rule[-0.200pt]{2.409pt}{0.400pt}}
\put(220.0,581.0){\rule[-0.200pt]{2.409pt}{0.400pt}}
\put(1426.0,581.0){\rule[-0.200pt]{2.409pt}{0.400pt}}
\put(220.0,598.0){\rule[-0.200pt]{2.409pt}{0.400pt}}
\put(1426.0,598.0){\rule[-0.200pt]{2.409pt}{0.400pt}}
\put(220.0,612.0){\rule[-0.200pt]{2.409pt}{0.400pt}}
\put(1426.0,612.0){\rule[-0.200pt]{2.409pt}{0.400pt}}
\put(220.0,624.0){\rule[-0.200pt]{2.409pt}{0.400pt}}
\put(1426.0,624.0){\rule[-0.200pt]{2.409pt}{0.400pt}}
\put(220.0,635.0){\rule[-0.200pt]{2.409pt}{0.400pt}}
\put(1426.0,635.0){\rule[-0.200pt]{2.409pt}{0.400pt}}
\put(220.0,644.0){\rule[-0.200pt]{2.409pt}{0.400pt}}
\put(1426.0,644.0){\rule[-0.200pt]{2.409pt}{0.400pt}}
\put(220.0,652.0){\rule[-0.200pt]{4.818pt}{0.400pt}}
\put(198,652){\makebox(0,0)[r]{1}}
\put(1416.0,652.0){\rule[-0.200pt]{4.818pt}{0.400pt}}
\put(220.0,113.0){\rule[-0.200pt]{0.400pt}{4.818pt}}
\put(220,68){\makebox(0,0){0.1}}
\put(220.0,632.0){\rule[-0.200pt]{0.400pt}{4.818pt}}
\put(355.0,113.0){\rule[-0.200pt]{0.400pt}{4.818pt}}
\put(355,68){\makebox(0,0){0.2}}
\put(355.0,632.0){\rule[-0.200pt]{0.400pt}{4.818pt}}
\put(490.0,113.0){\rule[-0.200pt]{0.400pt}{4.818pt}}
\put(490,68){\makebox(0,0){0.3}}
\put(490.0,632.0){\rule[-0.200pt]{0.400pt}{4.818pt}}
\put(625.0,113.0){\rule[-0.200pt]{0.400pt}{4.818pt}}
\put(625,68){\makebox(0,0){0.4}}
\put(625.0,632.0){\rule[-0.200pt]{0.400pt}{4.818pt}}
\put(760.0,113.0){\rule[-0.200pt]{0.400pt}{4.818pt}}
\put(760,68){\makebox(0,0){0.5}}
\put(760.0,632.0){\rule[-0.200pt]{0.400pt}{4.818pt}}
\put(896.0,113.0){\rule[-0.200pt]{0.400pt}{4.818pt}}
\put(896,68){\makebox(0,0){0.6}}
\put(896.0,632.0){\rule[-0.200pt]{0.400pt}{4.818pt}}
\put(1031.0,113.0){\rule[-0.200pt]{0.400pt}{4.818pt}}
\put(1031,68){\makebox(0,0){0.7}}
\put(1031.0,632.0){\rule[-0.200pt]{0.400pt}{4.818pt}}
\put(1166.0,113.0){\rule[-0.200pt]{0.400pt}{4.818pt}}
\put(1166,68){\makebox(0,0){0.8}}
\put(1166.0,632.0){\rule[-0.200pt]{0.400pt}{4.818pt}}
\put(1301.0,113.0){\rule[-0.200pt]{0.400pt}{4.818pt}}
\put(1301,68){\makebox(0,0){0.9}}
\put(1301.0,632.0){\rule[-0.200pt]{0.400pt}{4.818pt}}
\put(1436.0,113.0){\rule[-0.200pt]{0.400pt}{4.818pt}}
\put(1436,68){\makebox(0,0){1}}
\put(1436.0,632.0){\rule[-0.200pt]{0.400pt}{4.818pt}}
\put(220.0,113.0){\rule[-0.200pt]{292.934pt}{0.400pt}}
\put(1436.0,113.0){\rule[-0.200pt]{0.400pt}{129.845pt}}
\put(220.0,652.0){\rule[-0.200pt]{292.934pt}{0.400pt}}
\put(45,382){\makebox(0,0){$k_1$}}
\put(828,23){\makebox(0,0){$1/T$}}
\put(220.0,113.0){\rule[-0.200pt]{0.400pt}{129.845pt}}
\put(1436,323){\raisebox{-.8pt}{\makebox(0,0){$\Diamond$}}}
\put(1211,344){\raisebox{-.8pt}{\makebox(0,0){$\Diamond$}}}
\put(1050,355){\raisebox{-.8pt}{\makebox(0,0){$\Diamond$}}}
\put(929,367){\raisebox{-.8pt}{\makebox(0,0){$\Diamond$}}}
\put(836,375){\raisebox{-.8pt}{\makebox(0,0){$\Diamond$}}}
\put(760,380){\raisebox{-.8pt}{\makebox(0,0){$\Diamond$}}}
\put(699,387){\raisebox{-.8pt}{\makebox(0,0){$\Diamond$}}}
\put(648,391){\raisebox{-.8pt}{\makebox(0,0){$\Diamond$}}}
\put(625,393){\raisebox{-.8pt}{\makebox(0,0){$\Diamond$}}}
\put(576,398){\raisebox{-.8pt}{\makebox(0,0){$\Diamond$}}}
\put(535,401){\raisebox{-.8pt}{\makebox(0,0){$\Diamond$}}}
\put(423,415){\raisebox{-.8pt}{\makebox(0,0){$\Diamond$}}}
\put(355,423){\raisebox{-.8pt}{\makebox(0,0){$\Diamond$}}}
\put(254,436){\raisebox{-.8pt}{\makebox(0,0){$\Diamond$}}}
\put(220,442){\raisebox{-.8pt}{\makebox(0,0){$\Diamond$}}}
\put(220,442){\rule{1pt}{1pt}}
\put(232,440){\rule{1pt}{1pt}}
\put(245,438){\rule{1pt}{1pt}}
\put(257,437){\rule{1pt}{1pt}}
\put(269,435){\rule{1pt}{1pt}}
\put(281,433){\rule{1pt}{1pt}}
\put(294,432){\rule{1pt}{1pt}}
\put(306,430){\rule{1pt}{1pt}}
\put(318,428){\rule{1pt}{1pt}}
\put(331,427){\rule{1pt}{1pt}}
\put(343,425){\rule{1pt}{1pt}}
\put(355,423){\rule{1pt}{1pt}}
\put(367,422){\rule{1pt}{1pt}}
\put(380,420){\rule{1pt}{1pt}}
\put(392,418){\rule{1pt}{1pt}}
\put(404,417){\rule{1pt}{1pt}}
\put(417,415){\rule{1pt}{1pt}}
\put(429,413){\rule{1pt}{1pt}}
\put(441,412){\rule{1pt}{1pt}}
\put(453,410){\rule{1pt}{1pt}}
\put(466,409){\rule{1pt}{1pt}}
\put(478,407){\rule{1pt}{1pt}}
\put(490,405){\rule{1pt}{1pt}}
\put(503,404){\rule{1pt}{1pt}}
\put(515,402){\rule{1pt}{1pt}}
\put(527,400){\rule{1pt}{1pt}}
\put(539,399){\rule{1pt}{1pt}}
\put(552,397){\rule{1pt}{1pt}}
\put(564,395){\rule{1pt}{1pt}}
\put(576,394){\rule{1pt}{1pt}}
\put(588,392){\rule{1pt}{1pt}}
\put(601,390){\rule{1pt}{1pt}}
\put(613,389){\rule{1pt}{1pt}}
\put(625,387){\rule{1pt}{1pt}}
\put(638,385){\rule{1pt}{1pt}}
\put(650,384){\rule{1pt}{1pt}}
\put(662,382){\rule{1pt}{1pt}}
\put(674,380){\rule{1pt}{1pt}}
\put(687,379){\rule{1pt}{1pt}}
\put(699,377){\rule{1pt}{1pt}}
\put(711,376){\rule{1pt}{1pt}}
\put(724,374){\rule{1pt}{1pt}}
\put(736,372){\rule{1pt}{1pt}}
\put(748,371){\rule{1pt}{1pt}}
\put(760,369){\rule{1pt}{1pt}}
\put(773,367){\rule{1pt}{1pt}}
\put(785,366){\rule{1pt}{1pt}}
\put(797,364){\rule{1pt}{1pt}}
\put(810,362){\rule{1pt}{1pt}}
\put(822,361){\rule{1pt}{1pt}}
\put(834,359){\rule{1pt}{1pt}}
\put(846,357){\rule{1pt}{1pt}}
\put(859,356){\rule{1pt}{1pt}}
\put(871,354){\rule{1pt}{1pt}}
\put(883,352){\rule{1pt}{1pt}}
\put(896,351){\rule{1pt}{1pt}}
\put(908,349){\rule{1pt}{1pt}}
\put(920,347){\rule{1pt}{1pt}}
\put(932,346){\rule{1pt}{1pt}}
\put(945,344){\rule{1pt}{1pt}}
\put(957,343){\rule{1pt}{1pt}}
\put(969,341){\rule{1pt}{1pt}}
\put(982,339){\rule{1pt}{1pt}}
\put(994,338){\rule{1pt}{1pt}}
\put(1006,336){\rule{1pt}{1pt}}
\put(1018,334){\rule{1pt}{1pt}}
\put(1031,333){\rule{1pt}{1pt}}
\put(1043,331){\rule{1pt}{1pt}}
\put(1055,329){\rule{1pt}{1pt}}
\put(1068,328){\rule{1pt}{1pt}}
\put(1080,326){\rule{1pt}{1pt}}
\put(1092,324){\rule{1pt}{1pt}}
\put(1104,323){\rule{1pt}{1pt}}
\put(1117,321){\rule{1pt}{1pt}}
\put(1129,319){\rule{1pt}{1pt}}
\put(1141,318){\rule{1pt}{1pt}}
\put(1153,316){\rule{1pt}{1pt}}
\put(1166,314){\rule{1pt}{1pt}}
\put(1178,313){\rule{1pt}{1pt}}
\put(1190,311){\rule{1pt}{1pt}}
\put(1203,310){\rule{1pt}{1pt}}
\put(1215,308){\rule{1pt}{1pt}}
\put(1227,306){\rule{1pt}{1pt}}
\put(1239,305){\rule{1pt}{1pt}}
\put(1252,303){\rule{1pt}{1pt}}
\put(1264,301){\rule{1pt}{1pt}}
\put(1276,300){\rule{1pt}{1pt}}
\put(1289,298){\rule{1pt}{1pt}}
\put(1301,296){\rule{1pt}{1pt}}
\put(1313,295){\rule{1pt}{1pt}}
\put(1325,293){\rule{1pt}{1pt}}
\put(1338,291){\rule{1pt}{1pt}}
\put(1350,290){\rule{1pt}{1pt}}
\put(1362,288){\rule{1pt}{1pt}}
\put(1375,286){\rule{1pt}{1pt}}
\put(1387,285){\rule{1pt}{1pt}}
\put(1399,283){\rule{1pt}{1pt}}
\put(1411,281){\rule{1pt}{1pt}}
\put(1424,280){\rule{1pt}{1pt}}
\put(1436,278){\rule{1pt}{1pt}}
\sbox{\plotpoint}{\rule[-0.400pt]{0.800pt}{0.800pt}}%
\put(220,428){\rule{1pt}{1pt}}
\put(232,427){\rule{1pt}{1pt}}
\put(245,426){\rule{1pt}{1pt}}
\put(257,425){\rule{1pt}{1pt}}
\put(269,424){\rule{1pt}{1pt}}
\put(281,423){\rule{1pt}{1pt}}
\put(294,422){\rule{1pt}{1pt}}
\put(306,421){\rule{1pt}{1pt}}
\put(318,420){\rule{1pt}{1pt}}
\put(331,419){\rule{1pt}{1pt}}
\put(343,417){\rule{1pt}{1pt}}
\put(355,416){\rule{1pt}{1pt}}
\put(367,415){\rule{1pt}{1pt}}
\put(380,414){\rule{1pt}{1pt}}
\put(392,413){\rule{1pt}{1pt}}
\put(404,412){\rule{1pt}{1pt}}
\put(417,411){\rule{1pt}{1pt}}
\put(429,410){\rule{1pt}{1pt}}
\put(441,409){\rule{1pt}{1pt}}
\put(453,408){\rule{1pt}{1pt}}
\put(466,407){\rule{1pt}{1pt}}
\put(478,406){\rule{1pt}{1pt}}
\put(490,405){\rule{1pt}{1pt}}
\put(503,404){\rule{1pt}{1pt}}
\put(515,403){\rule{1pt}{1pt}}
\put(527,402){\rule{1pt}{1pt}}
\put(539,401){\rule{1pt}{1pt}}
\put(552,399){\rule{1pt}{1pt}}
\put(564,398){\rule{1pt}{1pt}}
\put(576,397){\rule{1pt}{1pt}}
\put(588,396){\rule{1pt}{1pt}}
\put(601,395){\rule{1pt}{1pt}}
\put(613,394){\rule{1pt}{1pt}}
\put(625,393){\rule{1pt}{1pt}}
\put(638,392){\rule{1pt}{1pt}}
\put(650,391){\rule{1pt}{1pt}}
\put(662,390){\rule{1pt}{1pt}}
\put(674,389){\rule{1pt}{1pt}}
\put(687,388){\rule{1pt}{1pt}}
\put(699,387){\rule{1pt}{1pt}}
\put(711,386){\rule{1pt}{1pt}}
\put(724,385){\rule{1pt}{1pt}}
\put(736,384){\rule{1pt}{1pt}}
\put(748,383){\rule{1pt}{1pt}}
\put(760,381){\rule{1pt}{1pt}}
\put(773,380){\rule{1pt}{1pt}}
\put(785,379){\rule{1pt}{1pt}}
\put(797,378){\rule{1pt}{1pt}}
\put(810,377){\rule{1pt}{1pt}}
\put(822,376){\rule{1pt}{1pt}}
\put(834,375){\rule{1pt}{1pt}}
\put(846,374){\rule{1pt}{1pt}}
\put(859,373){\rule{1pt}{1pt}}
\put(871,372){\rule{1pt}{1pt}}
\put(883,371){\rule{1pt}{1pt}}
\put(896,370){\rule{1pt}{1pt}}
\put(908,369){\rule{1pt}{1pt}}
\put(920,368){\rule{1pt}{1pt}}
\put(932,367){\rule{1pt}{1pt}}
\put(945,366){\rule{1pt}{1pt}}
\put(957,365){\rule{1pt}{1pt}}
\put(969,363){\rule{1pt}{1pt}}
\put(982,362){\rule{1pt}{1pt}}
\put(994,361){\rule{1pt}{1pt}}
\put(1006,360){\rule{1pt}{1pt}}
\put(1018,359){\rule{1pt}{1pt}}
\put(1031,358){\rule{1pt}{1pt}}
\put(1043,357){\rule{1pt}{1pt}}
\put(1055,356){\rule{1pt}{1pt}}
\put(1068,355){\rule{1pt}{1pt}}
\put(1080,354){\rule{1pt}{1pt}}
\put(1092,353){\rule{1pt}{1pt}}
\put(1104,352){\rule{1pt}{1pt}}
\put(1117,351){\rule{1pt}{1pt}}
\put(1129,350){\rule{1pt}{1pt}}
\put(1141,349){\rule{1pt}{1pt}}
\put(1153,348){\rule{1pt}{1pt}}
\put(1166,347){\rule{1pt}{1pt}}
\put(1178,345){\rule{1pt}{1pt}}
\put(1190,344){\rule{1pt}{1pt}}
\put(1203,343){\rule{1pt}{1pt}}
\put(1215,342){\rule{1pt}{1pt}}
\put(1227,341){\rule{1pt}{1pt}}
\put(1239,340){\rule{1pt}{1pt}}
\put(1252,339){\rule{1pt}{1pt}}
\put(1264,338){\rule{1pt}{1pt}}
\put(1276,337){\rule{1pt}{1pt}}
\put(1289,336){\rule{1pt}{1pt}}
\put(1301,335){\rule{1pt}{1pt}}
\put(1313,334){\rule{1pt}{1pt}}
\put(1325,333){\rule{1pt}{1pt}}
\put(1338,332){\rule{1pt}{1pt}}
\put(1350,331){\rule{1pt}{1pt}}
\put(1362,330){\rule{1pt}{1pt}}
\put(1375,329){\rule{1pt}{1pt}}
\put(1387,328){\rule{1pt}{1pt}}
\put(1399,326){\rule{1pt}{1pt}}
\put(1411,325){\rule{1pt}{1pt}}
\put(1424,324){\rule{1pt}{1pt}}
\put(1436,323){\rule{1pt}{1pt}}
\sbox{\plotpoint}{\rule[-0.500pt]{1.000pt}{1.000pt}}%
\put(1436,486){\makebox(0,0){$+$}}
\put(1211,508){\makebox(0,0){$+$}}
\put(1050,519){\makebox(0,0){$+$}}
\put(929,528){\makebox(0,0){$+$}}
\put(835,537){\makebox(0,0){$+$}}
\put(760,543){\makebox(0,0){$+$}}
\put(625,555){\makebox(0,0){$+$}}
\put(535,563){\makebox(0,0){$+$}}
\put(355,583){\makebox(0,0){$+$}}
\put(254,598){\makebox(0,0){$+$}}
\put(220,603){\makebox(0,0){$+$}}
\sbox{\plotpoint}{\rule[-0.600pt]{1.200pt}{1.200pt}}%
\put(220,600){\rule{1pt}{1pt}}
\put(232,599){\rule{1pt}{1pt}}
\put(245,597){\rule{1pt}{1pt}}
\put(257,596){\rule{1pt}{1pt}}
\put(269,595){\rule{1pt}{1pt}}
\put(281,593){\rule{1pt}{1pt}}
\put(294,592){\rule{1pt}{1pt}}
\put(306,590){\rule{1pt}{1pt}}
\put(318,589){\rule{1pt}{1pt}}
\put(331,588){\rule{1pt}{1pt}}
\put(343,586){\rule{1pt}{1pt}}
\put(355,585){\rule{1pt}{1pt}}
\put(367,583){\rule{1pt}{1pt}}
\put(380,582){\rule{1pt}{1pt}}
\put(392,580){\rule{1pt}{1pt}}
\put(404,579){\rule{1pt}{1pt}}
\put(417,578){\rule{1pt}{1pt}}
\put(429,576){\rule{1pt}{1pt}}
\put(441,575){\rule{1pt}{1pt}}
\put(453,573){\rule{1pt}{1pt}}
\put(466,572){\rule{1pt}{1pt}}
\put(478,571){\rule{1pt}{1pt}}
\put(490,569){\rule{1pt}{1pt}}
\put(503,568){\rule{1pt}{1pt}}
\put(515,566){\rule{1pt}{1pt}}
\put(527,565){\rule{1pt}{1pt}}
\put(539,563){\rule{1pt}{1pt}}
\put(552,562){\rule{1pt}{1pt}}
\put(564,561){\rule{1pt}{1pt}}
\put(576,559){\rule{1pt}{1pt}}
\put(588,558){\rule{1pt}{1pt}}
\put(601,556){\rule{1pt}{1pt}}
\put(613,555){\rule{1pt}{1pt}}
\put(625,554){\rule{1pt}{1pt}}
\put(638,552){\rule{1pt}{1pt}}
\put(650,551){\rule{1pt}{1pt}}
\put(662,549){\rule{1pt}{1pt}}
\put(674,548){\rule{1pt}{1pt}}
\put(687,546){\rule{1pt}{1pt}}
\put(699,545){\rule{1pt}{1pt}}
\put(711,544){\rule{1pt}{1pt}}
\put(724,542){\rule{1pt}{1pt}}
\put(736,541){\rule{1pt}{1pt}}
\put(748,539){\rule{1pt}{1pt}}
\put(760,538){\rule{1pt}{1pt}}
\put(773,537){\rule{1pt}{1pt}}
\put(785,535){\rule{1pt}{1pt}}
\put(797,534){\rule{1pt}{1pt}}
\put(810,532){\rule{1pt}{1pt}}
\put(822,531){\rule{1pt}{1pt}}
\put(834,529){\rule{1pt}{1pt}}
\put(846,528){\rule{1pt}{1pt}}
\put(859,527){\rule{1pt}{1pt}}
\put(871,525){\rule{1pt}{1pt}}
\put(883,524){\rule{1pt}{1pt}}
\put(896,522){\rule{1pt}{1pt}}
\put(908,521){\rule{1pt}{1pt}}
\put(920,520){\rule{1pt}{1pt}}
\put(932,518){\rule{1pt}{1pt}}
\put(945,517){\rule{1pt}{1pt}}
\put(957,515){\rule{1pt}{1pt}}
\put(969,514){\rule{1pt}{1pt}}
\put(982,512){\rule{1pt}{1pt}}
\put(994,511){\rule{1pt}{1pt}}
\put(1006,510){\rule{1pt}{1pt}}
\put(1018,508){\rule{1pt}{1pt}}
\put(1031,507){\rule{1pt}{1pt}}
\put(1043,505){\rule{1pt}{1pt}}
\put(1055,504){\rule{1pt}{1pt}}
\put(1068,502){\rule{1pt}{1pt}}
\put(1080,501){\rule{1pt}{1pt}}
\put(1092,500){\rule{1pt}{1pt}}
\put(1104,498){\rule{1pt}{1pt}}
\put(1117,497){\rule{1pt}{1pt}}
\put(1129,495){\rule{1pt}{1pt}}
\put(1141,494){\rule{1pt}{1pt}}
\put(1153,493){\rule{1pt}{1pt}}
\put(1166,491){\rule{1pt}{1pt}}
\put(1178,490){\rule{1pt}{1pt}}
\put(1190,488){\rule{1pt}{1pt}}
\put(1203,487){\rule{1pt}{1pt}}
\put(1215,485){\rule{1pt}{1pt}}
\put(1227,484){\rule{1pt}{1pt}}
\put(1239,483){\rule{1pt}{1pt}}
\put(1252,481){\rule{1pt}{1pt}}
\put(1264,480){\rule{1pt}{1pt}}
\put(1276,478){\rule{1pt}{1pt}}
\put(1289,477){\rule{1pt}{1pt}}
\put(1301,476){\rule{1pt}{1pt}}
\put(1313,474){\rule{1pt}{1pt}}
\put(1325,473){\rule{1pt}{1pt}}
\put(1338,471){\rule{1pt}{1pt}}
\put(1350,470){\rule{1pt}{1pt}}
\put(1362,468){\rule{1pt}{1pt}}
\put(1375,467){\rule{1pt}{1pt}}
\put(1387,466){\rule{1pt}{1pt}}
\put(1399,464){\rule{1pt}{1pt}}
\put(1411,463){\rule{1pt}{1pt}}
\put(1424,461){\rule{1pt}{1pt}}
\put(1436,460){\rule{1pt}{1pt}}
\sbox{\plotpoint}{\rule[-0.500pt]{1.000pt}{1.000pt}}%
\put(220,586){\rule{1pt}{1pt}}
\put(232,585){\rule{1pt}{1pt}}
\put(245,584){\rule{1pt}{1pt}}
\put(257,583){\rule{1pt}{1pt}}
\put(269,582){\rule{1pt}{1pt}}
\put(281,581){\rule{1pt}{1pt}}
\put(294,580){\rule{1pt}{1pt}}
\put(306,579){\rule{1pt}{1pt}}
\put(318,578){\rule{1pt}{1pt}}
\put(331,577){\rule{1pt}{1pt}}
\put(343,576){\rule{1pt}{1pt}}
\put(355,575){\rule{1pt}{1pt}}
\put(367,574){\rule{1pt}{1pt}}
\put(380,573){\rule{1pt}{1pt}}
\put(392,572){\rule{1pt}{1pt}}
\put(404,571){\rule{1pt}{1pt}}
\put(417,570){\rule{1pt}{1pt}}
\put(429,569){\rule{1pt}{1pt}}
\put(441,568){\rule{1pt}{1pt}}
\put(453,567){\rule{1pt}{1pt}}
\put(466,566){\rule{1pt}{1pt}}
\put(478,565){\rule{1pt}{1pt}}
\put(490,564){\rule{1pt}{1pt}}
\put(503,563){\rule{1pt}{1pt}}
\put(515,562){\rule{1pt}{1pt}}
\put(527,562){\rule{1pt}{1pt}}
\put(539,561){\rule{1pt}{1pt}}
\put(552,560){\rule{1pt}{1pt}}
\put(564,559){\rule{1pt}{1pt}}
\put(576,558){\rule{1pt}{1pt}}
\put(588,557){\rule{1pt}{1pt}}
\put(601,556){\rule{1pt}{1pt}}
\put(613,555){\rule{1pt}{1pt}}
\put(625,554){\rule{1pt}{1pt}}
\put(638,553){\rule{1pt}{1pt}}
\put(650,552){\rule{1pt}{1pt}}
\put(662,551){\rule{1pt}{1pt}}
\put(674,550){\rule{1pt}{1pt}}
\put(687,549){\rule{1pt}{1pt}}
\put(699,548){\rule{1pt}{1pt}}
\put(711,547){\rule{1pt}{1pt}}
\put(724,546){\rule{1pt}{1pt}}
\put(736,545){\rule{1pt}{1pt}}
\put(748,544){\rule{1pt}{1pt}}
\put(760,543){\rule{1pt}{1pt}}
\put(773,542){\rule{1pt}{1pt}}
\put(785,541){\rule{1pt}{1pt}}
\put(797,540){\rule{1pt}{1pt}}
\put(810,539){\rule{1pt}{1pt}}
\put(822,538){\rule{1pt}{1pt}}
\put(834,537){\rule{1pt}{1pt}}
\put(846,536){\rule{1pt}{1pt}}
\put(859,535){\rule{1pt}{1pt}}
\put(871,534){\rule{1pt}{1pt}}
\put(883,533){\rule{1pt}{1pt}}
\put(896,532){\rule{1pt}{1pt}}
\put(908,531){\rule{1pt}{1pt}}
\put(920,530){\rule{1pt}{1pt}}
\put(932,529){\rule{1pt}{1pt}}
\put(945,528){\rule{1pt}{1pt}}
\put(957,527){\rule{1pt}{1pt}}
\put(969,526){\rule{1pt}{1pt}}
\put(982,525){\rule{1pt}{1pt}}
\put(994,524){\rule{1pt}{1pt}}
\put(1006,523){\rule{1pt}{1pt}}
\put(1018,522){\rule{1pt}{1pt}}
\put(1031,521){\rule{1pt}{1pt}}
\put(1043,520){\rule{1pt}{1pt}}
\put(1055,519){\rule{1pt}{1pt}}
\put(1068,518){\rule{1pt}{1pt}}
\put(1080,517){\rule{1pt}{1pt}}
\put(1092,516){\rule{1pt}{1pt}}
\put(1104,515){\rule{1pt}{1pt}}
\put(1117,514){\rule{1pt}{1pt}}
\put(1129,513){\rule{1pt}{1pt}}
\put(1141,512){\rule{1pt}{1pt}}
\put(1153,511){\rule{1pt}{1pt}}
\put(1166,510){\rule{1pt}{1pt}}
\put(1178,509){\rule{1pt}{1pt}}
\put(1190,508){\rule{1pt}{1pt}}
\put(1203,507){\rule{1pt}{1pt}}
\put(1215,506){\rule{1pt}{1pt}}
\put(1227,505){\rule{1pt}{1pt}}
\put(1239,504){\rule{1pt}{1pt}}
\put(1252,503){\rule{1pt}{1pt}}
\put(1264,502){\rule{1pt}{1pt}}
\put(1276,501){\rule{1pt}{1pt}}
\put(1289,500){\rule{1pt}{1pt}}
\put(1301,499){\rule{1pt}{1pt}}
\put(1313,498){\rule{1pt}{1pt}}
\put(1325,497){\rule{1pt}{1pt}}
\put(1338,496){\rule{1pt}{1pt}}
\put(1350,495){\rule{1pt}{1pt}}
\put(1362,494){\rule{1pt}{1pt}}
\put(1375,493){\rule{1pt}{1pt}}
\put(1387,493){\rule{1pt}{1pt}}
\put(1399,492){\rule{1pt}{1pt}}
\put(1411,491){\rule{1pt}{1pt}}
\put(1424,490){\rule{1pt}{1pt}}
\put(1436,489){\rule{1pt}{1pt}}
\end{picture}

  \caption{Arrhenius plot for the rate constant for reaction \ref{Reac1}.
  The lower curve is for 1024 particles and the upper one is for 8192
  particles. Note that the rate constants are not normalised according 
  to the number of particles and therefore the rate constants for the
  8192 particle simulations are 8 times larger. The dotted lines are
  the best fitted lines to the data in that particular temperature
  region.\label{ArrhPlot}}
\end{figure}

We notice there is a crossover at $T \approx 3$. The crossover seems not 
just to be a finite-size effect since we see it at simulations with 1024 and 
8192 particles.

We believe that this effect comes from the fact that the system goes from 
being phase separated to a homogeneous phase when we increase the 
temperature. In the homogeneous phase the
three reactions can occur at all places in the simulation box while below
the phase separating temperature, the reactions will mainly occur at the 
boundaries between clusters of, say, $X$ and $Y$. The critical
temperature for a similar system of a binary mixture is $T_{c} \approx
4.7$ \cite{tox3}. In the critical region the phase separation dynamics
is extremely slow; but well below this region two processes will
compete: diffusion (to the interfacial reaction zone) and
reaction. The figure shows that the activation energy in the low
temperature region is not the activation energy for the reaction but
the activation energy for combined process of diffusion and reaction.

\subsection*{Acknowledgements}
The authors wish to thank the National Science Foundation (in Denmark) for
computing time on the IBM SP2 at the National Computer Centre (Uni-C). We 
also wish to thank M.\ Laradji for his many advices.

\begin{thebibliography}{1}
\bibitem{toxvaerd} Toxvaerd, S., 1991, Mol.\ Phys.\ \textbf{72}(1), 159.
\bibitem{Blocking} Flyvbjerg, H.\ and Petersen, H.G., 1989, J.\ Chem.\ Phys.\
                   \textbf{91}(1), 461.
\bibitem{Laidler} Chemical Kinetics, 3rd edition. K.J.\ Laidler. HarperCollins,
                  1987.
\bibitem{glotzer} Glotzer, S.C., Stauffer, D., and Jan, N., 1994, Phys.\ Rev.\ 
                 Lett.\ \textbf{72}(26), 4109.
\bibitem{tox2} Toxvaerd, S., 1996, \textit{to appear in} Phys.\ Rev.\
               E \textbf{53}.
\bibitem{lotka1} Lotka, A., 1910, J.\ Phys.\ Chem.\ \textbf{14}, 271.
\bibitem{lotka2} Lotka, A., 1920, J.\ Am.\ Chem.\ Phys.\ \textbf{42},
                 1593.
\bibitem{tox3} Toxvaerd, S.\ and Velasco, E., 1995, Mol.\ Phys.\
               \textbf{86}, 845.\\
\end{thebibliography}




